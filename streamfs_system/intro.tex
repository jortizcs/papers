% \section{Outline}

% StreamFS analytical contributions:

% \begin{itemize}
% \item Managing physical data with entity-relationship and naming.
% \item Use entity-relationship graph to inform aggregate operations on the data
% \item Data processing at aggregation points in the entity-relationship graph	
% 	\begin{itemize}
% 	\item re-sampling
% 	\item aggregation
% 	\end{itemize}
% \end{itemize}


% \section{Introduction 1}
% %First opening
% The days of oblivious energy consumption is quickly coming to an end.  With the rise in earth's surface 
% temperature~\cite{b_earth}, due to green-house gas emissions, causing noticeable weather 
% changes~\cite{climatechange} and 
% increasing energy prices there's a pressing need to optimize and reduce our consumption of energy.
% With the proliferation of embedded and mobile sensors in the environment, we have access to energy
% information at a level of granularity that was previously unattainable.  Sensors provide physical
% data about a phenomenon at a particular location in space over time.  Collections of sensors give us
% a more holistic view of the phenomenon, providing views over larger and larger spatio-temporal intervals.

% However, there are fundamental
% challenges~\cite{jgray_sensorsdb} in collecting, organizing, and processing sensor data before we can extract 
% information we can use to live and operate more sustainably.  Analysis goes beyond spatio-temporal associations,
% they also include categorical and other dimensional groupings.  In this paper we describe StreamFS, an analytical
% framework that addresses some of the challenges in doing energy analytics.  With a combination of naming,
% entity-relationship data model, and use of the entity-relationship graph to do aggregation, we address these issues.

\section{Introduction}
StreamFS is a file system the represents the physical world as a collection of files.  

Contributions:
\begin{itemize}
\item StreamFS treats metadata as a first-class citizen, allowing queries across snapshots of the structured, semi-structured,
and timeseries data.

\item The StreamFS view of the world revolves around signals or streams.  Streams with history, both in value and in metadata.  
In StreamFS, a standard log file is decomposed into a collection of signals/streams files, whose associated pathnames and tags
are maintained and querable in time.

\item Job execution scheduling based to minimize calculation error, with backwards-updates for errors that occurred looking forward.
		Gives results with approximate errors.

\item Non append-only streaming output is handled as a function that minimizes historical error calculations.~\cite{syncsql}.
		You should use LCM to determine when the reverse scan and update will take place.  The other trigger could be 
		cumulative error -- if it gets too high, run the process again on the historical data and update.


\end{itemize}


\subsection{Sematics of a stream file}

\subsection{Inputting to a file with Regex and mapper}
Use a regular expression and a map file to parse incoming data and put it in a file(s).

\section{Introduction}% 2}
%Second opening
Buildings consume 40\% of the energy produced in the United States and nearly three quarters of 
the electricity produced~\cite{epabuildings}.  It is conjectured that as much as 30-80\% of their energy is 
wasted~\cite{waste_science, next10_waste}.  With such astounding figures, buildings are a clear optimization
target.  In order to optimize building performance, we need fine grained visibility into the energy flow
throughout the building.  With over 70\% of commercial buildings, 100,000 square feet or 
larger, having a building management system~\cite{cbecs2003}, the infrastructure to attain that level of 
visibility is available and widespread.  

Building management systems (BMS) consist of thousands of sensors
deployed throughout the entire building, measuring temperature, pressure, flow, and other physical phenomenon.
They collect data from each of these sensors and present them to the building manager to show the current
state of building and the system components within it. 
Despite this level of penetration and data availablity, energy-flow visibility is poor. 
This is mainly due to two reasons:

\begin{enumerate}
\item Building management system are designed for monitoring, not analytics.
\item Deep analysis requires careful meta/data management and processing to clean the sensor data.
\end{enumerate}
\vspace{0.08in}

In this paper we describe a system we have designed to address these short-comings.  StreamFS is an
analytical framework that uses an entity-relationship graph (ERG) to model the logical and physical objects and 
inter-relationships that inform the questions posed to the system about energy-flow.
StreamFS provides traditional \emph{slice and dice}, \emph{drill-down}, and \emph{roll-up} OLAP~\cite{olap} 
operations; which come naturally from the graphical structure.  It also informs 
our naming scheme, which we use to allow the user to see and manipulate the state of graph
to capture the underlying physical and logical relationships and consequently update aggregate
data calculations upstream.

% making clear to the user how the data are inter-related. 
% We also introduce a technique called \emph{dynamic aggregation}, that combines the graph
% with stream-data processing techniques commonly used in continuous query systems~\cite{tcq,stream},
% which allows us to maintain an accurate account of aggregate calculations even objects move
% or inter-relationships change over time.

% Our naming convention is hierarchically structured with support for linking between sub-trees.

% StreamFS allows users to choose points of aggregation in the entity-relationship graph.

% We describe our naming scheme which allows users to name arbitrary objects and annotate them with
% descriptive properties.  We use

Our naming scheme is hierarchically structured, like traditional filesystem naming, with support
for symbolic links, allowing arbitrary links between sub-trees.  We argue that this naming scheme
is crucial, as it exposes the inter-relationships which inform aggregation semantics intended
by the user.  StreamFS distinguishes between nodes that represent streaming data sources in the real-world
and those that do not.  Those that do not, however, can be tagged as aggregation points.  As part of the 
tagging processes, a user specifies the units of aggregation, with additional options for cleaning
and processing.

We use StreamFS to organize and support applications using building data from three different
buildings.  The first one is a 110,000 square foot, seven-story building, the second one is an eleven-story
250,000 square-foot building, and the third is a 150,000 square-foot eleven-story building.
Our contributions are:

\begin{itemize}
\item A naming scheme for physical objects and inter-relationship that is used to construct
		an entity-relationship graph.
\item Use of the entity-relationship graph to provide OLAP \emph{roll-up}, \emph{drill-down},
		and \emph{slice and dice} operations.
\item Show how sliding-window operations can be used on real-time data in combination with the entity-relationship
		graph to maintain accurate aggregates as the underlying objects and inter-relationships change.
\end{itemize}

\vspace{0.08in}

We also discuss how we deal with the fundamental challenges that come with sensor data.  Specifically, we 
address \emph{re-sampling} and \emph{processing models}.  The incoming data does not have a common
time source, so combining the signals meaningfully involves interpolation.  There are various options that we
provide for performing the interpolation, chosen by the user depending on the units of the data.  For example,
temperature data may involve fitting a heat model with the data to attain missing values in time.  In addition,
aggregation is done as a function of the underlying constituents: they can be combined arbritarily, by adding
subtracting, multiplying or dividing corresponding values.  We provide an interface to the user that
allows them to specify how to combine the aggregate signals as a function of the child nodes in the entity-graph.
Futhermore, they can filter the data by unit.  This kind of flexibility useful for visualizing
energy consumption over time.
















\subsection{Entity-relationship model}

%In the vast majority of cases, in-time answers are imperative.  
Tracking the operational energy consumption of a building requires the ability answer a series of questions
about energy flow -- energy data aggregated across multiple logical classes to determine how, where, and how 
much is being used.  Sometimes, it even involves extrapolating forward in time to estimate 
future consumption patterns that could influence immedaite decisions.  The ability to \emph{slice and dice} 
the data allows the analyst
to gain better insight into how the energy is being used, where it can be used more effectively, and how
to change the operation of the building -- through better equipment or activity scheduling -- 
in order to optimize and reduce its energy consumption.
% For building-oriented energy analytics applications the building manager and occupants typically want answers
% to the following set of questions:
Below is a typical list of questions:

\begin{enumerate}
\item How much energy is consumed in this room/floor/building?  On average?
\item What is the current power draw by this pump? cooling tower? heating sub-system?  Over
		the last month?
\item How much power is this device currently drawing? Over the last hour?
%\item What percentage of the total building power is being drawn by the plug-load devices? 
\item How much energy have I consumed today?  Versus yesterday?
\item How much energy does the computing equipment in this building consume?
%\item {\bf For all queries:} What's the trend over time?
\end{enumerate}
\vspace{0.08in}

Notice, these question span spatial, temporal, and other arbitrary aggregates -- some physical, some
categorical.  There is also
an implicit hierarchical aspect to the grouping, in some cases.  For example, there are many
rooms on a floor and many floors in a building.  Naturally, to answer the first question we can
aggregate the data from the room up to the whole building.  This hierarchical relationship
is not as evident in the HVAC sub-components specified in the second question.  However,
local hierarchically relationships \emph{do exist}.  For example, the cooling system consists
of the set of pumps, cooling towers, and condensers in the HVAC system that push condensor
fluid and water to remove heat from spaces in the building.

We can model this as a set of objects and inter-relationships which inform how
to \emph{drill-down}, \emph{roll-up}, and \emph{slice and dice} the data -- traditional OLAP operations.
The main difference between this setup and traditional OLAP is the underlying dynamics of the
inter-relationships: objects, particularly those meant to represent physical entities, are added and removed and 
their inter-relationships change over time.  \emph{The natural evolution of buildings and activities 
within them makes tracking energy-flow fundamentally challenging}.

In this paper, we show how the entity-relationship model~\cite{Chen76theentity-relationship} helps simplify 
this problem, both as an interface
to the user and a data structure for the aggregation processes.  We argue that the use of this model is a cleaner
fit for this application scenario because it captures important semantic information about the real-world;
facts critical for picking which questions to ask and how to answer them.  In contrast, it has been shown 
that a relational model loses this information~\cite{SenkoDB}.














\subsection{An example}
Lets examine the requirements for answering the first question.
A building is unware that there are rooms. Typically spaces in a building are called \emph{zones} and, 
at construction time, walls are added to make rooms within zones.  This makes rooms an abstract
entity, used to group associated items with respect to it.  It also means
%The basic control unit for the Heating, Ventilation, and Air conditioning (HVAC) system as well as the electrical 
%panels and plugs, is a zone, not a room.  
we typically do not have a single meter that is measuring the energy of a room; it
must be calculated from the set of energy-consuming constituents.

What are the energy consuming constituents of a typical room?  It is the set of energy-consumers that
are active within or onto the room.  Broadly, it consists of three things:

\begin{itemize}
\item Plug-loads
\item Lights
\item HVAC
\end{itemize}
\vspace{0.08in}

For simplicity of demonstration, lets consider only plug-loads.  In our construction of an entity-relationship
graph lets assume there are nodes for each plug-load item and each room.  For the room in question, the relationship
between the plug-loads and the room is child to parent, respectively.  The total energy consumed by
the plug-loads can be aggregated at the parent node, the room, so the user can query the room for
the total.  Over time, plug-loads are removed and added to/from the room, but the relationship does not
change.  This simplifies the query; to obtain the total consumption over time, the query need only
go to the room node.  The parent-child relationship informs which constituents to aggregate over time
to calculate the total.

\subsection{General Approach}
To realize this design we need to maintain the entity-relationship graph, present it to the user in a meaningful
way; allowing them to update it directly to capture physical state and relationship changes.  We also need to
use this graphical structure to direct data flow throughout the underlying network.  This allows us
to accurately maintain the running aggregates as the deployment and activities churn.

We present the graph to the user through a filesystem-like naming and linking mechanisms.  The combination of a
hierarhical naming scheme and support for symbolic links allows the user to access and manipualte underlying objects
and relationships.  Moreover, the underlying graph structure is overloaded with upstream communication mechanisms
and buffering to allow data to flow from the data-producing leave nodes to the aggregation-performing
parent nodes.  Furthermore, the buffering lets us deal with the streaming nature of data flow from the physical
world to StreamFS and lets us maintain a real-time view of energy flow in the system.
Traversing the graph provides a natural way for the user to implicitly execute the OLAP operations necessary 
to give the user the kind of insight into energy usage in the building necessary to understand, optimize and 
reduce it.

% Show how the OLAP operations are realized with this model
% Drill-down
% Roll up
% slice and dice
% pivot
% You should show how to reduce the entity-relationship model to an OLAP data cube.  It's
% not clear from your explanation thus far.

% The plug-load and lights are easiest.  The as-built construction plans (as-builts, for short) contain light
% fixtures in each room, and typically lights are only either on or off with a fix comsumption amount.  Plug-load
% meters can be attached to all the plugs, with period power-readings reported.  The HVAC is a non-trivial.
% It involves using a simple model to calculate the heat load in the room, so that we can determine how
% much energy is takes to remove that load to maintain a constant temperature.  For simplicity we will
% ignore the lights and the HVAC.  Once we have the list of plugs and their measurements in the that room we 
% need to add them together.  

% Notice the there's fundamentally a set of abstract and real, physical objects, classified under some physically
% meaningful, logically arbirtrary label called a room.  This grouping is indicative of a physical relationship and that
% physical relationship informs how we inter-related the physical measurements from each of the meters.
% This is why we chose to use an entity-relationship data model.  It's natural for the organization of our
% queries and semantic understanding of the physical relationships that we're trying to capture and learn
% more about.  Furthermore, the entity-relationship model has many advanatages over other models, such as the 
% relational data model, with respect to preserving semantic information~\cite{Chen76theentity-relationship};
% a property that we show is of utmost importance for the applications we wish to support.




% {\bf Keep it general, use building, mobile phone apps as a case studies}

% Energy analytics is emerging as an important area of computing related to recording, tracking, 
% and projecting energy usage data.  Applications range from personal energy accounting~\cite{personal} to buildings
% energy tracking~\cite{buildings} to whole cities~\cite{cities}.  Understanding energy use is critical 
% for optimizing and reducing our consumption of it.  It also involves dealing with fundamental challenges 
% with data collection, organization, cleaning, and processing; issues fundamental to processing sensor 
% data~\cite{jgray_sensorsdb}.

%Motivation -->  Challenges --> Why are these challenges different? -->  What does StreamFS offer?


