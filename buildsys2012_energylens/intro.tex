\section{Introduction}
The United States leads the world in per-capita energy consumption; 
our electricity use has consistently increased over the last 40 years~\cite{oecd2011}; and, other parts of the world are rising all 
too rapidly.  With the specter of climate change and the increasing cost of energy, we must explore new
ways for individuals to gain visibility and insight into their energy consumption in order to optimize and reduce it. 
With the increasing penetration of embedded sensors in the environment and
the continued rise in smartphone adoption, we see an opportunity for smartphones to bridge the physical world
to our computational infrastructure and provide an `energy lens' on the physical world.  

Our work examines how mobile phones can be used to construct an entity-relationship view of the physical world and combine it with 
streaming, energy-related 
sensor data in order to provide detailed energy-attribution analytics.  In order to enable these, we must first capture the entities
in the world and the relationships between them.  We limit the scope of the world to a single building domain.  We have 
designed and implemented an application, called the `Energy Lens', that allows building occupants to both inform us 
of things in the world and their inter-relationships and combines the information with streaming meter data.  This paper focuses
on the challenges of collecting this information, maintaining consistency, as relationships between entities change, and 
discusses various heuristics for avoiding and mechanisms for dealing with conflicts.

The use of mobile phones presents classical, fundamental challenges related to mobility.  Typically, mobility
refers to the phone as the person carrying it moves from place to place.  However, in the energy-attribution
context, we are also refering to the movement of energy-consuming objects.  Tracking their relationships to spaces 
and people is as important as tracking people.  In our deployment, we describe how we deal with \emph{both moving people and 
moving objects}.  We show that these historically difficult problems can be addressed relatively easily, if the proper infrastructure is 
in place.  We provide evidence that the approach is simple, incrementally deployable, and scalable.

Our system uses QR codes to tag things and locations in the physical world.  Once tagged, there are three types of interactions -- 
registration, linking, and scanning -- which establish important relationships.  Registration is the act of creating a virtual object 
to represent a physical one.  Linking captures the relationship between pairs of objects.  Scanning is the act of performing an item-lookup.
Each of these interactions requires a set of swiping gestures.  Linking requires two tag swipes while the other two actions
require a single tag swipe.  Internally, we maintain a \emph{entity-relationship graph} of things, people, and locations, that gets
updated through these sets of gestures.

In order to connect these components, we rely on having `ubiquitous' network connectivity.  However, in practice, network
connectivity can be intermittent and our system must deal with the challenges of intermittency.  We discuss how caching
and logging are used to address these challenges.  Moreoever, when connectivity is re-established, we must deal with
updating the entity-relationship graph to reflect the latest state of the world.  As such, we must address potential
conflicts while replaying logs.  Finally, certain physical-state transitions are represented as a set of updates to the entity-relationship
graph that must be applied atomically.  We use the notion of state-transition transactions, which is also used by our
log-replay and transaction manager.

Our `Energy Lens' system is deployed in a building on our campus.  We discuss
its archictecture and our design choices.  As the deployment is on-going, we evaluate the
transaction manager and conflict-resolution algorithms via synthetic traces -- capturing conflicts that we expect to occur
at scale.  %, as more and more users interact with the system and need to deal with intermittent network connectivity.  
We also
discuss novel strategies for tracking moving people/things and describe how we implement these in our system.  In summary, our work
makes the following contributions:

\begin{itemize}
\item We design and implement a system that captures and combines physical entities, their inter-relationships, and real-time sensor data 
		in buildings.% using mobile phones, qr code, and a cloud-based infrastructure.
\item We observe that certain combinations of swipes give us useful information to set the location of people and things over time.
		We codify this observation in our \emph{context-tracker} and use it to maintain consistency between the entity-relationship graph and the 
		state of the physical world.  To the best of our knowledge, this is radically different from the approaches in standard 
		localization techniques.  However, we argue that it can be used to \emph{enhance} their accuracy and overall performance.
\item We design and implement a log-replay and transaction manager to deal with conflicts that arise when multiple users re-establish
		a connection with the server and wish to execute updates to the entity-relationship graph.  We describe how different 
		conflict-resolution policies can be implemented and our rationale for the policies we chose.
\end{itemize}

\vspace{0.08in}

% We also argue about the broader implication of our observations and solution approaches.  We claim that issues of mobility
% and consistency management are fundamental challenges that must be overcome in order to provide \emph{physical data services} -- a new class
% of applications that has emerged at the intersection of energy and buildings.  We define physical data services and give some examples.
In the next section we give some background and related work, followed by the system architecture, evaluation, and 
broader discussion.

% \begin{itemize}
% \item QR codes are a convenient choice for tagging items because they are customizable, cheap, easily produced, and easily replaceable.
% \item Smartphones equipped with a camera can download QR code scanning software freely, making it a pervasive, effective, mobile scanner.
% \item Network connectivity is ubiquitous.  Smartphones can connect to the internet through the cellular network or WiFi.
% \end{itemize}

% \vspace{0.08in}

% In addition, we describe how we address the following fundmanetal issues:

% \begin{itemize}
% \item Mobility.  In order to provide energy-visibility, we need to track people \emph{and} objects.
% \item Consistency management.  In order to know what analytics to run we need to maintain an accurate
% 		view of the physical world in our virtual representation; that is,
% 		the objects and their inter-relationship.
% \item Apportionment and aggregation.  In order to provide real-time energy analytics we need to deal with the
% 		dynamics of a changing set data sources; fundamentally linked with the virutal view of the physical world.
% \end{itemize}

% \vspace{0.08in}
% We address each of these through a series of gestures that give us implicit and explicit information about people
% and the objects around them.  We also use the virtual representation of the world to manage moving aggregates of
% physical data. Ultimately, we hope that detailed understanding of personal energy use will induce behavioral changes that reduce
% overall energy consumption.

% This paper examines two fundamental challenges -- mobility and consistency management -- in providing real-time, find-grained physical-data 
% services in buildings through mobile phones.  Physical data services are \emph{data-analytic} and \emph{control} services 
% based on real-time sensor data and related contextual information.  For this paper, we focus on data analytics but discuss the implications
% on control.  What makes these services unique is that the analytical processes are informed by the entities in the world and
% their inter-relationships.  Adding to the complexity of the problem, we must also deal with the temporal nature of these relationship
% and capture how they evolve over time.  Through a combination of QR codes and mobile phones, we record
% the entities and construct an entity-relationship graph with a series of gestures by made users when they swipe QR codes.
% QR codes function as a cheap, universal tagging facility for items in the world.  Mobile phones function as an entity and relationship 
% recorder and personal viewer -- through which analytical data services are provided.  

% We discuss our experience from one large deployment and our re-designed on-going second deployment in a building on our campus.
% In the first iteration, we tagged 351 items over 139 rooms in a 141,000 square-foot building on campus.  We used a small team
% of students to tag items throughout the building and mainly used it to collect and provide information to occupants
% about the distribution of plug-loads throughout the building.  We used a mobile application and associated cloud-based
% infrastructure to record and query the information.  Based on this experience, we have re-designed the system and started a
% second deployment in the same building.  It includes various features that address the fundamental challenges we faced in the
% first iteration and includes live metering information and real-time analytics.  This deployment, thus far, consists of 
% {\bf 20 wireless plug-load meters and 50 QR codes, spread throughout a single floor in the same building.} -- More here\dots

% Our work makes the following contributions:




% \subsection{Motivation} 
% Greater visibility and control of building environment can provide more incentives and opportunities to reduce energy consumption in buildings

% Examples of these from previous work:
% \begin{itemize}
% \item Dashboards
% \item Andrew’s lighting thing
% \item Kamin’s thing
% \item personalized energy attribution thing
% \item BEMs
% \end{itemize}

% \subsection{Generalized definition of physical data services}

% \begin{itemize}
% \item energy analytics
% \item control
% \end{itemize}

% \subsection{Short-comings of related work}
% not fined-grained enough, not personal, not real-time.

% \subsection{main thesis}  
% In order to provide fine-grained, personal, real-time services we need to address fundamental challenges in

% \begin{itemize}
% \item mobility and consistency management: tracking the movement of people and things over time and keeping track of how they are inter-related.
% \item there are other challenges, but these are the ones addressed in this paper.
% \end{itemize}