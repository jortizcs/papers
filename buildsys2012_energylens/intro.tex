\section{Introduction}

This paper examines two fundamental challenges -- mobility and consistency management -- in providing real-time, find-grained physical-data 
services in buildings through mobile phones.  Physical data services are \emph{data-analytic} and \emph{control} services 
based on real-time sensor data and related contextual information.  For this paper, we focus on data analytics but discuss the implications
on control.  What makes these services unique is that the analytical processes are informed by the entities in the world and
their inter-relationships.  Adding to the complexity of the problem, we must also deal with the temporal nature of these relationship
and capture how they evolve over time.
Through a combination of QR codes and mobile phones, we record
the entities and construct an entity-relationship graph with a series of gestures by made users when they swipe QR codes.
QR codes function as a cheap, universal tagging facility for items in the world.  Mobile phones function as an entity and relationship recorder and personal viewer -- through which analytical data services are provided.  

We discuss our experience from one large deployment and our re-designed on-going second deployment in a building on our campus.
In the first iteration, we tagged 351 items over 139 rooms in a 141,000 square-foot building on campus.  We used a small team
of students to tag items throughout the building and mainly used it to collect and provide information to occupants
about the distribution of plug-loads throughout the building.  We used a mobile application and associated cloud-based
infrastructure to record and query the information.  Based on this experience, we have re-designed the system and started a
second deployment in the same building.  It includes various features that address the fundamental challenges we faced in the
first iteration and includes live metering information and real-time analytics.  This deployment, thus far, consists of 
{\bf 20 wireless plug-load meters and 50 QR codes, spread throughout a single floor in the same building.} -- More here\dots

Our work makes the following contributions:

\begin{itemize}
\item We design and implement a system that captures physical entities, their inter-relationships, and real-time sensor data and 
		allows users to input/edit these over time.
%\item We carefully design QR codes to minimize swipe times and increase robustness under varying lighting conditions.
\item We observe that QR-code swiping gestures give us useful information about where people and items are located over time.  We
		use these user-initiated actions to track people and things.
\item We show how log-replay and transaction processing is used in our system to maintain a consistent view of the physical world.
\end{itemize}


\subsection{Motivation} 
Greater visibility and control of building environment can provide more incentives and opportunities to reduce energy consumption in buildings

Examples of these from previous work:
\begin{itemize}
\item Dashboards
\item Andrew’s lighting thing
\item Kamin’s thing
\item personalized energy attribution thing
\item BEMs
\end{itemize}

\subsection{Generalized definition of physical data services}

\begin{itemize}
\item energy analytics
\item control
\end{itemize}

\subsection{Short-comings of related work}
not fined-grained enough, not personal, not real-time.

\subsection{main thesis}  
In order to provide fine-grained, personal, real-time services we need to address fundamental challenges in

\begin{itemize}
\item mobility and consistency management: tracking the movement of people and things over time and keeping track of how they are inter-related.
\item there are other challenges, but these are the ones addressed in this paper.
\end{itemize}