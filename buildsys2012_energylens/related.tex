\section{Related work}

%\begin{itemize}
% \item dashboard
% \item andrew's lightin control work
% \item Kamin's hvac control work
% \item BEMs
% \item sMAP stuff
%\item Buildsys 2010 work~\cite{hbci}
%\item distributed consistency management: COPS
%\item mobility: tracking things with RFID~\cite{rfid_gonz2006}
%\item mobility: tracking of people, wifi indoor localization
%\item entity-relationship graphs
%\item homeOS [microsoft]
%\item HP Cooltown~\cite{cooltown}
%\end{itemize}

This work focuses on the object definition, tracking, and management component of Hsu et al.~\cite{hbci}.  Their
work stratefied the set of challenges that one could potentially face if the application were deployed at scale.  Our
work, in constrast, bases its design rationale on a real deployment that is taking place at a large scale.
The real systems challenges in dealing with intermittent connectivity
and conflict resolution (when connectivity is re-established).  We also focus on leveraging gestures to minimize
the cost of interaction for users, while maximizing the information we can attain about the state of the world.

% Tracking people/indoor localization
An important aspect of the Energy Lens is determining when people and things have moved.  This requires some form 
of indoor localization.  There's a large body of literature in the area of indoor localization with mobile phones ranging from 
using wifi~\cite{radar}, to sonar~\cite{cricket}, to ambient noise~\cite{abs}, and a combination of sensors on the 
phone~\cite{surroundsense, darwinphone}.  Dita~\cite{dita} uses acoustic localization of mobile phones and also uses the infrastructure 
to determine gestures in free-space that are classified into pre-defined control actions.  Each of these require relatively complex 
software and/or infrastrure.  
We take a radically different, simple approach.  We use cheap, easy to re/produce tags (QR codes), place them on things in the 
environment over incrementally and use the natural \emph{swiping gesture} that users make, when interacting with the Energy Lens 
application, to track when they have moved or when the objects around them have moved.  The working principal is to attain as much 
information from their gesture to determine when something/one has moved.  We discuss our heuristics in section~\ref{sec:swipes}.

ACE~\cite{ACE}

% Tracking things
Logistic systems focus on the tracking of objects as the move through distribution sites to warehouses, stores, shelves,
and purchase.  Items are tracked through bar code or RFID readers.  However, the workload is very structured and well
defined.  The authors of~\cite{rfid_gonz2006} describe this structure and leverage it to minimize storage
requirements and optimize query-processing performance.  Energy Lens uses the QR codes as the tag and the phone as an active
reader.  As objects move, users scan those items to their new location.  However, objects may belong to one or
many people, they can be metered by multiple meters a day, and their history in the system
is on-going.  In contrast, a typical logistics workload has a start (distribution site) and end point (leaving the store
after a sale).  In our workload, relationship semantics are important; we need to know whether the meter is \emph{bound-to}
rather than simple \emph{attached-to} an item.  We discuss the difference later in the paper.
% In addition to traditional logistics-style queries -- \emph{What is the average time that it took coffee-makers to move from the 
% warehouse to the shelf and finally to the checkout counter in January of 2004?} -- energy-analytics requires queries to group
% partial traces from meter data by tracking what meters the item attached to over the specified time-frame.
% The Energy Lens system collects and manages this kind of information to enable such queries.
Furthermore, we take advatange of natural gestures the user makes with the phone while scanning QR codes to extract
information about the current location of the user or things.

% Tagging items, virtual services
The key idea in the HP Cooltown~\cite{bridgingphysical,cooltown} work is to web-enable `things' in the world, grouped-by
`place', and accessed by `people' via a standardized acquisition protocol (HTTP) and format (HTML, XML).  
Cooltown creates a web presence for things in the world either directly (embedded web server) or indirectly 
(URL-lookup tag) as a web page page that display the services it provides.  Many of the core concepts in Cooltown 
also show up in Energy Lens.  The main overlap is the use of tags in the world that contain a reference to a virtual 
resource, accessible via HTTP through
a network connection.  Cooltown, however, explicitly chooses not maintain a centralized relationship
graph, it leverages the decentralized, linking structure of the web to group associated web pages together.
Furthermore, things are assumed to not move.  People are the main mobile entities.  The kind of applications
we wish to support must track where things are and their specific inter-relationships.  We imposed a richer set of 
semantics on our, centrally maintained, relationship graph and use it to provide detailed energy information.

