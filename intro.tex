\section{Introduction}
Buildings consume an enourmous amount of energy in countries around the world.  In 
Japan, 28\% of the energy produced is consumed in buildings~\cite{japanbuildings} while in the United 
States it is as high as 40\%~\cite{epabuildings}.  Moreover, studies show that between 30-80\% of it
is wasted~\cite{waste_science, next10_waste}.  Large commercial buildings are typically instrumented
with a large number of sensors measuring various aspects of building operation.  Although this data is
typically used to assure operational stability, they may also be used to measure, observe, and identify
instances of wasted use.

Identifying instances of wasted energy use is non-trivial.  System efficiency is defined as the ratio of the 
useful work done to the energy it consumes.  In the case of buildings, we broadly define useful work as 
the energy used to support occupant activities.  From the persective of the building that means maintaining
a comfortable temperature setting, providing power for plug-load devices, and providing adequate lighting
conditions; particularly in spaces that \emph{are occupied}.  However, identifying efficient use of resources,
even when a space is occupied, is difficult.  Typically it involves deep knowledge of the usage scenario and
a meaningful understanding of what it takes to support the activity.  Furthermore, situations and activities differ
greatly.  The outside weather changes, varying schedules affect occupancy, rooms have lectures, class,
or other office activities.  Simple put, the process is time conusming, requires specialized knowledge,
and does not scale.

We conjecture that efficient energy use can be identified through anomolies in the correlation
patterns of resource usage.  A simple example is the use of a heater.  When the outside temperature is low
the heater turns on.  If the outside temperature is high and the heater is on, we signal a flag that this
behavior is problematic (it also happens to be wasteful).  If applied correctly, our conjecture should narrow
the search space where a subset of those anomolies \emph{may} also be inefficient.
It also generalizes different sensors and measurement units.  In summary, we are interested
in correlated patterns of usage in order to:

\begin{itemize}
\item Uncover potentially un/correlated usage patterns that lead to efficiencies.
\item Identify anomolies from normal usage patterns.  These often indicate instances of waste.
\item Provide a way to codify efficient or expected usage over time and identify deviations
		from specification.
\end{itemize}

In this paper we 

In this paper, we present early result in our research for constructing a correlated-usage model.  We show that

\begin{enumerate}
\item Raw-trace correlation is noisy for 
\end{enumerate}

% Green IT

% Understand the energy consumption of a building and identify savings opportunities.

% Identification of energy consuming devices that are correlated.
% Uncover usage patterns of correlated device that are energy efficient.
% Detect deviation from the energy efficient pattern and report to the user.

% During the design of our application the first difficulty was to identify the set of devices that have related energy consumption.

% This article focuses on this problem.

% Results:
% \begin{enumerate}
% \item Correlation is noisy and can't find inter-relationships between sensors
% 		with subtle differences.
% \item Underlying behavior should extract most-common denominator in comparing traces
% 		to observe truly correlated behavior.
% \item Empirical mode decomposition (EMD) can be used to compare underlying behavior after the
% 		removal of the dominant frequencies in the signal.
% \end{enumerate}

% \subsection{ideas}

% Future work:
% \begin{enumerate}
% \item We can create a time-varying dependency graph to compare ``normal'' versus ``abnormal'' behavioral
% 		patterns in underlying use.
% \item We can codify ``normal'' or ``efficient'' graphs and compare with real graph constructs over time.
% \end{enumerate}

% Possible algorithms:
% \begin{enumerate}
% \item find correlated and uncorrelated sensors
% \item construct correlation network where the nodes are the sensors and an edge implies correlation above
% 		threshold. (We can also construct the complement of that.)
% \end{enumerate}

