\section{Introduction}
Buildings consume an enormous amount of energy in countries around the world.  In 
Japan, 28\% of the energy produced is consumed in buildings~\cite{japanbuildings} while in the United 
States it is as high as 40\%~\cite{epabuildings}.  Moreover, studies show that between 30-80\% of it
is wasted~\cite{waste_science, next10_waste}.  Large commercial buildings are typically instrumented
with a large number of sensors measuring various aspects of building operation.  Although this data is
typically used to assure operational stability, they may also be used to measure, observe, and identify
instances of wasted use.

Identifying instances of wasted energy use is non-trivial.  System efficiency is defined as the ratio of the 
useful work done to the energy it consumes.  In the case of buildings, we broadly define useful work as 
the energy used to support occupant activities.  From the perspective of the building that means maintaining
a comfortable temperature setting, providing power for plug-load devices, and providing adequate lighting
conditions; particularly in spaces that are occupied.  However, identifying efficient use of resources,
\emph{especially} when a space is occupied, is difficult.  Typically it involves deep knowledge of the usage scenario and
a meaningful understanding of what it takes to support the activity.  Furthermore, situations and activities differ
greatly.  The outside weather changes, varying schedules affect occupancy, rooms have lectures, class,
or other office activities.  Simply put, the process is time consuming, requires specialized knowledge,
and does not scale.

Devices are typically used in together in some fashion.  For example, in an office
setting a person enters their office, turns on their PC or connects their laptop, turn on the lights, etc.
When the person leaves the office, they revert back to state the office was in when it was
not in use.  Waste occurs when something is left on.  Another example is the way a heater is used.  
When the outside temperature is low the heater turns on.  If the temperature is high and the heater is on, there
is a problem.  Morever, if the heating and cooling system that serve the same space are on simultaneously, that
is a problem that is \emph{particulary} wasteful and hard to detect by the occupants.  
Fundamentally, understanding ``normal'' spatio-temporal usage patterns between devices could help
identify problems when devices are not being used correctly. % in concert.
% If the outside temperature is high and the heater is on, we signal a flag that this
% behavior is problematic; it also happens to be wasteful.  If identified, this waste can be avoided or minimized.
We conjecture that inefficient energy use can be identified through anomalies in the correlation
patterns of resource usage.  We examine device correlation patterns in this paper and look specifically
at processing raw sensor signals, such that the correlations we find are meaningful.
% If applied correctly, our conjecture should narrow
% the search space where a subset of those anomalies \emph{may} also be inefficient.


 
%It also generalizes acrossdifferent sensors and measurement units.  
% In summary, we are interested
% in correlated patterns of usage in order to uncover potentially un/correlated usage patterns 
% that lead to efficiencies, identify anomolies from normal usage patterns, provide a way to codify efficient 
% or expected usage over time, and identify deviations from specification.

In this paper, we present early results for correlating usage patterns across a large number of sensors
in a single deployment.  We analyzed data from a 12-story office building at the University of Tokyo.  
The deployment consists of almost 700 sensors monitoring a broad range of devices within and outside 
the building.  Our initial observations and results include the following:

\begin{enumerate}
\item Raw-trace correlation analysis is too influenced by the common underlying trends in the data
	to identify meaningful relationships.
%is too noisy to be meaningful in our correlation model.
\item Using a technique called empirical model decomposition (EMD)~\cite{huang:emd1998} removes the 
		underlying trend and this helps identify truly correlated sensor traces.
\item We can construct clusters of correlated sensor that are spatio-temporally correlated, without
		apriori knowledge of their placement.
\end{enumerate}

In the rest of the paper we explain EMD and how we use it, we show various examples of our technique on real-world
traces, and we discuss the implications and future work.

% Green IT

% Understand the energy consumption of a building and identify savings opportunities.

% Identification of energy consuming devices that are correlated.
% Uncover usage patterns of correlated device that are energy efficient.
% Detect deviation from the energy efficient pattern and report to the user.

% During the design of our application the first difficulty was to identify the set of devices that have related energy consumption.

% This article focuses on this problem.

% Results:
% \begin{enumerate}
% \item Correlation is noisy and can't find inter-relationships between sensors
% 		with subtle differences.
% \item Underlying behavior should extract most-common denominator in comparing traces
% 		to observe truly correlated behavior.
% \item Empirical mode decomposition (EMD) can be used to compare underlying behavior after the
% 		removal of the dominant frequencies in the signal.
% \end{enumerate}

% \subsection{ideas}

% Future work:
% \begin{enumerate}
% \item We can create a time-varying dependency graph to compare ``normal'' versus ``abnormal'' behavioral
% 		patterns in underlying use.
% \item We can codify ``normal'' or ``efficient'' graphs and compare with real graph constructs over time.
% \end{enumerate}

% Possible algorithms:
% \begin{enumerate}
% \item find correlated and uncorrelated sensors
% \item construct correlation network where the nodes are the sensors and an edge implies correlation above
% 		threshold. (We can also construct the complement of that.)
% \end{enumerate}

