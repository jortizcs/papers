Despite the growing impact of climate change and energy prices, 
per-capita energy consumption is rising. Part of the problem is visibility. We do not 
have scalable means of observing our energy consumption patterns and determining how to optimize and reduce our
consumption.
Mobile smartphones present a unique opportunity to enable an energy view on the physical world. 
They can bridge the physical world, information infrastructure, and people
through a rich set of sensors, ubiquitous connectivity, and highly personal user interface. 
With QR codes as cheap tags on items and places in the physical world, the
camera becomes a portable scanner in your pocket, in addition to its
traditional functions.  We explore this
unique triple point
and re-examine classical problems of context and consistency management in mobile
systems.  We also examine this combination as it pertains to energy management of physical
devices.  In doing so, we are re-introduced to problems of apportionment and aggregation of sensor data,
except with a continuously changing set of constituents.  We describe our solution in a technique
called \emph{dynamic aggregation} that maintains moving aggregates as the
set of data sources changes over time.  We deployed our system in a 
141,000 square-foot building, tagging 351 items over 139 room across 7 floors.

% When combined with QR codes, the on-board camera provides us with a portable scanner

% The camera,
% when combined with QR codes, gives us a portable scanner and convenient mechanism for tying these world together. 
% In this paper, we describe our system and deployment experience for a mobile phone application the provides 
% user-centric energy-view of the physical world. We describe the challenges, specifically dealing with mobility, 
% and how we address them in a set of three separate applications: an energy auditing application, a 
% device energy scanner, and a personal energy counter. We also introduce a technique called \emph{dynamic aggregation}
% which allows us to seamlessly track the constituents of aggregated energy calculations, as they move from one 
% location to another.

% Despite the recent impact of global warming and a steady increase in energy prices, 
% per-capita energy consumption is rising. Part of the problem is about visibility. We simply do not 
% have any good ways of seeing how we consume energy, and therefore, how to optimize and reduce it. 