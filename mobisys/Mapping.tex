Mapping items in a granular fashion allows for personalized visualization of energy use. Figure <FIGURE NUMBER> shows an early stage mock of a test office that has been documented with the mobile auditing application and mapped on top of an overhead view of the room. the resulting map of the room can publish live meter data visually and allow occupants to observe their energy use spatially and temporally in a continuous fashion. This kind of device specific feedback will raise user's awareness of their energy usage, and if combined with the online publishing system in future revisions, will allow users to have an alternate method of assessing their energy use that is distinct from time-series graphics.  It should be noted that the figure's axes are in units of half feet, which corresponds to the grid resolution of the mapping system used to plot item's locations. Future revisions of this system will allow users to select items visually on the map and receive energy saving tips and usage histories for the device specified - this will further enhance the linkage between the physical and cyber existence of the device. 
	This mapping service can exist as an additional layer on top of the monitoring system that exists within the application as built, and will serve to help users connect with the actual physical nature of the system. While it takes a slightly longer amount of time to build a map of the room, it will be very informative in future studies to document how users with and without this feature compare and contrast in their capabilities of saving energy and levels of understanding relating to how their devices use electricity. This kind of rapid assessment and display capability is much in line with the accelerated auditing procedure demonstrated, and will represent a quickly rewarding capacity to integrate feedback into monitoring systems. 