Despite the recent impact of global warming and a steady rise in energy prices, 
per-capita energy consumption is rising. Part of the problem is about visibility. We simply do not 
have any good ways of seeing how we consume energy, and therefore, how to optimize and reduce it. 
Mobile smartphones present a unique opportunity to enable an energy view on the physical world. 
Smartphones provide a convenient bridge between the physical world, users, and computational infrastructure 
through its rich set of sensors, user interface, and ubiquitous connectivity. The camera, specifically,
when combined with QR codes, gives us a portable scanner and convenient mechanism for tying these world together. 
In this paper, we describe our system and deployment experience for a mobile phone application the provides 
user-centric energy-view of the physical world. We describe the challenges, specifically dealing with mobility, 
and how we address them in a set of three separate applications: an energy auditing application, a 
device energy scanner, and a personal energy counter. We also introduce a technique called \emph{dynamic aggregation}
which allows us to seamlessly track the constituents of aggregated energy calculations, as they move from one 
location to another.