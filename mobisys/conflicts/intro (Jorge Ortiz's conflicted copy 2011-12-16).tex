\section{Introduction}
The United States leads the world in per-capita energy consumption.  Furthermore, 
our electricity use has consistently increased over the last 40 years~\cite{oecd2011}.
With the specter of global warming and the increasing cost of energy, we must explore new
ways for individuals to gain better visibility and insight about their energy consumption in order to optimize and reduce it. 
With the increasing penetration of embedded sensors in the environment and the
the continued rise in the number of smartphone users, we see an opportunity to bridge the physical world
to our computational infrastructure to provide a `energy lens' on the physical world.  This paper describes the design
and implementation of a personalized, mobile energy visibility system.  We discuss
the architecture, experience, and challenges in a deployment we have done inside a campus building.
%, and challenges in the deployment of the system in a building on our campus.

%Mobile phones are obvious bridge between the physical world and better visibility.  Many people own a mobile phone
Smartphones serve as a natural point of intersection between the physical world, computation, and users.  The camera serves as 
the input interface from the physical world, the screen functions as the
interface to the user, and its connectivy serves to overlay virtual services on top of the physical capture to be presented
to the user.  In addition, since phones are highly personal items, they serve as a good proxy for information about the user
that can personalize the views and services.  Moreover, as mobile devices, views can be tailored to location and other 
contextual cues in the environment.

The use of mobile phones presents classical, fundamental challenges related to mobility.  Moreover, objects in the physical world 
also move from place to place.  This makes it even more difficult in trying to do fine-grained energy accounting.  Not only must we
deal with mobile users, but also mobile, energy-consuming objects in the environment.  For example, to account for the total
energy consumed on floor {\tt X} at time {\tt t} we must know all the energy-consumers on floor {\tt X} at that time.  Over time, this list
changes, and these changes should be reflected in the total calculation at \emph{any} point in time.  Ultimately, physical-world state 
needs to be tracked by the application; it needs to know where things are and
where people are so that it can answer queries about these things relative to where/who is asking it.  Consistency management 
is non-trivial.

Our system uses QR codes to tag items in the physical world.  Once tagged, there are three types of interaction: 
registration, binding, and scanning.  During the registration phase a user scans the QR code and enters information about the
item it is attached to.  To bind items together a user scans multiple items.  This creates creates virtual links between items that
can be used to fetch or compute information about the item in relation to other items.  For example, when a meter is bound
to an item the data produced by the meter can be used a proxy for the item with respect to the measurements being taken
by that meter.  We used a network of wired and wireless meters and pushed their data into a cloud-based service that
acts as a proxy for meters in the physical world.  In addition, during the registration phase, metadata about each item is recorded
that allows us to aggregate measurements in ways that are relevant to the user.  For example, 
we can give the user visibility into the energy consumption of the room they are in, the item they scan, or
the energy consumption of the items that belong to them.

Our system was deployed incrementally in a 7-story, 141,000 square-foot building.  
We tagged 351 items spread over 139 rooms throughout all floors.  On this infrastructure we built an energy auditing application,
a device energy viewer, and a personalized energy application.  Based on the initial deployment experience
we found that:

\begin{itemize}
\item QR codes are a convenient choice for tagging items because they are cheap, easily produced, and easily replaceable.
\item Most smartphones are equipped with a camera and can download QR code scanning software for free, making it a effective, mobile scanner.
\item Network connectivity is ubiquitous.  Smartphones can connect to the internet through the cellular network or WiFi.
\end{itemize}

\vspace{0.08in}

Although the infrastructure takes some time to set up at scale, the deployment can be done incrementally.  
%Most of the infrastructure
%was already in place, as wifi is generally available throughout the building.  
% Furthermore, for users that cannot connect to the wifi,
% they can typically connect to a network through the 3G/4G network in the area, provided by their cellular service provider.  
Furthermore, once deployed, the infrastructure provides immediate value for all building occupants and sets a good foundation
for building different types of applications.  We describe some of the applications we built later in the paper.
%Also, maintenance is simple, as tags are easily printed
%on any printer and can be generated to replace worn or damaged tags.

Building applications that use the infrastructure reveal many challenges.  Some common challenges include:

\begin{itemize}
\item Mobility.  Maintaining user location (and context, more broadly) is difficult.  However, tagged spaces throughout
		the building provide a check-in mechanism that lets us keep track a user's location over time.  Still, users
		often forget to re-scan their location to their set spatial context.  However, we learn a lot about context based
		on natural user action.
\item Consistency maintenance between the physical world and virtual representation.  Users forget to capture changes
		in item inter-relationships.  For example, item A was once attached to meter M, but is now attached to meter \emph{m}.
\item Apportionment/accounting accuracy is difficult to control due to the challenges with mobility and consistency, but also 
		due to metadata consistency over time (even when the mechanism that solve the first two problems, are correctly 
		implemented).  We introduce a feature called \emph{dynamic aggregation} that keeps track of changes inputted by users
		to track the energy consuming constituents over time.
\end{itemize}

\vspace{0.08in}

% \begin{itemize}
% \item we must capture all devices, meters, \emph{and} their inter-relationship, but this can be done iteratively
% \item we must capture some location and other information about each entity for accounting
% \item engagement is desirable
% \end{itemize}

We discuss the various ways we have addressed some of these challenges in our deployment.  Addressing these challenges is an 
on-going effort to reduce the interaction overhead between the user
and her environment.  An ideal solution would enable full, consistent visibility of the energy consumption of items in the environment
as well as allow a user to effectively observe their own footprint, with minimal dependence on users maintain view consistency.  
Ultimately, we hope that detailed understanding of personal energy use will induce behavioral changes that reduce
overall energy consumption.

% Much of the process is manual and engagement is key for building up the infrastructure.  The mobile phone application helps, but it
% is still quite cumbersome to get everything deployed and running from the ground up.  However, the processes scales with the number of
% users and it is a one-time cost to set up.

% There's also a list of fundamental challenges we ran into and offer initial solutions for in order to realize our vision.



% It is challenging to keep track of users automatically, it still requires engagement from the user which can be considered burdonsome.
% Because many of the processes are not automatic, users often forget to associate/dissociate the relationship between meters and items.
% This leads to inconsistencies between the physical world and the virtual world.  Finally, these inconsistencies leads to energy accounting
% errors that present a false view of the energy-consuming state of the physical world.  We discuss some initial approaches to address
% some of these issues and our plans for future work.