\section{Introduction}
The United States leads the world in per-capita energy consumption; 
our electricity use has consistently increased over the last 40 years~\cite{oecd2011}; and, other parts of the world are rising all too rapidly.
With the specter of climate change and the increasing cost of energy, we must explore new
ways for individuals to gain visibility and insight into their energy consumption in order to optimize and reduce it. 
With the increasing penetration of embedded sensors in the environment and
the continued rise smartphone adoption, we see an opportunity for smartphones to bridge the physical world
to our computational infrastructure and provide an `energy lens' on the physical world.  

Smartphones serve as a natural point of intersection between the physical world, information, and people.  The camera
accepts input from the physical world, the screen functions as the
interface to the individual, and its connectivity serves to overlay virtual services on top of the physical capture to be presented
to through the screen.  
In addition, since phones are highly personal items, they serve as a good proxy for information about the user
that can be used to personalize the views and services.  Moreover, as mobile devices, views can be tailored to location and other 
contextual cues in the environment.

% The use of mobile phones presents classical, fundamental challenges related to mobility.  Moreover, objects in the physical world 
% also move from place to place.  This makes it even more difficult in trying to do fine-grained energy accounting.  Not only must we
% deal with mobile users, but also mobile, energy-consuming objects in the environment.  For example, to account for the total
% energy consumed on floor {\tt X} at time {\tt t} we must know all the energy-consumers on floor {\tt X} at that time.  Over time, this list
% changes, and these changes should be reflected in the total calculation at \emph{any} point in time.  Ultimately, physical-world state 
% needs to be tracked by the application; it needs to know where things are and
% where people are so that it can answer queries about these things relative to where/who is asking it.  Consistency management 
% is non-trivial.

The use of mobile phones presents classical, fundamental challenges related to mobility.  Typically, mobility
refers to the phone as the person carrying it moves from place to place.  However, in the energy-tracking
context, we are also refering to the movement of energy-consuming objects.  Tracking their relationships to spaces and people is as important as tracking people.
%Morever, information about either the object or the person can be used to infer movement of another object or person.
In our deployment, we describe how we deal with both moving people and moving objects.  We show that these historically difficult
problems can be addressed relatively easily, if the proper infrastructure is in place.  We provide evidence that the approach is simple, incrementally deployable, and scalable.

Our system uses QR codes to tag items and spaces in the physical world.  Once tagged, there are three types of interactions - registration, linking, and scanning -  establish important relationships.  Registration is the act of creating a virtual object to represent a physical one.  Linking
captures the relationship between a pair of objects.  Scanning is the act of performing an item-lookup.
Each of these interactions requires a set of swiping gestures.  Linking requires two tag swipes while the other two actions
require a single tag swipe.  We incorporate these gestures in a set of applications in a deployment we did inside
a building on our campus.  Our system was deployed incrementally in a 7-story, 141,000 square-foot building.  
We tagged 351 items spread over 139 rooms throughout all floors.  On this infrastructure we built an energy auditing application,
a device energy viewer, and a personalized energy tracker.  Based on our initial deployment experience
we observe that:

\begin{itemize}
\item QR codes are a convenient choice for tagging items because they are customizable, cheap, easily produced, and easily replaceable.
\item Smartphones equipped with a camera can download QR code scanning software freely, making it a pervasive, effective, mobile scanner.
\item Network connectivity is ubiquitous.  Smartphones can connect to the internet through the cellular network or WiFi.
\end{itemize}

\vspace{0.08in}

In addition, we describe how we address the following fundmanetal issues:

\begin{itemize}
\item Mobility.  In order to provide energy-visibility, we need to track people \emph{and} objects.

% Maintaining user location (and context, more broadly) is difficult.  However, tagged spaces throughout
% 		the building provide a check-in mechanism that lets us keep track a user's location over time.  Still, users
% 		often forget to re-scan their location to their set spatial context.  However, we learn a lot about context based
% 		on natural user action.
\item Consistency management.  In order to know what analytics to run we need to maintain an accurate
		view of the physical world in our virtual representation; that is,
		the objects and their inter-relationship.

	 % maintenance between the physical world and virtual representation.  Users forget to capture changes
		% in item inter-relationships.  
		%For example, item A was once attached to meter M, but is now attached to meter \emph{m}.
\item Apportionment and aggregation.  In order to provide real-time energy analytics we need to deal with the
		dynamics of a changing set data sources; fundamentally linked with the virutal view of the physical world.

		%of objects, changing over time to reflect the state of the world; 

		% /accounting accuracy is difficult to control due to the challenges with mobility and consistency, but also 
		% due to metadata consistency over time (even when the mechanism that solve the first two problems, are correctly 
		% implemented).  We introduce a feature called \emph{dynamic aggregation} that keeps track of changes inputted by users
		% to track the energy consuming constituents over time.
\end{itemize}

\vspace{0.08in}

% During the registration phase a user scans the QR code and enters information about the
% item it is attached to.  To link items together a user scans a pair items.  This creates creates virtual links between items that
% can be used to fetch or compute information about the item in relation to other items.  For example, when a meter is bound
% to an item the data produced by the meter can be used a proxy for the item with respect to the measurements being taken
% by that meter.  We used a network of wired and wireless meters and pushed their data into a cloud-based service that
% acts as a proxy for meters in the physical world.  In addition, during the registration phase, metadata about each item is recorded
% that allows us to aggregate measurements in ways that are relevant to the user.  For example, 
% we can give the user visibility into the energy consumption of the room they are in, the item they scan, or
% the energy consumption of the items that belong to them.

% Our system was deployed incrementally in a 7-story, 141,000 square-foot building.  
% We tagged 351 items spread over 139 rooms throughout all floors.  On this infrastructure we built an energy auditing application,
% a device energy viewer, and a personalized energy application.  Based on the initial deployment experience
% we found that:


% Although the infrastructure takes some time to set up at scale, the deployment can be done incrementally.  
  
% Furthermore, once deployed, the infrastructure provides immediate value for all building occupants and sets a good foundation
% for building different types of applications.  We describe some of the applications we built later in the paper.

% Building applications that use the infrastructure reveal many challenges.  Some common challenges include:



% \begin{itemize}
% \item we must capture all devices, meters, \emph{and} their inter-relationship, but this can be done iteratively
% \item we must capture some location and other information about each entity for accounting
% \item engagement is desirable
% \end{itemize}

We address each of these through a series of gestures that give us implicit and explicit information about people
and the objects around them.  We also use the virtual representation of the world to manage moving aggregates of
physical data.
% We discuss the various ways we have addressed some of these challenges in our deployment.  Addressing these challenges is an 
% on-going effort to reduce the interaction overhead between the user
% and her environment.  An ideal solution would enable full, consistent visibility of the energy consumption of items in the environment
% as well as allow a user to effectively observe their own footprint, with minimal dependence on users maintain view consistency.  
Ultimately, we hope that detailed understanding of personal energy use will induce behavioral changes that reduce
overall energy consumption.

% Much of the process is manual and engagement is key for building up the infrastructure.  The mobile phone application helps, but it
% is still quite cumbersome to get everything deployed and running from the ground up.  However, the processes scales with the number of
% users and it is a one-time cost to set up.

% There's also a list of fundamental challenges we ran into and offer initial solutions for in order to realize our vision.



% It is challenging to keep track of users automatically, it still requires engagement from the user which can be considered burdonsome.
% Because many of the processes are not automatic, users often forget to associate/dissociate the relationship between meters and items.
% This leads to inconsistencies between the physical world and the virtual world.  Finally, these inconsistencies leads to energy accounting
% errors that present a false view of the energy-consuming state of the physical world.  We discuss some initial approaches to address
% some of these issues and our plans for future work.