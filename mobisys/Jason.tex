Comments on figures. 

figure 8 � % chart of devices. 

An ability to auto-generate and update the device distribution in a building lends itself to creating a building management system that can automatically switch management profiles as users move in and out. In addition to automatic adaptability, device profiling allows users can see what kind of devices are common among other users and enables comparison of energy saving strategies that are employed by different users. 

figure 9- Non-Coincident power draw of devices

Upon completing the first round audit, this static representation of the power draw of plug loads in the building was constructed. It is very noticeable that resistive loads such as toasters, coffee makers, and tea kettles, make up a very large percentage of possible energy demand in the building, yet without metering data, it is impossible to tell if is actually contributing to power draw during a peaking event. This flaw in auditing methodology can be corrected by adding the ability to rapidly assemble cross-sections of load characteristics in the building, a technique known as dynamic aggregation that will be discussed later in this paper. 


figure 10 - power heatmap

This figure is an elegant method of displaying the power consumption of all plug loads in the building aggregated over all the meters for every minute of every day of the month of october. Where in a time series representation of power consumption, height usually corresponds to magnitude, in this figure color represents magnitude and each row on the Y axis is another day in the month. This is a common method of representation for power loads in a building, and this construction of data will lend itself to analysis of patterns in the building as the analysis moves forward into the next stage. 

figure 11 - whole building plug load power draw, october 12

This curve of aggregated plug meter data represents the total of individual plug loads. While this graph is constructed as the sum of six floor level meters, any possible construction under the mobile system should be possible in order to present relevant information to the user. The user will be able to dissect and reconstruct pieced of this curve using dynamic aggregation in order to find out how their power consumption plays into the whole building system. It is important to note that this is just one bar in the heatmap represented with the usual convention of power magnitude vs time. 

figure 12 load duration curve 

In the previous figures, power consumption information is presented as a time series but this data can also be displayed as a load duration curve, indicating how much time in each day the load exceeds a certain level of power draw. This method of displaying load vs duration is common in the power industry when explaining how often power plants are on and active, but can also be used on the load side of the system. The load during this month displayed a number of characteristics, with the maximum load being 95.2320 kW, the minimum being 27.5840 kW, and the average being 41.6632 kW at any given time. The load duration curve is an excellent representation of the short time scales that are being dealt with with respect to shifting load, and can provide a meter based context to enhance the social efficacy of load shedding.
	