\section{Related Work}

%similar to~\cite{Caswell:2000,Kindberg00aweb,activebadge1992, hbci, livinglab, lbnlmels, Chen76theentity-relationship, taxonomy, Hodes, plugloadreport}.

\begin{itemize}
\item Categories:  mobile computing, plug-load studies, combination
\item mobile: \cite{Caswell:2000,Kindberg00aweb,activebadge1992, Hodes}
\item plugload:  \cite{plugloadreport, taxonomy, lbnlmels}
\item both: \cite{hbci, livinglab}
\item data: \cite{Chen76theentity-relationship}
\end{itemize}

Prior work in this area falls into 3 main categories discussed below.  The first is a set of literature on 
mobile systems that is concerned with various aspects of mobile computing that highlight fundamental challenges with
mobility, consistency, and infrastructure setup.  They also explore deeper questions about services and their delivery
through the infrastructure.  Recently, there has been an increase in the characterization of plug-loads in buildings, since
they account for about a third of the energy-consumption envelope.  These sets of studies are about methodology
for data collection, organzation of data, and plug-load characterisitics.  The last set of studies combine elements
of these two by using mobile systems to interact directly with the physical world to learn and control electrical loads
more effectively.  Electrical loads are a superset in which plug-load fall into.  These works explore
similar issues to those that fall into mobile system in combination both methological challenges
with data-collection and context-aware challenges.

The other piece of related literature is in data modeling and processing, specifically the entity-relationship and
in-network aggregation.  In order to provide a view of the data that make sense
to the user, data organization and management is of primary importance.  We discuss these in the following next 
sub-sections.


\subsection{Mobile systems}

\subsection{Plug-load metering}

\subsection{Combination}

\subsection{Data modeling and management}