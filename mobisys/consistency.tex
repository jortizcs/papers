\section{Consistency management}
Consistency management is a fundamental challenge for providing energy services.  Our system needs to maintain a consistent view
of the objects in physical spaces, people, and to have a general sense of context.  Furthermore, the communication environment
itself presents classic challenges in disconnected operations that must be addressed as well.  In this section
we discuss our approach for solve both sides of this problem.  We introduce a technique we call local crowd sourcing, whereby
building occupants help us maintain a consistent view of the world over time and state-change transactions, which combine
local caching of application-level transactions on the backend system that maintains the state of the world and performs
real-time analytics.  For either approach we follow the principle of \emph{eventual consistency}.

\subsection{Distributed consistency management}
We distribute the consistency task to building occupants in exchange for energy data and control services.  One of the fundamental challenges
for providing detailed energy services is in tracking energy-consuming ``objects'' in the physical world and the association
between the devices that meters them and the object itself.  With the large number of plug-loads in the building, a centralized 
inventory-process approach does not scale.

This problem can be broadly solved, at scale, with two basic approaches:

\begin{enumerate}
\item embedded metering and discovery.
\item local crowd-sourcing: ad-hoc metering and human consistency management through the mobile phone.
\end{enumerate}

The first approach requries ubiquitious connectivity and embedded intelligence in the environment, as well as a discovery protocol
that captures relative location and context.  In the ideal case these would be available.  Instead we choose the second solutions, as it
approximates the first.

\subsection{State-change transactions}
Sometimes network connectivity can be problematic.  There are some locations in the building where wifi coverage is limted or 
intermitent access-point issues cause periodic disconnections.  StreamFS supports update-transactions.  A transaction request
can be submitted to StreamFS for processingly mutliple updates in a single transaction.  On the client side, if it becomes
disconnected from the StreamFS server, a transaction request is constructed to include all update requests and send them to the
server when a connection to the server is re-established.  Therefore a transaction manager is baked into the mobile application
to assure that groups of changes occur all together or none at all.