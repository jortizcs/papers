\section{Metadata timeseries queries}
\label{sec:mtsq}
whataksndlkasndlkasn


\subsection{Objects and data}
%
% 	join by global object identifier (128 uuid: 96-bit object_id, 32-bit version sequence)
%
%	Create a lookup service (read INS paper for ideas) to lookup the object
%	It will probably involve something like the work in mobile IP, I3/Ocala, and INS
%
%	Example lookup syntax:
%		sfs://jortiz81.homelinux.com:8080::550e8400-e29b-41d4-a716-4466-554400CF


\subsection{Structural snapshot}

\subsubsection{Storage}

\subsubsection{Querying}

\section{Relationship to OLAP}
The main terms in OLAP consist of Dimensions, Measures, Hierarchies, and Grain.  Dimensions are the axes of aggregation.
For example, you may want to aggregate with respect to location, time, or type.  Measures are the units or category of
measurement that you're making for the data values assocaited with a dimension.  For example, cost is a type of measure,
so is revenus and quatity.  There are hierarchies in the data that dictate the relationship between the dimensions
and how that relationship influences the amount of derived data and computation that needs to be done to satisfy
these hierarchical aggregations.  The grain of the data is the lowest level of granularity.  Abstractly, the lowest grain
in OLAP is the actual transaction.  In the context of StreamFS, it's the raw stream coming in from a stream source.

In a typical OLAP setup, hierarchies do not change, dimenions do not change, grain does not change, measures do not change.
OLAP is perfect for industries with structured analytical accounting such as finance and accounting but less of a fit for
sales, operations, marketing, and R\&D.

% Keyword:  dimensional-relational model -- using the relationshal model (star, snowflake, constellation schemas)
% to logically construct the OLAP hypercube.

% It's extremely difficult, using the star schema in the OLAP hypercube approach, to add a dimension.  The main operations
% on the hypercube are privoting, roll-up, and sub-cube extraction.

What's different here is that the dimensions change -- measures have multiple coordinates at any point in time and the coordinates
change as a function of time, not just measures themselves.  Dimensional coordinates are expressed by tags, where the time dimension,
exists both for measures produced by stream objects and the tags themselves.  The tags are used to reconstructed the relational-DAG
between the dimensions and the objects at any point in time.

% Indexing will have to be addressed.  How do you set up the index structures to get comparable results	to traditional OLAP queries?

\begin{itemize}
\item Explain how are the tags used to handle dynamic dimensionality.
\item Explain how the tags are used to handle hierarchical relationships.
\item Explain how tags are used to support multi-dimenionality (multi-naming).
\item Explain how OLAP operations are performed using these naming approach.
	\begin{itemize}
	\item Rollup.
	\item Drill-down.
	\item Pivoting.
	\item Slice and Dice.
	\item time.
	\end{itemize}
\item Explain how the indexing is done to support these operations efficiently -- use each of the operations for demonstration.
\item Explain how timeseries queries account for dimensionality and hierarchical changes.
\end{itemize}


StreamFS is an analytical framework that provides streaming OLAP for operational processes with schema timeline consistency.
\begin{itemize}
\item streaming olap
\item changing dimensions
\item metadata timeline consistency management
\end{itemize}






