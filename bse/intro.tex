\section{Introduction}

Buildings are sites of very large sensor deployments, typically containing
up to several thousand sensors reporting physical measurement, continuously.
Moreover, with the recent interest in reducing building energy consumption, it
is important to consider ways to quickly bootstrap a set of building data streams
into an anlytical pipeline to determine where there are opportunities for energy savings,
discovery of broken sensors, and assessing and tracking overall building performance.
However, current `point' naming conventions form a bottleneck in the scalability of
the data integration process.  A `point' refers to a physical location where
a sensor is taking measurements. Each building vendor uses their own naming scheme and
uniques variants of each scheme are implemented from building to building; variations exist
even across buildings that have contracted the same vendor to set up their deployment.
This makes the integration process laborious and fundamentally unscalable.  If we
wish to have a broad impact across the entire building stock, at scale, we need
to explore methods for overcoming this challenge.

We observe that every naming scheme looks to capture three point attributes: 
1) the location in space, 2) its relation to an subsytem, and 3) the type of 
measurement it is taking.  

We want to make the streams searchable.  How do we do that?
1) We need index the metadata for the streams but the metadata available is not enough
2) We need to expand the metadata, but how?
3) name expansion --> tag unification
4) timeseries feature extraction --> tag unification

top things to expand upon:  location, type, system
secondary: statistical features about the data
 
