\section{Introduction}

Buildings are sites of very large sensor deployments, typically containing
up to several thousand sensors reporting physical measurement, continuously.
Moreover, with the recent interest in reducing building energy consumption, it
is important to consider ways to quickly bootstrap a set of building data streams
into an anlytical pipeline to determine where there are opportunities for energy savings,
discovery of broken sensors, and assessing and tracking overall building performance.
However, current `point' naming conventions form a bottleneck in the scalability of
the data integration process.  A `point' refers to a physical location where
a sensor is taking measurements. Each building vendor uses their own naming scheme and
uniques variants of each scheme are implemented from building to building; variations exist
even across buildings that have contracted the same vendor to set up their deployment.
This makes the integration process laborious and fundamentally unscalable.  If we
wish to have a broad impact across the entire building stock, at scale, we need
to explore methods for overcoming this challenge.

A simple analysis application might include the ability
to identify anomalous readings from a specific kind of sensor.  In order to run the application
we need to know what the name of the sensor(s) is and how to attain readings from that 
stream.  If the analysis is for a particular kind of system, then the name is used to refer to 
only the set of streams that the application knows how to process.  Even within a single building,
the stream identification process is manual.  The deployer looks loads the interface to the 
building management system, track down the spatial or system view, clicks through several windows
to locate the location of the point(s) of interest, mouses over the point(s) and records the name,
and then uses that point name to request it from the data-fetch protocol -- typically BACNet or 
LonTalk or another open standard.  This process is repeated in \emph{every building} that wants
the application deployed.

In order to meaningfully deal with disparate building data streams in a scalable 
fashion the streams should be \emph{searchable} across various properties, such
as building name, room location, and statistical trends.  Moreover, we
assert that wide searchability is necessary for achieving scalability.  By providing a tool for
searching across building streams, we minimize the deployment time for applications that 
offer deeper insight into building performance -- allowing application  to be used in \emph{all}
buildings, not just a single one.


We observe that every naming scheme looks to capture three point attributes: 
1) the location in space, 2) its relation to an subsytem, and 3) the type of 
measurement it is taking.  

We want to make the streams searchable.  How do we do that?
1) We need index the metadata for the streams but the metadata available is not enough
2) We need to expand the metadata, but how?
3) name expansion --> tag unification
4) timeseries feature extraction --> tag unification

top things to expand upon:  location, type, system
secondary: statistical features about the data
 
