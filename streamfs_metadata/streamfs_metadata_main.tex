%\documentclass[10pt,conference]{IEEEtran}
%\documentclass[9pt,conference]{sig-alternate}
%\documentclass[9pt,conference]{sig-alternate}
%\documentclass[10pt,print,letterpaper,nocopyrightspace]{sigplan-proc-varsize}
\documentclass[10pt,print,letterpaper]{sigplan-proc-varsize}

\usepackage{amsmath,epsfig}
\usepackage{url}
\usepackage{xspace}
\usepackage{colortbl}
\usepackage{subfigure}
\usepackage{dsfont}
\usepackage{boxedminipage}
\ifx\pdfoutput\undefined
\usepackage[hypertex]{hyperref}
\else
\usepackage[pdftex,hypertexnames=false]{hyperref}
\fi

\usepackage{amssymb}
\usepackage{wasysym}
\usepackage[left=2.54cm,top=2.54cm,right=2.54cm,bottom=2.54cm,nohead,nofoot]{geometry}



\DeclareMathOperator*{\argmax}{argmax}
%\usepackage{times}

\def\ucb{$^{\dagger}$}
\def\stanford{$^{\ddagger}$}
\def\arch{$^{\star}$}

\newcommand{\kb}{kB}
\newcommand{\rene}{Ren{\'e}\xspace}
\newcommand{\reneii}{Ren{\'e}2\xspace}
\newcommand{\wec}{WeC\xspace}
\newcommand{\mica}{Mica\xspace}
\newcommand{\micaii}{Mica2\xspace}
\newcommand{\micaz}{MicaZ\xspace}
\newcommand{\micadot}{Mica2Dot\xspace}
\newcommand{\iic}{I$^2$C\xspace}
\newcommand{\uA}{$\mu$A\xspace}
\newcommand{\dotmote}{Dot\xspace}
\newcommand{\mhz}{MHz\xspace}
\newcommand{\ghz}{GHz\xspace}
\newcommand{\kbps}{kbps\xspace}
\newcommand{\dsn}{DSN\xspace}
\newcommand{\io}{I/O\xspace}
\newcommand{\telos}{Telos\xspace}

\newcommand{\T}{\mathds{T}}
\newcommand{\XXXnote}[1]{{\bf\color{red} XXX: #1}}


\begin{document}

%\conferenceinfo{Sensys'08,} {November 5--7, 2008, Raleigh, NC, USA.}  
%\CopyrightYear{2008} 
%\crdata{978-1-60558-096-8/08/09} 

\title{StreamFS:\\An analytical framework for physical data}
%\numberofauthors{1} 
%\author{\alignauthor Stephen Dawson-Haggerty, Jorge Ortiz, Xiaofan Jiang and David Culler\\
%\affaddr{Computer Science Division}\\
%\affaddr{University of California, Berkeley} \\ 
%\affaddr{Berkeley, California 94720} \\
%\email{\{stevedh,jortiz,xjiang,culler\}@cs.berkeley.edu}
%} 


%\subtitle{Paper \# Insert Reg Number Here}

%\title{Alternate {\ttlit ACM} SIG Proceedings Paper in LaTeX
%Format\titlenote{(Produces the permission block, and
%copyright information). For use with
%SIG-ALTERNATE.CLS. Supported by ACM.}}
%\subtitle{[Extended Abstract]
%\titlenote{A full version of this paper is available as
%\textit{Author's Guide to Preparing ACM SIG Proceedings Using
%\LaTeX$2_\epsilon$\ and BibTeX} at
%\texttt{www.acm.org/eaddress.htm}}}
%
% You need the command \numberofauthors to handle the 'placement
% and alignment' of the authors beneath the title.
%
% For aesthetic reasons, we recommend 'three authors at a time'
% i.e. three 'name/affiliation blocks' be placed beneath the title.
%
% NOTE: You are NOT restricted in how many 'rows' of
% "name/affiliations" may appear. We just ask that you restrict
% the number of 'columns' to three.
%
% Because of the available 'opening page real-estate'
% we ask you to refrain from putting more than six authors
% (two rows with three columns) beneath the article title.
% More than six makes the first-page appear very cluttered indeed.
%
% Use the \alignauthor commands to handle the names
% and affiliations for an 'aesthetic maximum' of six authors.
% Add names, affiliations, addresses for
% the seventh etc. author(s) as the argument for the
% \additionalauthors command.
% These 'additional authors' will be output/set for you
% without further effort on your part as the last section in
% the body of your article BEFORE References or any Appendices.

\numberofauthors{2} %  in this sample file, there are a *total*
% of EIGHT authors. SIX appear on the 'first-page' (for formatting
% reasons) and the remaining two appear in the \additionalauthors section.
%
\author{
% You can go ahead and credit any number of authors here,
% e.g. one 'row of three' or two rows (consisting of one row of three
% and a second row of one, two or three).
%
% The command \alignauthor (no curly braces needed) should
% precede each author name, affiliation/snail-mail address and
% e-mail address. Additionally, tag each line of
% affiliation/address with \affaddr, and tag the
% e-mail address with \email.
%
% 1st. author
\alignauthor
%Prabal Dutta\\
%       \affaddr{Computer Science Division}\\
%       \affaddr{Univ. of California, Berkeley}\\
%       \affaddr{Berkeley, CA 94720}\\
%       \email{prabal@cs.berkeley.edu}
% 2nd. author
%\alignauthor
%David Culler\\
%       \affaddr{Computer Science Division}\\
%       \affaddr{Univ. of California, Berkeley}\\
%       \affaddr{Berkeley, CA 94720}\\
%       \email{culler@cs.berkeley.edu}
% 3rd. author
%\alignauthor
%Scott Shenker\\
%       \affaddr{Computer Science Division}\\
%       \affaddr{Univ. of California, Berkeley}\\
%       \affaddr{Berkeley, CA 94720}\\
%       \email{shenker@cs.berkeley.edu}
%}
Jorge Ortiz\\
       %\affaddr{Department}\\
	\affaddr{Computer Science Division}\\
       \affaddr{University of California, Berkeley}\\
       %\affaddr{City, State Zip}\\
       \email{jortiz@cs.berkeley.edu}
}


\maketitle

%\date{14 April 2007}
%\maketitle

\begin{abstract}
Despite the growing impact of climate change and energy prices, 
per-capita energy consumption is rising. Part of the problem is visibility. We do not 
have scalable means of observing our energy consumption patterns and determining how to optimize and reduce our
consumption.
Mobile smartphones present a unique opportunity to enable an energy view on the physical world. 
They can bridge the physical world, information infrastructure, and people
through a rich set of sensors, ubiquitous connectivity, and highly personal user interface. 
With QR codes as cheap tags on items and places in the physical world, the
camera becomes a portable scanner in your pocket, in addition to its
traditional functions.  We explore this
unique triple point
and re-examine classical problems of context and consistency management in mobile
systems.  We also examine this combination as it pertains to energy management of physical
devices.  In doing so, we are re-introduced to problems of apportionment and aggregation of sensor data,
except with a continuously changing set of constituents.  We describe our solution in a technique
called \emph{dynamic aggregation} that maintains moving aggregates as the
set of data sources changes over time.  We deployed our system in a 
141,000 square-foot building, tagging 351 items over 139 room across 7 floors.

% When combined with QR codes, the on-board camera provides us with a portable scanner

% The camera,
% when combined with QR codes, gives us a portable scanner and convenient mechanism for tying these world together. 
% In this paper, we describe our system and deployment experience for a mobile phone application the provides 
% user-centric energy-view of the physical world. We describe the challenges, specifically dealing with mobility, 
% and how we address them in a set of three separate applications: an energy auditing application, a 
% device energy scanner, and a personal energy counter. We also introduce a technique called \emph{dynamic aggregation}
% which allows us to seamlessly track the constituents of aggregated energy calculations, as they move from one 
% location to another.

% Despite the recent impact of global warming and a steady increase in energy prices, 
% per-capita energy consumption is rising. Part of the problem is about visibility. We simply do not 
% have any good ways of seeing how we consume energy, and therefore, how to optimize and reduce it. 
\end{abstract}

% A category with the (minimum) three required fields
%\category{B.0}{Hardware}{General}
%\category{B.4}{Hardware}{Input/Output \& Data Communication}

%\terms{Design, Implementation, Performance, Experimentation}

%\keywords{Churn, Link, Routing, Wireless, Sensor Network, Mote}

%\newpage

\subsection{Introduction}
The United States leads the world in per-capita energy consumption.
Our electricity use has consistently increased over the last 40 years~\cite{oecd2011} and other parts of the world are rising all 
too rapidly.  With the specter of climate change and the increasing cost of energy, we must explore new
ways for individuals to gain visibility and insight into their energy consumption in order to optimize and reduce it. 
With the increasing penetration of embedded sensors in the environment and
the continued rise in smartphone adoption, we see an opportunity for smartphones to bridge the physical world
to our computational infrastructure and provide an `energy lens' on the physical world.  

We use mobile phones to construct an entity-relationship 
graph of the physical world and combine it with streaming sensor data in order to perform detailed energy-attribution.
We limit the scope of the world to a single building domain.  We have designed and implemented a real-time, mobile energy auditing
application, called the `Energy Lens', that allows us to collect information about 
things throughout the building and how they are related to each other.  For example, computer X is inside 
room Y and connected to meter Z.  Then, we use these relationships to guide our data look-up and analytical
calculations.  For example, the load curve of room Y consists of the sum of all the power traces for loads
inside room Y.  We use the mobile smartphone as the main input tool.  Our work examines \emph{three main challenges} in setting up and 
deploying a real, whole-building infrastructure to support real-time, 
fined grained energy analytics.  

The first challenge is related to tracking and mobility.
The use of mobile phones presents classical, fundamental challenges related to mobility.  Typically, mobility
refers to the phone, as the person carrying it moves from place to place.  However, in the energy-attribution
context, we are also referring to the movement of energy-consuming objects.  Tracking their relationships to spaces 
and people is as important as tracking people.  We describe how we deal with \emph{both moving people and 
moving objects} and show that these historically difficult problems can be addressed relatively easily, if the proper infrastructure is 
in place.  %We provide evidence that the approach is simple, incrementally deployable, and scalable.

The second challenge is about capturing the inter-relationship semantics and having these inform our  analytics.
We adopt the general notion of physical tags that identify objects in the world.  Our system uses \emph{QR codes} to tag things and locations 
in the physical world.  However, \emph{any tag that provides a unqiue identifier for an object could serve the same purpose}.
Once tagged, there are three types of interactions -- 
registration, linking, and scanning -- which establish important relationships.  Registration is the act of creating a virtual object 
to represent a physical one.  Linking captures the relationship between pairs of objects.  Scanning is the act of performing an item-lookup.
Each of these interactions requires a set of swiping gestures.  Linking requires two tag swipes while the other two actions
require a single tag swipe.  Internally, we maintain a \emph{entity-relationship graph (ERG)} of things, people, and locations, that gets
updated through these sets of gestures.

The third challenge is about indoor network connectivity and access.
In order to connect these components, we rely on having `ubiquitous' network connectivity.  However, in practice, network
\emph{availability} is intermittent and our system must deal with the challenges of intermittency.  We discuss how caching
and logging are used to address these challenges.  Moreover, when connectivity is re-established, we must deal with
applying updates to the ERG, as captured by the phone while disconnected.  
% Conflicts can also occur during an update.  For example, the two updates may disagree about which items are attached
% to which meters.  We implement a very simple conflict resolution scheme, described in section~\ref{sec:conflicts}.
% Finally, certain physical-state transitions are represented as a set of updates to the ERG that must be applied 
% atomically.  We implement transactions in the log-replay and transaction manager.
% Our `Energy Lens' system is deployed in a building on our campus.  We discuss
% its architecture and our design choices.  
  
% We also discuss novel strategies for tracking moving people/things and describe how we implement these in our system.  In summary, our work
% makes the following contributions:

% \begin{itemize}
% \item We design and implement a system that captures and combines physical entities, their inter-relationships, and real-time sensor data 
% 		in buildings.% using mobile phones, qr code, and a cloud-based infrastructure.
% \item We observe that certain combinations of swipes give us useful information to set the location of people and things over time.
% 		We codify this observation in our \emph{context-tracker} and use it to maintain consistency between the entity-relationship graph and the 
% 		state of the physical world.  To the best of our knowledge, this is radically different from the approaches in standard 
% 		localization techniques.  However, we argue that it can be used to \emph{enhance} their accuracy and overall performance.
% \item We implement a prefetching algorithm to obtain context-dependent information to both improve performance and
% 		enable disconnected operation.  We also design and implement a log-replay and transaction manager over our data management layer.  We describe how different conflict-resolution policies can be implemented and our rationale for the policies we chose.
% \end{itemize}

% \vspace{0.08in}

% In the next sections we go through a motivating scenario.  We then discuss some related work, followed 
% by the system architecture, evaluation, and future directions.

\section{Motivation}

As networked, embedded sensing becomes more and more ubiquitous, it becomes important to formulate new ways to use the data produced by these devices.  One can view these devices as fundamentally linked to the physical world, reporting on physical measurement taken in the context of where they are placed.  For example, a sensor deployment on the Golden Gate bridge \cite{GGB} were used to monitor its structural health over time.  By taking fine-grained, high-frequency accelerometer readings, the authors were able to formulate a time-varying view of the bridge as a whole.  Another important application is in buildings, where sensors on embedded throughout the building environment to measure the ambient condition and to monitor the health of the equipment used to maintain safety, comfort, and security.  In either application, the placement of sensors must be known a priori in order to construct a wholistic view of the phenomena being observed.

The building context is particularly challenging in this regard.  The GGB project, for example, used 64 nodes spread 
throughout the bridge.  The location of each node was carefully recorded.  Buildings, on the other hand, typically have 
over an order of magnitude more sensors, distributed throughout the building.  Sutardja Dai Hall contains over 3000 
sensors spread throughout the building, and the number of sensors typically increases with the size (in square feet) 
of the building.  In each case, the location, type, and other information is important to record.  This metadata, is 
crucial for data interpretation.  Moreover, the relationship between the sensors, as described through the metadata, 
serves an even more critical role, as it allows the analyst to construct a holistic view/interpretation of the data.  
Without it, the data is useless.

{\bf Observation 1}:  The metadata that describes the context of the embedded device, is as important as the data the 
device produces.

{\bf Observation 2}:  The metadata must be normalized, for each deployment, in order to formulate a holistic interpretation 
of the measurements.

The phenomena being sensed is time-varying in nature.  Each data value has an associated timestamp and there’s some work 
must be done to either synchronized, normalize, or align timestamps across streams before analysis.  However, the real 
challenge lies in long-lived deployments.  Long-lived are challenging because the physical environment changes and these 
changes are difficult to track, at scale.  For example, buildings have lifespans that last multiple decades.  In that 
time, the environment and placement of sensors in them, goes through many, many changes.  In order to maintain an accurate, 
continuous assessment of the environment, such changes must be tracked.

{\bf Observation 3}:  Changes in the physical environment must be systematically tracked in order to maintain an accurate 
interpretation of the data produced by sensors embedded in that environment.

This observation is corroborated in several experimental deployments [references?] and captured in simulation engines, 
such as EnergyPlus, explicitly.  Most deployments collect data throughout the lifetime of the deployment and do both real-time 
and historical analysis of the data.  Designing for the physical changes, then, has deep implications on the design of the system.  
The first is that there should be mechanisms in place either 1) automatically describe the environment/placement, type, etc. and 
2) there are verification processes that continuous check the validity of the metadata.  Both are necessary to assure the accuracy 
of the analysis.  Moreover, 3) history must be recorded and accounted for throughout the lifetime of the deployment.  This assures 
that the analyst will place the data in the appropriate context, even as she runs through the historical data.

Finally, soft context should also be systematically tracked.  Software changes can cause changes in stream behavior that are an 
actual source of error during analysis.  This should also be captured in the metadata history.

{\bf Observation 4}: Metadata history is necessary in order to properly audit the environment and accurately perform a historical 
analysis.

We have designed a system, StreamFS, that address each of these observations for sensor deployments.  However, for this paper we focus on
the portion of the system that specifically addresses observations 1, 2, and 4.  We discuss the part of the system that addresses
observation 3, but refer the reader to other papers that specifically address this issue.  We also describe the use of StreamFS for
managing sensor deployments in buildings.  We integrate StreamFS into the Building Applciation Stack~\cite{BAS} and benchmark it i
a real world deployment.

\section{Related work}

All the data values are compressed using snappy~\cite{snappy}, before insertion into the database.

\begin{enumerate}
\item Git version control
	\begin{itemize}
	\item any change in the metadata is ``committed''
	\item tags name commits
	\item SHA-1 is used to save the contents of a commit operation (i.e. the struct that described the operation)
	\end{itemize}
\item Log-structured file system
	\begin{itemize}
	\item all changes made to the metadata is recorded as a timestamped operation in the database
	\item the details of the operation are stored elsewhere and \emph{only} fetched if explicitly queried
	\item The log is replayed from the start time to the end time of a query 
	\end{itemize}
\item provenance database systems
\item timeseries databases systems
\item BAS and BOSS
\item spatio-temporal databases
\end{enumerate}



\section{Motivating example}

Indoor localization work is on the rise again~\cite{papers}.  In buildings, the community has realized that coarse 
grained (room-level) localization is sufficient for most applications.  Although challenges remain in 
boundary-discovery~\cite{papers} and auto-calibration~\cite{papers}.

\section{Data}
In buildings, we collect data about the internal sub-systems, the spatial organization, and the sensors throughout the 
infrastructure.  We also collect data from sensors and use the infrastructural information to organize and categorize
the data collected from the sensors.  For example, we note which temperature sensors are in which room and which pumps,
cooling coils, fans are driven by their readings.  The former denotes physical placement, while the latter denotes
the control-loop relationship.  We also categorize according to various classification schemes~\cite{paper} 
which help us understand the data better.

\section{Queries}

% metadata search 
We might just want to get a list of all sensors of a certain type that are manufactured by Seimens.

%graph + timeseries
We might want to ascertain the temperature distribution on the first floor
of a building, or determine the average temperature for all rooms controlled by a particular heat-pump.  The former allows the
user to examine how well the HVAC system is maintaining a consistent climate while the latter can be used to determine if
there are any problems in the system by seeing which rooms are driving the energy consumption for that heat-pump.

%graph + timeseries + provenance
We are seeing a move towards mobile sensing~\cite{paper}.  Sensors produce a stream of readings,
temporally associated with locations in the building.  If we have a CO2 sensors on every occupant's phone 
and we have wifi to determine coarse-grained association with locations in the building, then the query that determines the
CO2 distribution at a particularly location becomes non-trivial, as it requires the association history for segments of the 
reading-stream produced over a certain time interval.  For a given time interval, we must determine all the phones
that passed through a location and gather only the segment in those intervals when the phone produced readings at that location.
%graph + timeseries + provenance + metadata search
We can narrow it further, by adding a filter by owner.  So if i want to know the CO2 distribution experience by me at a particular
location in the building for the coarse of a week, I would filter the list according to an ownership tag on the stream.

The queries we ask fall into three categories:
\begin{enumerate}
\item Graph queries
\item timeseries queries
\item metadata search
\end{enumerate}

timeseries queries imply consistent association between inter-relationship state and labels -- all change over time

By exposing the referential namespace, we expose the vocabulary and relationships between the objects, explicitly.	

\section{Implementation}

When you get information about a node, the children bins are labeled by type.  The type information can
be fetched from the filesystem as well in {\tt /rel}.

\section{Timeseries metadata}
On create, increase the version number of all entries and set the initial version for the newly created entry.
On an update, increase the version for all entries.  On delete, increase the version of all node except the one that
is deleted.


\section{Rationale}
We need the ability to integrate external applications easily.  Therefore we implement a POSIX-compliant FS interface.

Hierarchical naming and symbolic linking can express a directed graph.  Symbolic links also enable 
aliasing -- multiple names to refer to a object.

\section{Queries: Graphical, historical, histgraphical?}
The building can be represented as a collection of objects and the inter-relationship between them.

Typically the vocabulary is unknown, only general descriptions tend to be known beforehand.  We expose the inter-relationships
through hierarchical naming and symbolic links.  We also also allow arbitrary tags on the nodes.  These can be used to 
support general name search and queries based on general descriptions.
For example, a building analyst may want to know how many occupants were the chillers in the building or
to compare the relative occupancy load handled by the chillers versus the heaters, without having knowing specifically
which how each chiller/heater is physically connected to each space in the building and without knowing exactly when/where
each occupant was throughout the day.
Another, perhaps even more important query, is one related to mobile sensors -- i.e. ambient sensors on a mobile 
phone carried by occupants.  What's the distribution of temperature for the chillers versus the heaters, as experienced
by the occupants throughout the day.

Lets take an example in a different domain.  We could imagine a mobile environmental sensing application that associates individuals
with different locations (based on coordinate-based boundary conditions) as you walk around and passively collect environmental
readings, such as sound, light, Co2, temperature, etc.  A example query in this domain is ``what is the average Co2 level
in location 1 from time 1 to time 2?'' or ``what's the path with the smallest polution between location 1 and location 2 between time 1 and
time 2?''.

In manufactoring applications tracking context and associations over time is also necessary.  They ask questions like 
``which batches of powder are being used to form the ceramics at the heart of the batteries, how high a temperature is being used to bake them, how much energy is required to make each battery, and even the local air pressure.''~\cite{intman}.  In some cases, there is only a single
power meter taking measurement for a multi-stage machine with other sensors at each stage.  A related energy question, then, is 
what's the total power drawn by this machine for a specific batch of widgets -- requiring a search through the database to deduce
an indirect relationship between the meter and the widget over a certain time interval.

In a traditional database, these would have to be constructed iteratively.  We would have to determine the equipment and sensors involved, 
and their inter-relationship.  We would have to determine when that relationship existed and for how long and use those intervals 
to fetch the associated sensor readings for processing.  The implication for each is that a relationship history is maintained \emph{along with}
timeseries data produced by sensors.  In addition, this process implies the need for the end-user to know the names of the devices in order
to query for them explicitly, since the relationship between the equipment, the sensors, and other objects is a uniquely component in
the aggregate analysis.

\vspace{+0.5mm}
\vspace{+2mm}
\bibliographystyle{abbrv}
%\scriptsize
\bibliography{references}
%\bibliography{references}

\end{document}


