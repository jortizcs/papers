\section{Related Work}

In \cite{occmodels_buildsys11}, the authors present a multivariate Gaussian modeling parameter-fitting 
approach to fit the parameters of an occupancy model to match the occupancy data with a small data set.  
The model is thenused to drive HVAC settings to reduce energy consumption.  We ignore occupancy entirely 
in our approach.  Instead it is an implicit factor in correlated-usage patterns; changes in such a pattern 
should signal changes in occupancy that may affect efficiency.

\cite{kaminthermo} uses occupancy sensor from the in-home security system and user input to
optimize the heating and cooling schedule in the home.  Using occupancy statistics over time, they
are able to determine the times when the someone will be home and they set the heating schedule accordingly.
Our approach is unaware of what is optimal, only what is common.  Uncommon behavior is flagged in hopes
that they are meaningful with respect to efficiency.

\cite{Bellala_buildsys11} looks at various buildings to develop a model of efficient power usage using 
an unsupervised learning technique coupled with a Hidden Markov Model (HMM).  They also develop occupancy models based on
computer network port-level logs, to help determine more efficient management policies for lighting and HVAC.  
They claim a savings of 9.5\% in lighting on a single floor.

\cite{kim:buildsys2010} uses branch-level energy monitoring and IP traffic from user's PCs to determine the
causal relationships between occupancy and energy use.  Their approach is most similar to ours.  Understanding how IP 
traffic, as a proxy for occupancy, correlates with energy use can help determine where inefficiencies may lie.

%{\bf Anomaly Detection Using Projective Markov Models in a Distributed Sensor Network~\cite{buildanomaly}}

%{\bf Duty-cycling buildings aggressively: The next frontier in HVAC control}~\cite{AgarwalBDGW11}

In each of these studies and others~\cite{AgarwalBDGW11,buildanomaly}, cccupancy is used as an trigger
that drives efficient resource-usage policies.  Efficiency
when unoccupied means shutting everything off and efficiency when a space is occupied means anything
can be turned on.  There is no question this is an excellent way to identify savings opportunities, however, we
take a fundamentally different approach.  We are agnostic to the underlying cause or driver for efficient
policies to be implemented.  More generally, we look to understand \emph{how the equipment is used in
concert}.  This may help uncover unexpected underlying relationships and can be used in an anomoly detection application
to establish ``efficient'', ``normal'', or ``inefficient'' usage patterns.  The latter 
should identify savings opportunities in cases where the space is unoccupied as well 
as occupied, because it has to do with the underlying behavior of the machines and how they generally work
together.  This kind of application could achive both generality and scale.
% Our approach is flexible to be helpful when the assumption does not hold by either allowing the
% user to specify what efficient usage is or through the discovery of efficient usage over various time scales.
% In either case, the flexibility of our approach is its strength from the perspective of generality and scale.



