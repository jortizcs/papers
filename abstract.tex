\begin{abstract}
In this paper we investigate the utility of empirical mode decomposition (EMD) to identify intrinsically
correlated usage patterns among sensors in a large deployment.  We use data collected from almost $700$
sensors in a 12-story building measuring power, pressure, temperature, and other physical
phenomena.  We discover that doing a correlation analysis on the raw traces does not discriminate well enough
to identify meaningful relationships between sensors.  We correlate the trace from a pump with the rest of
the sensor traces and find that simple correlation filters only $50\%$ of the sensors as being correlated
with the behavior of the pump. 
In contrast, by running the correlation analysis on the constituent frequencies extracted by
the EMD process, we filter out over $99\%$ of the sensors as being correlated -- with the highest correlation coming 
from sensors that serve the same room as the pump.  We believe our approach can be used to 
construct inter-device correlation models that can help understand and identify misbehaving or inefficient
usage patterns.
\end{abstract}

