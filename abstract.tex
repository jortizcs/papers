\begin{abstract}
In this paper we investigate a general approach for identifying opportunities for energy savings
in real building data traces.  

We use a technique called Empirical Mode Decomposition (EMD) to
compare sensors at overlapping underlying frequencies and find that 
\end{abstract}

Results:
\begin{enumerate}
\item Correlation is noisy and can't find inter-relationships between sensors
		with subtle differences.
\item Underlying behavior should extract most-common denominator in comparing traces
		to observe truly correlated behavior.
\item Empirical mode decomposition (EMD) can be used to compare underlying behavior after the
		removal of the dominant frequencies in the signal.
\end{enumerate}

Future work:
\begin{enumerate}
\item We can create a time-varying dependency graph to compare ``normal'' versus ``abnormal'' behavioral
		patterns in underlying use.
\item We can codify ``normal'' or ``efficient'' graphs and compare with real graph constructs over time.
\end{enumerate}