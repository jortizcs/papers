\begin{abstract}
In this paper we investigate the utility of empirical mode decomposition (EMD) to identify intrinsically
correlated usage patterns among sensors in a large deployment.  We use data collected from almost 700
sensors in a 12-story building measuring power, pressure, temperature, and other physical
phenomenon.  We discover that doing a correlation analysis on the raw traces does not discriminate well enough
to identify meaningful relationship between sensors.  We correlate the trace from a pump with all other
traces and find that
correlation alone identifies over $50\%$ of the sensors as being correlated.  In contrast, by running 
the correlation analysis 
on the EMD output, we identify $< 1\%$ of the sensors as being correlated -- with the highest correlation coming 
from sensors that serve the same room as the pump.  We believe our approach can be used to 
construct inter-device correlation models that can help understand and identify misbehaving or inefficient
usage patterns.

% In our analysis, we 
% find that correlation on raw sensor data in a build

% We find that correlation of the raw
% data alone is unstable because of the underlying phenomena that drives the sensor readings.
% We use a deployment over for 650 sensors in a 12-story building on a university campus and find
% some statistics to summarize how it compares with EMD + Correlation and how it relates to spatio-temporal
% relationships.
\end{abstract}

