\begin{abstract}
In this paper we investigate the utility of emperical mode decomposition (EMD) to identify intrinsically
correlated usage patterns among sensors in large sensor deployments.  We use data collected from almost 700
sensors in a 12-story building measuring power, pressure, temperature, and various other physical
phenomenon.  We discover that doing a correlation analysis on the raw traces is not discriminatory enough
to identify meaningful relationship between sensors.  We correlate the trace from a single pump and find that
correlation alone identifies over 50\% sensors as being highly correlated.  By running the correlation analysis 
on the EMD output, we identify $< 1\%$ of the sensors to be correlated -- with the highest correlation coming 
from a sensor in the same room as the main sensor we compare against.  We believe our approach can be used to 
construct inter-device correlation models that can help us understand and identify misbehaving or inefficient
usage patterns.

% In our analysis, we 
% find that correlation on raw sensor data in a build

% We find that correlation of the raw
% data alone is unstable because of the underlying phenomena that drives the sensor readings.
% We use a deployment over for 650 sensors in a 12-story building on a university campus and find
% some statistics to summarize how it compares with EMD + Correlation and how it relates to spatio-temporal
% relationships.
\end{abstract}

