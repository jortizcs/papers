%\documentclass[10pt,conference]{IEEEtran}
%\documentclass[9pt,conference]{sig-alternate}
%\documentclass[9pt,conference]{sig-alternate}
%\documentclass[10pt,print,letterpaper,nocopyrightspace]{sigplan-proc-varsize}
%\documentclass[10pt,print,letterpaper]{sigplan-proc-varsize}
\documentclass[10pt,print,letterpaper]{sensys-proc}

\usepackage{amsmath,epsfig}
\usepackage{url}
\usepackage{xspace}
\usepackage{colortbl}
\usepackage{subfigure}
\usepackage{dsfont}
\usepackage{boxedminipage}
% \ifx\pdfoutput\undefined
% \usepackage[hypertex]{hyperref}
% \else
% \usepackage[pdftex,hypertexnames=false]{hyperref}
% \fi

\usepackage{amssymb}
\usepackage{wasysym}
\usepackage[left=2.54cm,top=2.54cm,right=2.54cm,bottom=2.54cm,nohead,nofoot]{geometry}
\usepackage{algorithm}
\usepackage{algpseudocode}
\usepackage{listings} 


\DeclareMathOperator*{\argmax}{argmax}
%\usepackage{times}

\def\ucb{$^{\dagger}$}
\def\stanford{$^{\ddagger}$}
\def\arch{$^{\star}$}

\newcommand{\kb}{kB}
\newcommand{\rene}{Ren{\'e}\xspace}
\newcommand{\reneii}{Ren{\'e}2\xspace}
\newcommand{\wec}{WeC\xspace}
\newcommand{\mica}{Mica\xspace}
\newcommand{\micaii}{Mica2\xspace}
\newcommand{\micaz}{MicaZ\xspace}
\newcommand{\micadot}{Mica2Dot\xspace}
\newcommand{\iic}{I$^2$C\xspace}
\newcommand{\uA}{$\mu$A\xspace}
\newcommand{\dotmote}{Dot\xspace}
\newcommand{\mhz}{MHz\xspace}
\newcommand{\ghz}{GHz\xspace}
\newcommand{\kbps}{kbps\xspace}
\newcommand{\dsn}{DSN\xspace}
\newcommand{\io}{I/O\xspace}
\newcommand{\telos}{Telos\xspace}

\newcommand{\T}{\mathds{T}}
\newcommand{\XXXnote}[1]{{\bf\color{red} XXX: #1}}


\begin{document}

\crdata{978-1-4503-1170-0}
\conferenceinfo{Buildsys'12,} {November 6, 2012, Toronto, ON, Canada.}
\CopyrightYear{2012}

\title{StreamFS: Metadata A First Class Citizen}
%\title{Systems challenges in real-time, fined-grained energy analytics through mobile phones}
%\title{Mobility and consistency management in real-time, fined-grained energy analytics for buildings through with mobile phones}
\numberofauthors{1} 
\author{\alignauthor Jorge Ortiz and David Culler\\
\affaddr{Computer Science Division}\\
\affaddr{University of California, Berkeley} \\ 
%\affaddr{Berkeley, California 94720} \\
\email{\{jortiz,david.su,culler\}@cs.berkeley.edu}
} 


%\subtitle{Paper \# Insert Reg Number Here}

%\title{Alternate {\ttlit ACM} SIG Proceedings Paper in LaTeX
%Format\titlenote{(Produces the permission block, and
%copyright information). For use with
%SIG-ALTERNATE.CLS. Supported by ACM.}}
%\subtitle{[Extended Abstract]
%\titlenote{A full version of this paper is available as
%\textit{Author's Guide to Preparing ACM SIG Proceedings Using
%\LaTeX$2_\epsilon$\ and BibTeX} at
%\texttt{www.acm.org/eaddress.htm}}}
%
% You need the command \numberofauthors to handle the 'placement
% and alignment' of the authors beneath the title.
%
% For aesthetic reasons, we recommend 'three authors at a time'
% i.e. three 'name/affiliation blocks' be placed beneath the title.
%
% NOTE: You are NOT restricted in how many 'rows' of
% "name/affiliations" may appear. We just ask that you restrict
% the number of 'columns' to three.
%
% Because of the available 'opening page real-estate'
% we ask you to refrain from putting more than six authors
% (two rows with three columns) beneath the article title.
% More than six makes the first-page appear very cluttered indeed.
%
% Use the \alignauthor commands to handle the names
% and affiliations for an 'aesthetic maximum' of six authors.
% Add names, affiliations, addresses for
% the seventh etc. author(s) as the argument for the
% \additionalauthors command.
% These 'additional authors' will be output/set for you
% without further effort on your part as the last section in
% the body of your article BEFORE References or any Appendices.

% \numberofauthors{2} %  in this sample file, there are a *total*
% % of EIGHT authors. SIX appear on the 'first-page' (for formatting
% % reasons) and the remaining two appear in the \additionalauthors section.
% %
% \author{
% % You can go ahead and credit any number of authors here,
% % e.g. one 'row of three' or two rows (consisting of one row of three
% % and a second row of one, two or three).
% %
% % The command \alignauthor (no curly braces needed) should
% % precede each author name, affiliation/snail-mail address and
% % e-mail address. Additionally, tag each line of
% % affiliation/address with \affaddr, and tag the
% % e-mail address with \email.
% %
% % 1st. author
% \alignauthor
% %Prabal Dutta\\
% %       \affaddr{Computer Science Division}\\
% %       \affaddr{Univ. of California, Berkeley}\\
% %       \affaddr{Berkeley, CA 94720}\\
% %       \email{prabal@cs.berkeley.edu}
% % 2nd. author
% %\alignauthor
% %David Culler\\
% %       \affaddr{Computer Science Division}\\
% %       \affaddr{Univ. of California, Berkeley}\\
% %       \affaddr{Berkeley, CA 94720}\\
% %       \email{culler@cs.berkeley.edu}
% % 3rd. author
% %\alignauthor
% %Scott Shenker\\
% %       \affaddr{Computer Science Division}\\
% %       \affaddr{Univ. of California, Berkeley}\\
% %       \affaddr{Berkeley, CA 94720}\\
% %       \email{shenker@cs.berkeley.edu}
% %}
% %%%Jorge Ortiz, Yongwoo Noh, Gavin Saldanha, David Su, Jason Trager, David Culler, and Paul Wright\\
%        %\affaddr{Department}\\
% 	%%%\affaddr{Computer Science Division}\\
%        %%%\affaddr{University of California, Berkeley}\\
%        %\affaddr{City, State Zip}\\
%        %%%\email{jortiz@cs.berkeley.edu}
% }


\maketitle


% \begin{abstract}
% In this paper we present a system called the Energy Lens -- a system that provides  
% deeper, real-time visibility of plug-load energy consumption in buildings.  Our initial work focuses
% on plug-load power metering, display, and real-time aggregation, presented to the user through
% a mobile phone.  We discuss the three main, non-trivial challenges that must be addressed to provide
% real-time energy analytics in buildings through mobile phones and our initial approach towards addressing 
% each challenge.
% \end{abstract}

% \category{H.4}{Information Systems Applications}{Miscellaneous}
% %A category including the fourth, optional field follows...
% %\category{D.2.8}{Software Engineering}{Metrics}[complexity measures, performance measures]
% \terms{Distributed Consistency, Mobile application}
% \keywords{Disconnected operations, network access, energy, power, plug loads}

\subsection{Introduction}
Buildings consume an enormous amount of energy in countries around the world.  In 
Japan, 28\% of the energy produced is consumed in buildings~\cite{japanbuildings} while in the United 
States it is as high as 40\%~\cite{epabuildings}.  Moreover, studies show that between 30-80\% of it
is wasted~\cite{waste_science, next10_waste}.  Large commercial buildings are typically instrumented
with a large number of sensors measuring various aspects of building operation.  Although this data is
typically used to assure operational stability, they may also be used to measure, observe, and identify
instances of wasted use.

Identifying instances of wasted energy use is non-trivial.  System efficiency is defined as the ratio of the 
useful work done to the energy it consumes.  In the case of buildings, we broadly define useful work as 
the energy used to support occupant activities.  From the perspective of the building that means maintaining
a comfortable temperature setting, providing power for plug-load devices, and providing adequate lighting
conditions; particularly in spaces that are occupied.  However, identifying efficient use of resources,
\emph{especially} when a space is occupied, is difficult.  Typically it involves deep knowledge of the usage scenario and
a meaningful understanding of what it takes to support the activity.  Furthermore, situations and activities differ
greatly.  The outside weather changes, varying schedules affect occupancy, rooms have lectures, class,
or other office activities.  Simply put, the process is time consuming, requires specialized knowledge,
and does not scale.

Devices are typically used together in some fashion.  For example, in an office
setting a person enters their office, turns on their PC and lights, etc.
When the person leaves the office, they revert back to the state their devices were in before arrival.
If one of the items is not reverted to its pre-arrival state, waste occurs. 
%Waste occurs when something is left on.
The same is true about equipment usage.  When the outside temperature is low the heater turns on.
% and
%the negation is also true.  If the temperature is high and the heater is on, waste occurs.  
\emph{Waste occurs when abnormal in-concert usage patterns arise}.  
%For example, 
% Moreover, if the heating and cooling system are on 
% simultaneously~\cite{simheatcool}, that is a problem that is \emph{particularly} wasteful and hard to 
% detect by occupants.  
Fundamentally, understanding ``normal'' spatio-temporal usage patterns between devices could help
identify problems when devices are not being used correctly.
We conjecture that inefficient energy use can be identified through anomalies in the correlation
patterns between devices.  We examine device correlation patterns in this paper and look specifically
at processing raw sensor traces, such that the correlations we find are meaningful.

In this paper, we present early results for correlating usage patterns across a large number of sensors
in a single deployment.  We analyze data from a 12-story office building at the University of Tokyo.  
The deployment consists of almost 700 sensors monitoring a broad range of devices inside and outside 
the building.  Our initial observations and results include the following:

\begin{enumerate}
\item Raw-trace correlation analysis is too strongly influenced by the common low-frequency trends in the data
	to identify meaningful relationships.
\item Using a technique called empirical model decomposition (EMD)~\cite{huang:emd1998} removes this 
		 trend and helps identify truly correlated sensor traces.
\item We can construct clusters of correlated sensors that are spatio-temporally correlated, \emph{without
		a priori knowledge of their placement}.
\end{enumerate}

In the rest of the paper we explain EMD and how we use it, we show various examples of our technique on real-world
traces, and we discuss the implications and future work.

% Green IT

% Understand the energy consumption of a building and identify savings opportunities.

% Identification of energy consuming devices that are correlated.
% Uncover usage patterns of correlated device that are energy efficient.
% Detect deviation from the energy efficient pattern and report to the user.

% During the design of our application the first difficulty was to identify the set of devices that have related energy consumption.

% This article focuses on this problem.

% Results:
% \begin{enumerate}
% \item Correlation is noisy and can't find inter-relationships between sensors
% 		with subtle differences.
% \item Underlying behavior should extract most-common denominator in comparing traces
% 		to observe truly correlated behavior.
% \item Empirical mode decomposition (EMD) can be used to compare underlying behavior after the
% 		removal of the dominant frequencies in the signal.
% \end{enumerate}

% \subsection{ideas}

% Future work:
% \begin{enumerate}
% \item We can create a time-varying dependency graph to compare ``normal'' versus ``abnormal'' behavioral
% 		patterns in underlying use.
% \item We can codify ``normal'' or ``efficient'' graphs and compare with real graph constructs over time.
% \end{enumerate}

% Possible algorithms:
% \begin{enumerate}
% \item find correlated and uncorrelated sensors
% \item construct correlation network where the nodes are the sensors and an edge implies correlation above
% 		threshold. (We can also construct the complement of that.)
% \end{enumerate}



\section{Metadata}
In sensor deployments, metadata is as important as the data being collected. Without the metadata, the data is effectively useless. 
Therefore it is very important to not only track changes in the metadata, but take them into account when the user is querying 
the data. The metadata must be a first-class citizen in the system.

Like the timeseries data collected from the sensors, metadata should also be maintained as a timeseries and 
used in conjunction with timeseries queries performed on the data. In querying deployment data, the user has two 
options: 1) she queries for some information from the logical context, which may include a changing set of sensor 
items over the time interval of interest or 2) she queries a specific sensor item, whose logical context may change 
during the interval of the query. In either case if the change is not noted and accounted for, the data can be return 
incorrect results.

Lets consider a concrete example where temperature sensors are deployed in a room. For the sake of demon- stration, lets 
assume that the temperature in that room is being taken by a mobile temperature sensor on cell phones of the occupants. For 
now we can assume that
context is accurately maintained as occupants with this application enter and leave the room. If the user queries for the average temperature in that room throughout the day the room for all temperature sensor readings taken and take the average. It is important to only consider the temperature readings taken when the cell phone was inside that room. The user, however, should not be con- cerned with how this accounted for, it should occur au- tomatically at query runtime.

Using the same scenario, lets imagine that the user wishes to know the average temperature reading taken by her cell phone throughout the day. In order to assure that the data is correctly interpreted, the query should return several averages; one for each setting as well as the aggregate average. Moreover, if the sensor was re- moved from the system before the upper-bound of the time interval constraint, the user should be alerted and the missing readings should not be counted in the aver- age.
These mechanisms should exist to assure the integrity of the interpretation of the data. Misinterpretation of the data can lead to gross discrepancies in the conclusions that are drawn from the data analysis and in the context of sensor deployments, where the sensor context is not always static, the analysis is even more prone to error without such treatment of the metadata.

%\date{14 April 2007}
%\maketitle


% A category with the (minimum) three required fields
%category{B.0}{Hardware}{General}

%\newpage
% \subsection{Introduction}
Buildings consume an enormous amount of energy in countries around the world.  In 
Japan, 28\% of the energy produced is consumed in buildings~\cite{japanbuildings} while in the United 
States it is as high as 40\%~\cite{epabuildings}.  Moreover, studies show that between 30-80\% of it
is wasted~\cite{waste_science, next10_waste}.  Large commercial buildings are typically instrumented
with a large number of sensors measuring various aspects of building operation.  Although this data is
typically used to assure operational stability, they may also be used to measure, observe, and identify
instances of wasted use.

Identifying instances of wasted energy use is non-trivial.  System efficiency is defined as the ratio of the 
useful work done to the energy it consumes.  In the case of buildings, we broadly define useful work as 
the energy used to support occupant activities.  From the perspective of the building that means maintaining
a comfortable temperature setting, providing power for plug-load devices, and providing adequate lighting
conditions; particularly in spaces that are occupied.  However, identifying efficient use of resources,
\emph{especially} when a space is occupied, is difficult.  Typically it involves deep knowledge of the usage scenario and
a meaningful understanding of what it takes to support the activity.  Furthermore, situations and activities differ
greatly.  The outside weather changes, varying schedules affect occupancy, rooms have lectures, class,
or other office activities.  Simply put, the process is time consuming, requires specialized knowledge,
and does not scale.

Devices are typically used together in some fashion.  For example, in an office
setting a person enters their office, turns on their PC and lights, etc.
When the person leaves the office, they revert back to the state their devices were in before arrival.
If one of the items is not reverted to its pre-arrival state, waste occurs. 
%Waste occurs when something is left on.
The same is true about equipment usage.  When the outside temperature is low the heater turns on.
% and
%the negation is also true.  If the temperature is high and the heater is on, waste occurs.  
\emph{Waste occurs when abnormal in-concert usage patterns arise}.  
%For example, 
% Moreover, if the heating and cooling system are on 
% simultaneously~\cite{simheatcool}, that is a problem that is \emph{particularly} wasteful and hard to 
% detect by occupants.  
Fundamentally, understanding ``normal'' spatio-temporal usage patterns between devices could help
identify problems when devices are not being used correctly.
We conjecture that inefficient energy use can be identified through anomalies in the correlation
patterns between devices.  We examine device correlation patterns in this paper and look specifically
at processing raw sensor traces, such that the correlations we find are meaningful.

In this paper, we present early results for correlating usage patterns across a large number of sensors
in a single deployment.  We analyze data from a 12-story office building at the University of Tokyo.  
The deployment consists of almost 700 sensors monitoring a broad range of devices inside and outside 
the building.  Our initial observations and results include the following:

\begin{enumerate}
\item Raw-trace correlation analysis is too strongly influenced by the common low-frequency trends in the data
	to identify meaningful relationships.
\item Using a technique called empirical model decomposition (EMD)~\cite{huang:emd1998} removes this 
		 trend and helps identify truly correlated sensor traces.
\item We can construct clusters of correlated sensors that are spatio-temporally correlated, \emph{without
		a priori knowledge of their placement}.
\end{enumerate}

In the rest of the paper we explain EMD and how we use it, we show various examples of our technique on real-world
traces, and we discuss the implications and future work.

% Green IT

% Understand the energy consumption of a building and identify savings opportunities.

% Identification of energy consuming devices that are correlated.
% Uncover usage patterns of correlated device that are energy efficient.
% Detect deviation from the energy efficient pattern and report to the user.

% During the design of our application the first difficulty was to identify the set of devices that have related energy consumption.

% This article focuses on this problem.

% Results:
% \begin{enumerate}
% \item Correlation is noisy and can't find inter-relationships between sensors
% 		with subtle differences.
% \item Underlying behavior should extract most-common denominator in comparing traces
% 		to observe truly correlated behavior.
% \item Empirical mode decomposition (EMD) can be used to compare underlying behavior after the
% 		removal of the dominant frequencies in the signal.
% \end{enumerate}

% \subsection{ideas}

% Future work:
% \begin{enumerate}
% \item We can create a time-varying dependency graph to compare ``normal'' versus ``abnormal'' behavioral
% 		patterns in underlying use.
% \item We can codify ``normal'' or ``efficient'' graphs and compare with real graph constructs over time.
% \end{enumerate}

% Possible algorithms:
% \begin{enumerate}
% \item find correlated and uncorrelated sensors
% \item construct correlation network where the nodes are the sensors and an edge implies correlation above
% 		threshold. (We can also construct the complement of that.)
% \end{enumerate}



%
% The marriage of pub/sub, streaming dbms, and filesystems
%

\vspace{+0.5mm}
\vspace{+2mm}
\bibliographystyle{abbrv}
\small
\bibliography{references}

\end{document}


