%\documentclass[10pt,conference]{IEEEtran}
%\documentclass[9pt,conference]{sig-alternate}
%\documentclass[9pt,conference]{sig-alternate}
%\documentclass[10pt,print,letterpaper,nocopyrightspace]{sigplan-proc-varsize}
%\documentclass[10pt,print,letterpaper]{sigplan-proc-varsize}
\documentclass[10pt,print,letterpaper]{sensys-proc}

\usepackage{amsmath,epsfig}
\usepackage{url}
\usepackage{xspace}
\usepackage{colortbl}
\usepackage{subfigure}
\usepackage{dsfont}
\usepackage{boxedminipage}
% \ifx\pdfoutput\undefined
% \usepackage[hypertex]{hyperref}
% \else
% \usepackage[pdftex,hypertexnames=false]{hyperref}
% \fi

\usepackage{amssymb}
\usepackage{wasysym}
\usepackage[left=2.54cm,top=2.54cm,right=2.54cm,bottom=2.54cm,nohead,nofoot]{geometry}
\usepackage{algorithm}
\usepackage{algpseudocode}
\usepackage{listings} 


\DeclareMathOperator*{\argmax}{argmax}
%\usepackage{times}

\def\ucb{$^{\dagger}$}
\def\stanford{$^{\ddagger}$}
\def\arch{$^{\star}$}

\newcommand{\kb}{kB}
\newcommand{\rene}{Ren{\'e}\xspace}
\newcommand{\reneii}{Ren{\'e}2\xspace}
\newcommand{\wec}{WeC\xspace}
\newcommand{\mica}{Mica\xspace}
\newcommand{\micaii}{Mica2\xspace}
\newcommand{\micaz}{MicaZ\xspace}
\newcommand{\micadot}{Mica2Dot\xspace}
\newcommand{\iic}{I$^2$C\xspace}
\newcommand{\uA}{$\mu$A\xspace}
\newcommand{\dotmote}{Dot\xspace}
\newcommand{\mhz}{MHz\xspace}
\newcommand{\ghz}{GHz\xspace}
\newcommand{\kbps}{kbps\xspace}
\newcommand{\dsn}{DSN\xspace}
\newcommand{\io}{I/O\xspace}
\newcommand{\telos}{Telos\xspace}

\newcommand{\T}{\mathds{T}}
\newcommand{\XXXnote}[1]{{\bf\color{red} XXX: #1}}


\begin{document}

\crdata{978-1-4503-1170-0}
\conferenceinfo{Buildsys'12,} {November 6, 2012, Toronto, ON, Canada.}
\CopyrightYear{2012}

\title{StreamFS: Metadata A First Class Citizen}
%\title{Systems challenges in real-time, fined-grained energy analytics through mobile phones}
%\title{Mobility and consistency management in real-time, fined-grained energy analytics for buildings through with mobile phones}
\numberofauthors{1} 
\author{\alignauthor Jorge Ortiz and David Culler\\
\affaddr{Computer Science Division}\\
\affaddr{University of California, Berkeley} \\ 
%\affaddr{Berkeley, California 94720} \\
\email{\{jortiz,david.su,culler\}@cs.berkeley.edu}
} 


%\subtitle{Paper \# Insert Reg Number Here}

%\title{Alternate {\ttlit ACM} SIG Proceedings Paper in LaTeX
%Format\titlenote{(Produces the permission block, and
%copyright information). For use with
%SIG-ALTERNATE.CLS. Supported by ACM.}}
%\subtitle{[Extended Abstract]
%\titlenote{A full version of this paper is available as
%\textit{Author's Guide to Preparing ACM SIG Proceedings Using
%\LaTeX$2_\epsilon$\ and BibTeX} at
%\texttt{www.acm.org/eaddress.htm}}}
%
% You need the command \numberofauthors to handle the 'placement
% and alignment' of the authors beneath the title.
%
% For aesthetic reasons, we recommend 'three authors at a time'
% i.e. three 'name/affiliation blocks' be placed beneath the title.
%
% NOTE: You are NOT restricted in how many 'rows' of
% "name/affiliations" may appear. We just ask that you restrict
% the number of 'columns' to three.
%
% Because of the available 'opening page real-estate'
% we ask you to refrain from putting more than six authors
% (two rows with three columns) beneath the article title.
% More than six makes the first-page appear very cluttered indeed.
%
% Use the \alignauthor commands to handle the names
% and affiliations for an 'aesthetic maximum' of six authors.
% Add names, affiliations, addresses for
% the seventh etc. author(s) as the argument for the
% \additionalauthors command.
% These 'additional authors' will be output/set for you
% without further effort on your part as the last section in
% the body of your article BEFORE References or any Appendices.

% \numberofauthors{2} %  in this sample file, there are a *total*
% % of EIGHT authors. SIX appear on the 'first-page' (for formatting
% % reasons) and the remaining two appear in the \additionalauthors section.
% %
% \author{
% % You can go ahead and credit any number of authors here,
% % e.g. one 'row of three' or two rows (consisting of one row of three
% % and a second row of one, two or three).
% %
% % The command \alignauthor (no curly braces needed) should
% % precede each author name, affiliation/snail-mail address and
% % e-mail address. Additionally, tag each line of
% % affiliation/address with \affaddr, and tag the
% % e-mail address with \email.
% %
% % 1st. author
% \alignauthor
% %Prabal Dutta\\
% %       \affaddr{Computer Science Division}\\
% %       \affaddr{Univ. of California, Berkeley}\\
% %       \affaddr{Berkeley, CA 94720}\\
% %       \email{prabal@cs.berkeley.edu}
% % 2nd. author
% %\alignauthor
% %David Culler\\
% %       \affaddr{Computer Science Division}\\
% %       \affaddr{Univ. of California, Berkeley}\\
% %       \affaddr{Berkeley, CA 94720}\\
% %       \email{culler@cs.berkeley.edu}
% % 3rd. author
% %\alignauthor
% %Scott Shenker\\
% %       \affaddr{Computer Science Division}\\
% %       \affaddr{Univ. of California, Berkeley}\\
% %       \affaddr{Berkeley, CA 94720}\\
% %       \email{shenker@cs.berkeley.edu}
% %}
% %%%Jorge Ortiz, Yongwoo Noh, Gavin Saldanha, David Su, Jason Trager, David Culler, and Paul Wright\\
%        %\affaddr{Department}\\
% 	%%%\affaddr{Computer Science Division}\\
%        %%%\affaddr{University of California, Berkeley}\\
%        %\affaddr{City, State Zip}\\
%        %%%\email{jortiz@cs.berkeley.edu}
% }


\maketitle


% \begin{abstract}
% In this paper we present a system called the Energy Lens -- a system that provides  
% deeper, real-time visibility of plug-load energy consumption in buildings.  Our initial work focuses
% on plug-load power metering, display, and real-time aggregation, presented to the user through
% a mobile phone.  We discuss the three main, non-trivial challenges that must be addressed to provide
% real-time energy analytics in buildings through mobile phones and our initial approach towards addressing 
% each challenge.
% \end{abstract}

% \category{H.4}{Information Systems Applications}{Miscellaneous}
% %A category including the fourth, optional field follows...
% %\category{D.2.8}{Software Engineering}{Metrics}[complexity measures, performance measures]
% \terms{Distributed Consistency, Mobile application}
% \keywords{Disconnected operations, network access, energy, power, plug loads}

\subsection{Introduction}
The United States leads the world in per-capita energy consumption.
Our electricity use has consistently increased over the last 40 years~\cite{oecd2011} and other parts of the world are rising all 
too rapidly.  With the specter of climate change and the increasing cost of energy, we must explore new
ways for individuals to gain visibility and insight into their energy consumption in order to optimize and reduce it. 
With the increasing penetration of embedded sensors in the environment and
the continued rise in smartphone adoption, we see an opportunity for smartphones to bridge the physical world
to our computational infrastructure and provide an `energy lens' on the physical world.  

We use mobile phones to construct an entity-relationship 
graph of the physical world and combine it with streaming sensor data in order to perform detailed energy-attribution.
We limit the scope of the world to a single building domain.  We have designed and implemented a real-time, mobile energy auditing
application, called the `Energy Lens', that allows us to collect information about 
things throughout the building and how they are related to each other.  For example, computer X is inside 
room Y and connected to meter Z.  Then, we use these relationships to guide our data look-up and analytical
calculations.  For example, the load curve of room Y consists of the sum of all the power traces for loads
inside room Y.  We use the mobile smartphone as the main input tool.  Our work examines \emph{three main challenges} in setting up and 
deploying a real, whole-building infrastructure to support real-time, 
fined grained energy analytics.  

The first challenge is related to tracking and mobility.
The use of mobile phones presents classical, fundamental challenges related to mobility.  Typically, mobility
refers to the phone, as the person carrying it moves from place to place.  However, in the energy-attribution
context, we are also referring to the movement of energy-consuming objects.  Tracking their relationships to spaces 
and people is as important as tracking people.  We describe how we deal with \emph{both moving people and 
moving objects} and show that these historically difficult problems can be addressed relatively easily, if the proper infrastructure is 
in place.  %We provide evidence that the approach is simple, incrementally deployable, and scalable.

The second challenge is about capturing the inter-relationship semantics and having these inform our  analytics.
We adopt the general notion of physical tags that identify objects in the world.  Our system uses \emph{QR codes} to tag things and locations 
in the physical world.  However, \emph{any tag that provides a unqiue identifier for an object could serve the same purpose}.
Once tagged, there are three types of interactions -- 
registration, linking, and scanning -- which establish important relationships.  Registration is the act of creating a virtual object 
to represent a physical one.  Linking captures the relationship between pairs of objects.  Scanning is the act of performing an item-lookup.
Each of these interactions requires a set of swiping gestures.  Linking requires two tag swipes while the other two actions
require a single tag swipe.  Internally, we maintain a \emph{entity-relationship graph (ERG)} of things, people, and locations, that gets
updated through these sets of gestures.

The third challenge is about indoor network connectivity and access.
In order to connect these components, we rely on having `ubiquitous' network connectivity.  However, in practice, network
\emph{availability} is intermittent and our system must deal with the challenges of intermittency.  We discuss how caching
and logging are used to address these challenges.  Moreover, when connectivity is re-established, we must deal with
applying updates to the ERG, as captured by the phone while disconnected.  
% Conflicts can also occur during an update.  For example, the two updates may disagree about which items are attached
% to which meters.  We implement a very simple conflict resolution scheme, described in section~\ref{sec:conflicts}.
% Finally, certain physical-state transitions are represented as a set of updates to the ERG that must be applied 
% atomically.  We implement transactions in the log-replay and transaction manager.
% Our `Energy Lens' system is deployed in a building on our campus.  We discuss
% its architecture and our design choices.  
  
% We also discuss novel strategies for tracking moving people/things and describe how we implement these in our system.  In summary, our work
% makes the following contributions:

% \begin{itemize}
% \item We design and implement a system that captures and combines physical entities, their inter-relationships, and real-time sensor data 
% 		in buildings.% using mobile phones, qr code, and a cloud-based infrastructure.
% \item We observe that certain combinations of swipes give us useful information to set the location of people and things over time.
% 		We codify this observation in our \emph{context-tracker} and use it to maintain consistency between the entity-relationship graph and the 
% 		state of the physical world.  To the best of our knowledge, this is radically different from the approaches in standard 
% 		localization techniques.  However, we argue that it can be used to \emph{enhance} their accuracy and overall performance.
% \item We implement a prefetching algorithm to obtain context-dependent information to both improve performance and
% 		enable disconnected operation.  We also design and implement a log-replay and transaction manager over our data management layer.  We describe how different conflict-resolution policies can be implemented and our rationale for the policies we chose.
% \end{itemize}

% \vspace{0.08in}

% In the next sections we go through a motivating scenario.  We then discuss some related work, followed 
% by the system architecture, evaluation, and future directions.


\section{Metadata}
In sensor deployments, metadata is as important as the data being collected. Without the metadata, the data is effectively useless. 
Therefore it is very important to not only track changes in the metadata, but take them into account when the user is querying 
the data. The metadata must be a first-class citizen in the system.

Like the timeseries data collected from the sensors, metadata should also be maintained as a timeseries and 
used in conjunction with timeseries queries performed on the data. In querying deployment data, the user has two 
options: 1) she queries for some information from the logical context, which may include a changing set of sensor 
items over the time interval of interest or 2) she queries a specific sensor item, whose logical context may change 
during the interval of the query. In either case if the change is not noted and accounted for, the data can be return 
incorrect results.

Lets consider a concrete example where temperature sensors are deployed in a room. For the sake of demon- stration, lets 
assume that the temperature in that room is being taken by a mobile temperature sensor on cell phones of the occupants. For 
now we can assume that
context is accurately maintained as occupants with this application enter and leave the room. If the user queries for the average temperature in that room throughout the day the room for all temperature sensor readings taken and take the average. It is important to only consider the temperature readings taken when the cell phone was inside that room. The user, however, should not be con- cerned with how this accounted for, it should occur au- tomatically at query runtime.

Using the same scenario, lets imagine that the user wishes to know the average temperature reading taken by her cell phone throughout the day. In order to assure that the data is correctly interpreted, the query should return several averages; one for each setting as well as the aggregate average. Moreover, if the sensor was re- moved from the system before the upper-bound of the time interval constraint, the user should be alerted and the missing readings should not be counted in the aver- age.
These mechanisms should exist to assure the integrity of the interpretation of the data. Misinterpretation of the data can lead to gross discrepancies in the conclusions that are drawn from the data analysis and in the context of sensor deployments, where the sensor context is not always static, the analysis is even more prone to error without such treatment of the metadata.

%\date{14 April 2007}
%\maketitle


% A category with the (minimum) three required fields
%category{B.0}{Hardware}{General}

%\newpage
% \subsection{Introduction}
The United States leads the world in per-capita energy consumption.
Our electricity use has consistently increased over the last 40 years~\cite{oecd2011} and other parts of the world are rising all 
too rapidly.  With the specter of climate change and the increasing cost of energy, we must explore new
ways for individuals to gain visibility and insight into their energy consumption in order to optimize and reduce it. 
With the increasing penetration of embedded sensors in the environment and
the continued rise in smartphone adoption, we see an opportunity for smartphones to bridge the physical world
to our computational infrastructure and provide an `energy lens' on the physical world.  

We use mobile phones to construct an entity-relationship 
graph of the physical world and combine it with streaming sensor data in order to perform detailed energy-attribution.
We limit the scope of the world to a single building domain.  We have designed and implemented a real-time, mobile energy auditing
application, called the `Energy Lens', that allows us to collect information about 
things throughout the building and how they are related to each other.  For example, computer X is inside 
room Y and connected to meter Z.  Then, we use these relationships to guide our data look-up and analytical
calculations.  For example, the load curve of room Y consists of the sum of all the power traces for loads
inside room Y.  We use the mobile smartphone as the main input tool.  Our work examines \emph{three main challenges} in setting up and 
deploying a real, whole-building infrastructure to support real-time, 
fined grained energy analytics.  

The first challenge is related to tracking and mobility.
The use of mobile phones presents classical, fundamental challenges related to mobility.  Typically, mobility
refers to the phone, as the person carrying it moves from place to place.  However, in the energy-attribution
context, we are also referring to the movement of energy-consuming objects.  Tracking their relationships to spaces 
and people is as important as tracking people.  We describe how we deal with \emph{both moving people and 
moving objects} and show that these historically difficult problems can be addressed relatively easily, if the proper infrastructure is 
in place.  %We provide evidence that the approach is simple, incrementally deployable, and scalable.

The second challenge is about capturing the inter-relationship semantics and having these inform our  analytics.
We adopt the general notion of physical tags that identify objects in the world.  Our system uses \emph{QR codes} to tag things and locations 
in the physical world.  However, \emph{any tag that provides a unqiue identifier for an object could serve the same purpose}.
Once tagged, there are three types of interactions -- 
registration, linking, and scanning -- which establish important relationships.  Registration is the act of creating a virtual object 
to represent a physical one.  Linking captures the relationship between pairs of objects.  Scanning is the act of performing an item-lookup.
Each of these interactions requires a set of swiping gestures.  Linking requires two tag swipes while the other two actions
require a single tag swipe.  Internally, we maintain a \emph{entity-relationship graph (ERG)} of things, people, and locations, that gets
updated through these sets of gestures.

The third challenge is about indoor network connectivity and access.
In order to connect these components, we rely on having `ubiquitous' network connectivity.  However, in practice, network
\emph{availability} is intermittent and our system must deal with the challenges of intermittency.  We discuss how caching
and logging are used to address these challenges.  Moreover, when connectivity is re-established, we must deal with
applying updates to the ERG, as captured by the phone while disconnected.  
% Conflicts can also occur during an update.  For example, the two updates may disagree about which items are attached
% to which meters.  We implement a very simple conflict resolution scheme, described in section~\ref{sec:conflicts}.
% Finally, certain physical-state transitions are represented as a set of updates to the ERG that must be applied 
% atomically.  We implement transactions in the log-replay and transaction manager.
% Our `Energy Lens' system is deployed in a building on our campus.  We discuss
% its architecture and our design choices.  
  
% We also discuss novel strategies for tracking moving people/things and describe how we implement these in our system.  In summary, our work
% makes the following contributions:

% \begin{itemize}
% \item We design and implement a system that captures and combines physical entities, their inter-relationships, and real-time sensor data 
% 		in buildings.% using mobile phones, qr code, and a cloud-based infrastructure.
% \item We observe that certain combinations of swipes give us useful information to set the location of people and things over time.
% 		We codify this observation in our \emph{context-tracker} and use it to maintain consistency between the entity-relationship graph and the 
% 		state of the physical world.  To the best of our knowledge, this is radically different from the approaches in standard 
% 		localization techniques.  However, we argue that it can be used to \emph{enhance} their accuracy and overall performance.
% \item We implement a prefetching algorithm to obtain context-dependent information to both improve performance and
% 		enable disconnected operation.  We also design and implement a log-replay and transaction manager over our data management layer.  We describe how different conflict-resolution policies can be implemented and our rationale for the policies we chose.
% \end{itemize}

% \vspace{0.08in}

% In the next sections we go through a motivating scenario.  We then discuss some related work, followed 
% by the system architecture, evaluation, and future directions.


%
% The marriage of pub/sub, streaming dbms, and filesystems
%

\vspace{+0.5mm}
\vspace{+2mm}
\bibliographystyle{abbrv}
\small
\bibliography{references}

\end{document}


