\section{Motivating scenario}
\label{sec:vision}
% Several things to mention:

% \begin{itemize}
% \item physical data services
% \item energy analytics
% \item we need to know where things are, people are, how they're inter-related
% \item integration of all building energy feeds into one system
% \end{itemize}

Imagine having the ability to walk through a building and see live, detailed energy data as
you point your phone at various things and locations.  As you enter the building, you scan its tag and see
the live breakdown of energy consumption traces, including HVAC, lighting, and plug-loads.  You continue
your walk through the building as you head to your office.  When you arrive to your floor
you scan the tag for the floor and observe similar figures, only this time they are in relation to that floor
alone.  Since there are several meeting rooms on that floor, you are curious how much is consumed by 
occupants versus visitors.  You choose to view the total load curve co-plotted with the occupant
load curve, specifically for that floor.  You see that approximately half the total energy is consumed
by visitors during the day.  

Curious about what portion of total are attributed to you, you select the 
personalized attribution option and you see \emph{your} personal load curve plotted
with the total load curve -- as well as accompanying statistics, such as the percent of total over time.
As you quickly examine the data on your phone, you see that you consumed energy during hours that you were not
there.  You choose to see a more detailed breakdown.  You enter your office, scan various items that you own, and see that
your computer did not shut down properly and your light switch was set to manual.  You immediately 
correct these.

Being able to interact with your environment and get a complete energy break-down can provide a useful tool
for tracing and correcting rampant energy consumption.  In buildings, having the occupants actively participate
allows for localized, personal solutions to efficiency management and is crucial to scaling to large buildings.
However, providing this detailed level of attribution is challenging.  There's lots of data coming from various systems 
in the building, and integrating them in real time is difficult.  Furthermore, attribution is non-trivial.  
We must be able to answer to following: How much of the total consumed on this floor went to charging laptops?  How
many of those charging laptops belong to registered occupants of this floor?
For centralized systems, multiple locations are served simultaneously.  It is non-trivial to determine the exact
break-down for each location.  At the plug-load level, some plug loads move from place to place throughout the building
over the course of the day.  Tracking where they are at any given time is difficult.

Answering these queries is relatively easy once the information is available, however, collecting the information
is non-trivial, especially over time.  Historically, it has been difficult to collect plug-load information.
Various studies have used wireless power meters to accomplish just this~\cite{stephscale, lanz, aceee}.
All previous work collected the data and performed post-processing to analyze it.  We want to take the next
natural set of steps: perform processing in real-time and present the occupants with live information.  

There are several systems challenges that must be overcome in order to achieve this vision.
%One of the 
% We need to place live metering on plug loads.  We need to integrate data from the BMS with plug load and other
% meter data.  We need to be able to approximate the attribution algorithms and tailor them for real-time processing.
We examine those dealing capturing and maintaining a view of the physical world.  We must record which things belong to whom, 
where people
and things are over time, how to deal with the mobile phone as the main interactive modality, and how to do this at the scale of 
hundreds to thousands of users and meters.  We argue that \emph{without addressing these systems issues, this
vision cannot be achieved}.  In our work, we start by deploying a network of wireless power meters and use the mobile
phone to re-create a model of people, things, and locations in the building.  We also use it to assist in tracking of 
people and things over time.

