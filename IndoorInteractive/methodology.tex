

\section{Traced-driven Analysis}
For our experiments, we ran a trace-driven analysis using traces from a mobile energy application conducted
in a campus building.  It includes the registration of new devices, meters, and spaces throughout the building
as well as scans for information about those devices, from occupants in the building that want to learn more
about the power-draw history of the devices.  The audit was conducted by multiple students simultaneously, each
walking around the different floor in the building, registering new devices and inputting information about
each device.  Because the traces do not include connectivity information, we 
take the indoor connectivity traces and simulate the effect of those conditions on the application traces.

We examine the consistency, availability and relative energy consumption of the application with and without
the use of CAL.

We measure the following:

\begin{itemize}
\item Percentage of time the application is unavailable.
\item The total amount of energy consumed by the either application. 
	\begin{itemize}
	\item something
	\end{itemize}
\item Graph of object id and absolutel access time (by offset from first access).
\item Graph of the trace based on object id and color each cell according to the closeness in time for the weight
		absolute difference in object access times.
\end{itemize}

\subsection{Energy Lens Application Trace-Based Simulation}

We collected several months of logs from a mobile energy audit application summarized here:

\begin{enumerate}
\item How many unique user scans?
\item How many items were scanned in sequence per user?
\item How many moves?
\item How many removals?
\item How many updates?
\end{enumerate}

We extracted the portion of the trace during which an audit took place.  With the original version of the application
an audit event must be aborted if the connection to the network becomes unavailable.  We modified the application using CAL,
allowing auditors to make local updates to the state of the audit and update the server when the connection becomes 
available.  The portion of the audit trace that we use took place on a single floor.  For energy audits in larger, older buildings
in locations with poor coverage, we expect that battery consumption will become an issue.  Moreover, auditors
 typically audit more than one building while on assignment~\cite{somethingHere}.

We used these as the basis for a simulation of the application with multiple simultaneous users.  We look to observe the 
effects of increased contention for physical interaction and virtual updates of the item
observe how CAL helps manage that contention along the aforementioned tradeoff space.  With connection-quality log traces
to simulated the effects of varying connectivity.

\begin{enumerate}
\item Items scans in a room.
\item Item tag updates.
\item Item movement.
\item Item removal.
\item Probability of going to another room.
\item Probability of going to another floor.
\end{enumerate}

Essentially we're simulating a many, simulatenous of readers and writers.

\subsection{Newsreader trace}
To demonstrate how CAL performs for other application, we construct a newsfeed trace.  The newsfeeds trace is constructed in a
as described by Higgins et al.~\cite{imp_mobisys2012}.  Google Reader provides statistics on the last 30
days of feeds with the average.

\emph{Take the most popular feeds, find the median article size, assume each article is read from a uniform distribution between
30 and 60 seconds per article and then fetch the next one.  We compare the response time, energy consumption, and availability.
We simulate the loss of internet connectivity and energy level?  Better consistency and availability (leverage the opportunity 
to prefetch an article.)  What about energy?  We're not necessarily minimizing energy consumption, we're managing the energy budget.}

\emph{We want to demonstrate how the application can choose between the 3 axes by changing the calls in the API.  }

\emph{We also want to show that the prefetching strategy works well for a certain class of applications -- the ones driven by physical-object-based
scanning of the environment.}