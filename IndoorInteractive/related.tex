\section{Related Work}
Bayou~\cite{bayou} is a system architecture that is designed to provide high-availability,
and variable consistency while supporting application-specifc conflict resolution
and application-controlled inconsistency management.  The system was specifically
designed for mobile clients with variable connection quality and periodic disconnection.
In many ways, the design goals of Bayou are very similar to ours.  We both aim to provide highly 
available read and write operations, we both rely on an eventual consistency model, both support
application-level detection of conflicts, application-specific conflict resolution, permitting
disconnected clients to see their own updates.  In contrast, we do not explicitly provide session
guarantees.  We assume all clients are writing to the same application servers and that updates to the server
by any client is consistency across a cluster of application servers.

Our system, however, introduces the notion of variable consistency and availability with respect
to energy consumption.  Typically applications are willing to spend more energy per I/O operation if 
connectivity is available and the cost/bit is cheap.  Moreover, we include a prefetcher in our framework that
is adjusts it prefetch download size based on time-based hierarchical clustering of object reference time, dictated
largely by the proximity of physical objects to the user interacting with the application.  We focus on
mobile applications that are interacting with objects in the physical world, not necessarily a virtual
object on a server being edited by a random set of clients.  Most reads on the virtual object
will come from clients near that object in physical space, thereby limiting the number of concurrent
interactions and the prefetch size to the space physical space occuppied by those client's primary users.
Bayou is a database system for mobile clients, we are mainly client-side framework with optional server-side
support to stronger consistency and ordering guarantees.


PocketWeb~\cite{pocketweb} is an prefetch framework for mobile phones that prefetching dynamic web content.
PocketWeb uses a machine learning approach based on stochastic gradient boosting techniques to model 
the mobile web browsing patterns of mobile users.  The main observation that allows their system to effectively
prefetch content is that there are user-specific spatio-temporal access patterns.  Their technique builds a model
per user and prefetches 80\%-90\% of the content for 60\% of users 2 minutes before they access it.  Proactive, periodic
prefetching is necessary dynamic web content, in particular.  By limiting our framework to the class of indoor, interactive
mobile application -- where the user is directly interacting with the space around her through her phone -- 
we limit the scope and access pattern to the spatial proximity to the world around them.
With the increase in embedded sensing, we will see more applications that fall into this class, and prefetching as many virtual
objects near the user as possible, will become more important.  We also allow the application designer
to control the frequency and scope of the prefetch.  Fetching larger sets more frequency when connectivity is good and
energy level is high.


