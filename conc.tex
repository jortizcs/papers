\subsection{Limitations}
EMD is useful for finding underlying behavioral relationships between traces of sensor data.  However,
when we set the timescales smaller than a day, the results were not as strong.
The trace has to be long enough to capture the trend.  For this data set, the underlying
trend is daily, therefore it requires there to be a significant number of samples over many days.
%  to
% for this method to be effective.
Although this was a limitation for this dataset, it really depends on the underlying phenomenon that
the sensors are measuring.  Its underlying trend is ultimately what EMD will be able to separate
from the intrinsic modes of the underlying signal.

\subsection{Discussion}
EMD allows us to effectively identify fundamental relationships between sensor traces.
% Using EMD to find fundamental relationships between sensor traces effectively identifies intrinsically
% related behavioral relationships.
% was quite promising for building
% up models of correlated usage.  
We believe that identifying meaningful usage-correlation patterns can help reduce oversights
by the occupants and faults that lead to energy waste.  A direct application of this is the identification
of simultaneous heating and cooling~\cite{simheatcool}.  Simultaneous heating and cooling is when heating
and cooling system either compete with one another or compete with the incoming air from outside and is
a cause for major energy waste in building~\cite{simheatcool}.  The longer it goes undetected,
the more waste there is.  However, it is very difficult to identify since the occupants do not notice
changes in temperature.  Our analysis can be used to build a correlation model between the outside
heating cooling temperature, the cooling coil temperature, and the outside air vent position.  If their behavior
is not correlated as expect, an alarm should be raised.

We can also apply it to other usage scenarios.  In our traces, for example, we found an instance where the pump
was on, but the lights were off.  The two are typically correlated, however, in this case they were not.
The air conditioning was pumping cool air into a room without occupants.
With our approach this could have been identified and corrected.  In future work, we intend to
package our solution to serve these kinds of applications.

% EMD works perfectly although the weekly pattern of the data is altered the last analyzed week.

% EMD inherently finds the interesting time scales.

% number of zero crossing = mean frequency/time scale:
%   TODO give the time scale for each IMF.


\section{Conclusion}
% what is our problem

% what we did

% contributions and main results

% future work


This paper set out to examine the underlying relationship between sensor traces to find interesting correlations
in use.  We used data from a large deployment of sensors in a building and found that direct correlation analysis on the raw
traces was not discriminatory enough to find interesting relationships.  Upon closer inspection, we noticed that
the underlying trend was dominating the correlation calculation.  In order to extract meaningful behavior this trend has
to be removed.  We concluded that empirical mode decomposition is a helpful analytical tool for de-trending 
non-linear, non-stationary data; inherent attributes contained by our traces.

We re-ran our analysis on the same traces, except we correlated the IMF outputs of the EMD operation on each of the traces and found that the pump was closely related to the lights in the same room served by the pump and
uncorrelated with a pump on the same floor serving a different room.  In order to corroborate the applicability
of our approach, we compared the pump trace with \emph{all} 674 sensor traces and found a strong correlation
between the relative spatial position of the sensors and their IMF correlations.  The most correlated IMFs were 
serving the same
area in the building.  As we relax the admittance criteria we find that the spatial correlation expands radially from
the main location served by the reference trace.

We plan to examine the use of this method in applications that help discovery changes in underlying relationships over time
in order to identify opportunities for savings in buildings.  We will use it to build inter-device correlation models
and use these models to establish ``normal'', ``abnormal'' usage patterns.  We hope to take it a step further and include a
supervised learning approach to distinguish between ``efficient'' and ``inefficient'' usage patterns as well.






