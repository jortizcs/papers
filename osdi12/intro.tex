\section{Introduction}
% technology trends
Technology trends towards greater instrumentation, data collection, and analysis are clear: low cost computation and communication make it feasible to collect data at much wider scale and finer granularity than was ever before practical.  At societal scales, even modest data volumes ``add up'' and become a challenge to store, process, and ultimately extract value from.  We are in the midst of a fundamental transition: data is changing from being a rare, high-quality, and clean product to a dirty yet ubiquitous commodity.  
% \subsection{Is this ``Big Data''?}

This transition is taking place in many sectors of the economy at different rates.  Traditional computing environments have been among the first to move towards this new world, since the infrastructure for collecting and managing the data can largely be built on the computing infrastructure itself.  Other areas like manufacturing plants and supply chains have also undergone significant automation and instrumentation as corporations seek additional efficiencies and cost-savings through strategies like just-in-time manufacturing and real-time inventory control.  However, the built environment -- the public, commercial, and residential buildings and streets where most people spend nearly all of their time -- has undergone significantly less change despite these underlying trends.

% the life of data
When considering the built environment, it is important to consider the sources and types of data we are concerned with.  In some sense, buildings are like plants, manufacturing a hospitable indoor environment; there are a large number of sources of data surrounding this process: data about temperature, humidity, set-points, flows, and light levels at many points of the plant.  However, in important ways buildings are {\it not} like plants, because they are essentially ``open systems'' whose operation is subject to the behavior of occupants.  Information about the occupants exists in many forms: building access logs, network traffic data, security camera footage, and work schedules yet is rarely incorporated into the system control plan.  Furthermore, the built environment exists within a larger system, some elements of which are amenable to prediction and analysis; weather and traffic forecasts are routine, and many more phenomena like water levels, snow-pack conditions, solar and wind energy availability, and fuel prices are routinely observed and could be factored into planning and control loops.

% existing things are broken
Existing systems systems in the built environment are particularly ill-suited to the data-centric and networked future.  Although some system components have been brought up-to-date, the architecture has not significantly evolved from the day where state-of-the-art control meant programing PID loops and ladder logic into mechanical computers.  As a result, integrating these components into a modern networked systems architecture is challenging because the underpinnings are obscure or proprietary, and the legacy systems do not provide the isolation, security, safety, and liveness properties that would allow the deployment of new technologies while lowering the consequences of failure.  Our contribution is to propose a forward-looking distributed operating system architecture that addresses these shortcomings; allowing management and control of distributed physical resources while enabling a broad range of applications for the built environment.
