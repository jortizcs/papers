\section{Process Management}
\label{sec:promngt}

The process management unit in StreamFS manages jobs submitted by users.  There are two different kinds of jobs: internal
jobs and external jobs.  Internal jobs are managed internally in a process-element cluster.  StreamFS accepts javascript code
from the user and maintains it until it is activated.  Activation occurs when a pipe process to the specified job
is instantiated.  External jobs are written in any language that run on the client machine and communicate with StreamFS
through the external-process job stub.  An external processing job
is one that interacts with external code.  StreamFS creates an associated file for managing 
the external jobs from a central location.

Processes are managed through the filesystem.  Our design is guided by the principle that everything is a file and
every entity is represented in the filesystem.  We use the term subscription and pipe interchangeably, however, for 
explicit disambiguation, we define a subscription as a one-way forwarding process of stream data to an external
target.  A pipe is a type of subscription to a stream from an instance of a process file.  The process can
be internal or external and \emph{always} has its output represented by a stream file.  The latter allows us
to construct processes chains that can be linked via their associated stream files.
This section discusses the difference between internal and external processing jobs.  
% In short, internal processing jobs
% are written in javascript and managed within a StreamFS process-cluster manager.  
