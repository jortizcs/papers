\section{Summary}

In this chapter we described the details and motivation in the process management and related components.  We introduced the 
notion of internal and external processing.  The former is used for small, simple data-cleaning jobs while the latter
is for integrating external processing jobs written in the client's native language.
We also showed how we comine the entity-relationship graph to provide the infrastructure necessary to support OLAP-style queries.
This is an important features, since many of the queries posed in the building domain have the following properties:

\begin{enumerate}
\item Temporally-driven, scan-heavy queries.
\item Hierarchical, unit-specific aggregates.
\end{enumerate}

Dynamic aggregation is an efficient design for these kinds of queries.  Unlike traditional OLAP, where the timestamps
are uniform across other dimensions, we must interpolate the values to keep the ``OLAP cube'' populated with data at all
intervals.  It is also necessary to provide accurate aggregates in time.

Finally, we articulate our observation of the importance of scheduling with jobs that want a set of readings that are collectively
the latest -- the collective buffer freshness is maximized.  We formalize the problem and present an algorithm solution and evaluation.
In the next chapter we discuss the files and associated semantics in StreamFS.  We show can they related to traditional filesystems
and discuss the motivation for its design.  We also present the mathematical tools for verifying the relationships between sensors that
is constructed through the namespace.