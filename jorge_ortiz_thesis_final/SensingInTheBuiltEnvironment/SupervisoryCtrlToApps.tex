\section{From Supervisory Control to Application Development in Buildings}
%From Supervisory Control to Application Development in Buildings

% Building information systems were built as tightly integrated silos, where the main application is the graphical user interface.
Features for interoperability were designed at two interface layers: 1) the protocol layer and 2) the presentation layer through a
data-export feature.  The protocol layer provides services for enabling devices to talk to one another through the network.  Note, 
we focus on BACNet, but the similar features exist in other protocols, such as LonTalk.  Several features were explicitly designed around
the notion of interoperabilty: trending, scheduling, management services, alarms and events, direct sharing.  The graphical
interface layer is mainly focused on providing periodic reports in a comma-separated value (CSV) file, which
contains point-name information and time-value pairs of data.
% Historically, the extent of interoperability objectives has included mainly the introduction of new devices onto the network or
% the exporting of data for other software program to use for analysis and report generation.  
For these protocols, interoperability means adding new devices or exporting the data is a common format.
Building informtation systems themselves
were built to mainly support the in-time management and supervisory control of the building.  Analysis does not 
extend far beyong univariate plotting and individual assessment of equipment and control.  Most tuning of control parameters is 
manual.  Hard-wired control logic at the outstation is rarely updated.


Over the last few years, however, there has been in increased interest in energy management and comfort as a primary objective 
in the design of new building applications~\cite{6146507,Yu1956572,Mamidi2343582}.  Moreoever, there is a broad interest in having buildings 
participating in hierarchical, 
global control schemes that optimize the performance of the grid in response to renewable source penetration and its inherent 
generation volatility~\cite{Taneja2223873,5985456,Lu2009}.  Model predictive control has introduced new ways of controling the components in the building
to make them more energy efficient~\cite{mpc}, there is an interest in performaning dynamic, real-time analysis of building health
and efficiency~\cite{dynamicLeed}.  There has even been interest is improving the visibility of the state of the building to the 
occupants through various modalities, including touch-screens and mobile phones~\cite{andrew_lighting, Hsu1878444, Ortiz2422540}.  
These, and other emerging applications,
have pushed the boundaries of demand beyond what a modern building information system can provide and a re-design must be considered.