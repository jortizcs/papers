\section{From Supervisor Control to Application Development in Buildings}


Building information systems were built as tightly integrated silos, where the main application is the graphical user interface.
Features for interoperability were designed at two interface layers: 1) the protocol layer and 2) data export features in the 
graphical interface.  The protocol layer -- we focus on BACNet, but the similar features exist in othe protocols, such as LonTalk --
provides services for enabling devices to talk to one another through the network.  Several features were explicitly designed around
the notion of interoperabilty, namely trending, scheduling, management services, alarms and events, direct sharing.  The graphical
interface layer is mainly focused on providing periodically reports, typically in the form of a comma-separated value file, which
contains point-name information and time-value pairs of data.

Historically, the extent of interoperability objectives has included mainly the introduction of new devices onto the network or
the exporting of data for other software program to use for analysis and report generation.  Building informtation systems themselves
were constructed mainly to support the in-time management and supervisor control of the building.  Analysis does not 
extend far beyong univariate plotting and individualized assessment of equipment and control not far beyond manual parameter tuning
of hard-wired control logic at the outstation.

Over the last few years, however, there has been in increased interest in energy management and comfort as a primary objective 
in the design of new building applications~\cite{}.  Moreoever, there is a broad interest in having buildings participating in hierarchical, 
global control schemes that optimize the performance of the grid in response to renewable source penetration and its inherent 
generation volatility~\cite{}.  Model predictive control has introduced new ways of controling the components in the building
to make them more energy efficient~\cite{mpc}, there is an interest in performaning dynamic, real-time analysis of building health
and efficiency~\cite{dynamicLeed}.  There has even been interest is improving the visibility of the state of the building to the 
occupants through various modalities, including touch-screens~\cite{} and mobile phones~\cite{andrew_lighting, buildsys1, buildsys2}.  
These are other emerging applications
have pushed the boudanries of the capabilities of building information systems and a re-design must be considered.