% \section{Architectural Re-design}
% We propose a new architecture to address the shortcomings we described.  The architecture deals with challenges 
% of real-time data processing, time-series data storage, and uniform access to building sensor data.  It also
% addresses the challenges with contextual accuracy.

\section{Summary}

% We use StreamFS to organize and support applications using building data from three different
% buildings.  The first one is a 110,000 square foot, seven-story building, the second one is an eleven-story
% 250,000 square-foot building, and the third is a 150,000 square-foot eleven-story building.

In this chapter we discussed the historical motivation for the design and implementation of building information systems.
We showed how they were built with the purpose of supporting the building manager's primary job of maintaining the 
building and dealing with occuppant complaints.  A BMS displays the data from a large sensor deployment and show it in the
context of the building schematics to the building manager at a central location (i.e. their office), so that they can 
find problems through a series of clicks, rather than physically walking around the building and inspecting
individual sensors and systems.

We discussed the notion of building ``app-ification'' in greater detail, separating the BMS intwo three distinct layers
and focusing on the graphical interface as just one possible interactive modality for buildings.  We explaine that if we
view the interface as just another interface, then the narrow waiste of interaction if set at the naming layer.  We also discuss
several emerging building applications: the graphical supervisory control app, the holistic building optimization app,
and the mobile auditing app.  We show that the current approach does not contain the necessary layering or architectural components
and mechanisms to support the emergy applications and consider what components \emph{are} necessary in order to do so.

We also introduce the notion of verification of building elements through software.  We talk about the
relationships betweens the systems and spaces, expressed in the naming of sensor streams and how these relationships are used
to interpret context necessary for emerging applications.  We argue that verification at the sofware level, should be part of any
architecture that manages the building.  We give an example through model predictive control, and how the mis-representation or
staleness of information can lead to gross control-related and energy-accounting errors, especially long term.

We have designed and implemented a system called StreamFS that contains each of these features as components in its architecture.
For points \ref{nw}, \ref{proc}, and \ref{rt} in section \ref{sec:shortcomings}, we observe that filesytem constructs solve these 
issues quite effectively.  We show
how a uniform, hierarchical naming scheme combined with a high-level pipe abstraction can be used to address issues with naming, processing,
and management of many streams.  We discuss how we translate these into different kinds of files in StreamFS, each with their
own read/write semantics and how you can construct complex processing pipes easily through this abstraction.  Point \ref{ts} requires
a traditional timeseries database approach.  In chapter~\ref{chap:SFSArchMain}, we explain how the time-oriented, scan-heavy
workload calls for a timeseries data store and how we combine the datastore with the access semantics of specific files in
StreamFS.

We consider point \ref{cntxt}, an important component in our archictectural proposal.  However, 
we separate it from the StreamFS architecture because it is a component that functions largely independently of \emph{how} the naming
constructs are logically constructed.  Instead, this piece of the thesis focuses more on the mathematics and methodological
underpinnings of a verification process that should be included in \emph{any} architecture that describes physical relationships
and monitors functional behavior of equipment in buildings.  We leave the details of our approach and methodology to chapter~\ref{chap:naming}
and the results are presented in chapter~\ref{chap:VerificationMain}.