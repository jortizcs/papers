% \section{Architectural Re-design}
% We propose a new architecture to address the shortcomings we described.  The architecture deals with challenges 
% of real-time data processing, time-series data storage, and uniform access to building sensor data.  It also
% addresses the challenges with contextual accuracy.

\section{Summary}

% We use StreamFS to organize and support applications using building data from three different
% buildings.  The first one is a 110,000 square foot, seven-story building, the second one is an eleven-story
% 250,000 square-foot building, and the third is a 150,000 square-foot eleven-story building.

In this chapter we discussed the historical motivation for the design and implementation of building management systems.
We showed how they were built, with the goal to support the building manager of her primary job: to maintain the 
building and deal with occupant complaints.  A BMS displays the data from a large sensor deployment and displays it in the
context of the building schematics, from at a central location (i.e. their office).  This makes it easier to  
find problems through a series of clicks, rather than by  walking around the building and inspecting
individual sensors and systems.

We introduce the notion of building ``app-ification''; separating the BMS into three distinct layers
and focus on the graphical interface as one possible interactive modality for buildings.  We explain that if we
it as one of many possible interfaces we can raise the narrow waist of interaction to the naming layer.  We also discuss
several emerging building applications: the graphical supervisory control app, the holistic building optimization app,
and the mobile auditing app.  We show that current systems does not contain the necessary layering or architectural components
and mechanisms to support emerging applications.  We propose a set of necessary components.
 % and consider what components \emph{are} necessary in order to do so.

We also introduce the notion of verification of building functionality and metadata, through software.  We talk about the
relationship between systems and spaces, expressed in the naming of sensor streams and how these relationships are used
to interpret context.  We argue that verification should be part of any
architecture that manages the building.  We give a model predictive control example and discuss how the mis-representation or
staleness of information can lead to gross contro/energy-accounting errors, especially long term.

% % We have design and implemented a system called StreamFS that contains each of these features as components in its architecture.
% For points \ref{nw}, \ref{proc}, and \ref{rt} in section \ref{sec:shortcomings}, we propose that filesytem constructs solve these 
% issues quite effectively.  We show
% how a uniform, hierarchical naming scheme combined with a high-level pipe abstraction can be used to address issues with naming, processing,
% and management of many streams.  We discuss how we translate these into different kinds of files in StreamFS, each with their
% own read/write semantics and how you can construct complex processing pipes easily through this abstraction.  Point \ref{ts} requires
% a traditional timeseries database approach.  In chapter~\ref{chap:SFSArchMain}, we explain how the time-oriented, scan-heavy
% workload calls for a timeseries data store and how we combine the datastore with the access semantics of specific files in
% StreamFS.

% We consider point \ref{cntxt}, an important component in our architectural proposal.  However, 
% we separate it from our architecture because it is a component that functions largely independently of \emph{how} the naming
% constructs are logically constructed.  Instead, this piece of the thesis focuses more on the mathematics and methodological
% underpinnings of a verification process that should be included in \emph{any} architecture that describes physical relationships
% and monitors functional behavior of equipment in buildings.  We leave the details of our approach and methodology to chapter~\ref{chap:naming}
% and the results are presented in chapter~\ref{chap:VerificationMain}.