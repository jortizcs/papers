
\section{Addressing BMS Shortcomings}
\label{sec:shortcomings}
% The current architecture of building systems is very tightly integrated and based on monitoring and supervisory control
% of local control loops.  
% Building systems were built as tightly integrated systems with a single application.  
The main layer of interaction between applications and BMS's is the underlying network layer and the data-export component.  
The BMS export feature decouples the protocol from the information about each sense/actuation point; time-value
pairs and the name of the point. As auditing applications emerged and energy became a prime target for reduction in buildings, 
these interface choices became insufficient.  Moreover, as the need to construct new classes of applications emerges, the architectural
pieces that are missing become more clear.  The following is a list of some of them:

\begin{enumerate}
% \item Network protocol details should remain opaque to end-user applications.
\item Narrow waist should be above the network layer. \label{nw}
\item A time-series store is necessary. \label{ts}
\item Mechanisms to distill the readings must be availble. \label{proc}
\item Real-time data forwarding should be available, especially for control applications. \label{rt}
\item Contextual relationships between sensor should be verified. \label{cntxt}
\end{enumerate}

The first four items are commonly built and re-built in emerging applications.  Therefore, we argue that they are fundamental 
to the future architecture of building information systems.  Moreover, we observe that dealing with network-protocol specific
calls is not only cumbersome, but usually circumvented in order to deal directly with the data.  Most applications that do
use the underlying protocol expose a name-time-value (NTV) tuple to the layers above.  This observation leads us to believe that
that's where the interface should be.

The NTV layer allows us to decouple the data from the network protocol.  This makes it easier to include 
new sensors that may not be directly on the building network; since the only information we need is the point name and the data it
produces.  For example, wireless plug-load power meters~\cite{acme}
can join the NTV layer by registering the individual points, while a translation layer between the NTV layer and the
wireless meter router provides the transformation of read/write request to/from points in the network.  The same is true for BACNet
or any point protocols for sensors/actuators.  Like many problems in computer science, this one can be solved through layer separation
and a level of indirection and translation.

Each of the services that require the end user to have a deeper understanding of the underyling dynamics 
of the building \emph{must} capture the notion of time.  Almost all anlytical processes or control decisions need a set of readings
over time.  Therefore, there a time-series data store must be part of future BMS design.  The service should be made available
through the NTV.  %This will allow applications to fetch the data for analysis either for display or complex processing.

Point \ref{proc} is motivated by the observation that sensor data, especially from cheap sensors, is dirty and typically goes
through a cleaning process before being forwarded to the application.  There are various operations that are commonly
performed on the data, that should be available as primitives.  These include re-sampling, filtering,  
missing-data identification, and aggregation.  Re-sampling refers to taking a set of streams and interpolating missing values to 
align their timestamps.  This is usually performed before aggregation, especially for generating time-varying aggregate statistics.
Filtering removes certain values based on a threshold value(s).  
% Usually the criteria is defined
% by a threshold, both lower-bound and upper-bound value threshold for a particular stream or set of streams.
Since data is often missing, due to intermittent connectivity problems or faulty sensor equipment, it becomes important to 
get a summary of missing time intervals in order to adjust the fetch parameters.  Finally, the data is usually more
useful in aggregate than as a univariate signal; for example, for generating a load curve/ % for a set of energy-consuming items.
Simple operators for combining values of various streams is key to enabling this procedure.

Finally, in order to enable control, real-time mechanisms must be exposed to the control application.
% , while maintaining the 
% layered integrity of the NTV layer.  
In addition, we observe the need to provide real-time services for analytical applications.
For example, LEED is proposes the use of building data to provide a dynamic performance metering~\cite{dynleed}.
There are also many dashboard companies that make use of streaming data to provide real-time statistics on the performance of the
building.  The mobile energy-audit application, from section~\ref{sec:mobile}, also requires a real-time forward and processing service.
%  to
% enable the application.  We believe that as more applications emerge they will likely need make use of real-time sensor data.

%extendible: able to add and remove stuff 
%scalable: able to scale with applications, data, and deployment size
%generalizable: able to accomodate many kinds of analytical/control applications
%ease of management: so many distributed things that it's hard to keep track of where everything lives.

The design of a new system must provide the features highlighted above and contain the following properties:

\begin{enumerate}

\item \emph{Extensibility}:  The system should be able to accomodate different kinds of sensors and actuators and it should
be simple add and remove them.

\item \emph{Scalability}:  The system should scale with the size of the deployment and the number of applications.% that it supports.

\item \emph{Generalizability}:  The system should provide a general set of primitives for application designers.  It should support applications
described in this chapter and emerging applications that we cannot currently anticipate.

\item \emph{Ease of Management}: The system should make it easier to manage large deployments and their associated applications.

\end{enumerate}

Modern BMS architectures do not contain these properties.  They are difficult to extend, as new sensors and actuators must physically join 
the network and follow both a high-level and low-level protocol in order to do so.  They are not scalable.  Most BMS's have a limit as to
how fast they can obtain data from sensors and limit the amount of trending that the system can do.  The central outstation is the only
machine handling incoming data.  The entire code-base runs on a single machine.  There are bottlenecks throughout the system in regards to
data storage -- including the outstation memory and disk storage on the local machine that houses the BMS.  BMS's are also not generalizable.
They only support one ``applicatoin'': the graphical interface.  The GUI does have a trending/plotting option, but extending the BMS
to provide other kinds of services is impossible.  Finally, the scope of management is quite limited in BMS's and although they do provide
ease of management of sensor/actuators on the system through centralized access, we contend that the scope is simply not broad enough.

In this thesis, we will describe a new system, StreamFS, which contains the properties missing in the current architecture.  We will demonstrate 
the existence of these properties through a series of applications that were built over it in several settings.  We describe the API and
the scope and usage of the application and draw out how the capabilities enabled by StreamFS in those apps demonstrate the properties highlighted
above.

































