% (This is included by thesis.tex; you do not latex it by itself.)

\begin{abstract}

% The text of the abstract goes here.  If you need to use a \section
% command you will need to use \section*, \subsection*, etc. so that
% you don't get any numbering.  You probably won't be using any of
% these commands in the abstract anyway.

% Invasive brag; forbearance.



This thesis examines the state of the art of building information systems and evaluates their architecture in the context
of emerging technologies and applications for deep analysis of the built environment.  
We observe that modern building information systems are difficult to extend, do not provide general services for application development, do not
scale, and are difficult to set up and manage.  
We propose a new architecture that embodies these system principles through a filesystem abstraction and data services, called StreamFS.  
It also provides a number of data services, made available to applications through the pipe abstraction.

We deploy StreamFS in seven different buildings with very different setting and compose several applications on top of it.  One of the driving
application is the Mobile Energy Lens.  The Energy Lens app provides occupants with mechanisms for collecting building information
in a unified platforms and provides a way to view aggregate energy consumption associated with the spatial deployment
of plug-load devices.  We present a 3-layer application architecture, where one of the main layers is implemented entirely through
StreamFS data management and data processing services.  

We introduce the notion of verification of physical relationship through empiricial data.  
We partition the verification problem into three sub problems: 1) functional verification, 2) spatial verification, and 3) categorical
verification.  We show how empirical mode decomposition, correlation, and simple machine learning techniques can give us 
information about how the sensors are related to each other, statistically and physically.
We demonstate an \emph{extensible, generalizable, scalable, and easy-to-manage} system for supporting the ``appification'' of the 
built environment.  


% We examine the 
\end{abstract}
