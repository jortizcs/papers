% (This is included by thesis.tex; you do not latex it by itself.)

\begin{abstract}

% The text of the abstract goes here.  If you need to use a \section
% command you will need to use \section*, \subsection*, etc. so that
% you don't get any numbering.  You probably won't be using any of
% these commands in the abstract anyway.

% Invasive brag; forbearance.



This thesis examines the state of the art of building information systems and evaluates their architecture in the context
of emerging technologies and applications for deep analysis of the built environment.  
We observe that modern building information systems are difficult to extend, do not provide general services for application development, do not
scale, and are difficult to set up and manage. 
We assert that a new architecture must be designed with four system properties -- \emph{extensibility, generalizability, scalability,
ease of management} -- in order to address these shortcomings.  
Our system, StreamFS, embodies these system properties through a filesystem abstraction and a set of data services.
% We propose a new architecture that embodies these system principles through a filesystem abstraction and data services, called StreamFS.  
Data services are made available to applications through an overloaded pipe abstraction.  This allows for dataflow specification
of processing streams to clean and analyze the streaming sensor data.

We deploy StreamFS in seven different buildings and compose several applications on top of it.  One of the driving
applications is a phone application called the Mobile Energy Lens.  The Energy Lens provides occupants with mechanisms for 
collecting building information
in a unified platform and provides a way to view aggregate energy consumption data associated with the spatial deployment configuration 
of plug-load devices.  We present a three-layer architecture, where one of the main layers is implemented entirely with 
the data management and processing services offered by StreamFS.  

We introduce the notion of verification of physical relationships through empiricial data.  
We partition the verification problem into three sub problems: 1) functional verification, 2) spatial verification, and 3) categorical
verification.  We show how empirical mode decomposition, correlation, and standard machine learning techniques can give us 
information about how the sensors are related to each other, statistically and physically.
We demonstate an \emph{extensible, generalizable, scalable, and easy-to-manage} system for supporting the ``appification'' of the 
built environment.  


% We examine the 
\end{abstract}
