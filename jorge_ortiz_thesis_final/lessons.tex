
\chapter{Lessons Learned and Future Work}
\label{chap:future}

Our experience with the design, imeplementation, and wide-spread deployment of StreamFS teaches us several lessons about
the mechanisms necessary to ``app-ify'' the building and the value of opening up the building as a platform for application 
development in the interest of 1) reducing energy consumption and increased operation efficiency, 2) providing deeper
insights about the operational dynamics of the building, 3) enabling the building to participate in a broader
software ecosystem in the interest of more intelligent use of resources. 

\section{Lesson Learned}
In this section we discuss the lesson learned from deploying our system.  We present 5 statements that we experienced 
and find true moving forward, all based on our deployment experience and through our many interactions with building
systems, building managers, and the broader building science software ecosystem. 

\subsection{Data Services Fundamental for Building Applications}
Fundamentally application in the building must deal with data coming from a distributed, diverse deployment of sensors
taking physical measurements.  The vast majority of building applications are analytical in nature and as we start coupling
streaming sensor data with actual building model and control, more data services and jobs will be used across 
several applications simultaneously.

We implemented and deployed 7 applications on top of StreamFS.  Some applications were for instance management of the various 
components in
StreamFS, however we built a viewer console application, which initiated time-series queries and displayed them to the user,
simple control applications that triggered a cascading re-activation sequence of home appliances when total energy consumption
was above a user-defined threshold, and the mobile Energy Lens application provides the user with aggregate statistics based on
the spatial configuration of the deployment.

All these applications shared a small set of processing elements.  One of the elements is for removing statistical outliers, 
another for interpolating values, one for computing the aggregate load curve, and another to compute a moving average of consumption.
Because StreamFS provides the ability to define the function once, it could be used across applications.
The hierarchical model prediction application presented in Chapter~\ref{chap:SensingInBuiltMain} would also require similar
\emph{clean} and \emph{aggregate} capabiltiies before the data is fed into the model.  For jobs that require any 
prediction or machine learning, the same assertion holds.

\subsection{Analytics is Dependent on Inter-relationships}
In StreamFS and the associated applications we demonstrate the value of coupling the inter-relationships explcitly with 
our aggregation jobs.  Many jobs require the streams be fetched in the context of the metadata of the deployment and what 
the metadata captures about the placement or category of the stream.  We find that it is convenient to consider
coupling these associations explicitly, since many applications require it.  We found the use of \emph{aggregation points}
to be a convenient tools for analysis, both from a historical perspective and from a live feed perspective.

Moreover, queries for determining which sensors are where and how many sensors there are in a particular space are crucial
for achieving generality across buildings -- a property that is rare to find in \emph{any} aspect of building design and operation.
Providing the ability to traverse the relationship structure and discover, through the structure, the relationship
between the sensors allows applications from one building to potentially be dropped into a new building without
much learning.  With this kind of capabilty, we could write applications for one building and port them across the building
stock; allowing a solution to have true impact on the entire building stock.


\subsection{Centralized Management is better than Distributed Management}
StreamFS adopts the Unix philosophy where everything is represented as a file.  This makes it simple to manage the raw
and application-specific derivative data.  It also simplifies the sharing of processing code across applications.  In addition, 
access control can be provided from a centralized location.  Application writers can devide which streams and processes they 
want to share from the individual resource level to groups of resources organized as a collection within a container file.
Generally, when there are many disparate devices that make up a system, there is a tension between extensibility 
and ease of management.  Large deployment typically enable protocols for joining the network, but not ones that are 
easy to manage.  The ones that are easy to manage, typically do not handle other kinds of sensors or devices easily joining
the deployment.

When the deployment is presented as a distributed system to the end user, there are usually complex mechanisms in place for doing 
discovery.  Disovery over a centralized management system is much easier since it is clear where to target the search.
Typically the argument against centralized management is a single point of failure, however the web-services model
deals with centralized services rather well.  The systems is presented in a unified fashion to applications, while 
internally it is actually distributed; shared for load management and replicated for failure tolerance.  StreamFS follows
these principles throughout the design of each of its components.

Similar to the proposal by Dixon et al.~\cite{homeos}, we believe that the abstraction should be raised from the network
to the operating system level.  The Unix philosphy allows us to present unified namespace and access to the building deployment
information and various levels granularity.  This allowed us to manage many applications simulataneously, control information
sharing between them, and provide a unified view to the application writer; making it clear what resources are available to her.
We believe these centralized view is the right one.  Although StreamFS is a distributed system in the cloud, it present a unified 
layer for access and control which makes application development and management simpler.


\subsection{Accurate Capture of Physical Configuration is Crucial}
As we move towards software-defined building infrastructure and more analytical and control applications rely on building 
state to make detailed control decision, it becomes very important to accurately capture the physical configuration of the building
and provide mechanisms for discovering inconsistencies between the virtual representation and the correspinding physical state.
There is a lot of work in the vision community for constructing a virtual representation of the physical world, accurately.
In our work we show ways the the data \emph{already being collected from the building sensor deployments} can be used to
ascertain physical relationships between sensors.  We introduced three kinds of verification -- functional, spatial, and cateogrical --
however there are other ways we could explore these physical relationships.  For example, we can combine 
these technique with vision-related work which uses cameras to determine the physical layout of a space and combine
it to use physical models of heat transfer to determine if the readings we are collecting are accurate.

Ultimately these should inform the layer presented to the application about the accuracy of the deployment information.  We
explained how inaccuracies leads to errors in aggregation and control.  For software interfaces for the built environment
to become more widespread, we must solve these issues related to verification.


\subsection{OS Abstractions for Managing Truly Physical Resources}
Mapping the jobs of organizing and controlling access to physical resources, the use of operating system abstractions
allows us to reason about how to componentize a management architecture and lets us frame how solutions can 
be constructed as applications that make use of the primitives provided by the operating system components.
In our work we presented a filesystem abstraction which adopted several data and management services for organizing
the information in the building.  Related work in the home~\cite{homeos} and in buildings~\cite{boss} take a similar
approach.

Questions remain about how building models fit into a operating system services architecture.  Perhaps they can be
included with the verification services.  Ideally, we can start moving towards automated plug-and-play building applications
with guarantees for service quality and efficient management of physical resources.  We believe that the best way to
make this a reality it to view the building as a hardware platform and to move towards a truly distributed operating system
for managing the systems, devices, and applications that run on the building platform.



\section{Future Work}

We explored many aspect of a design for building information information systems.  However, there are still many open questions have
have not been tackled in this work.  We discuss three main future or on-going project topics in this section.

\subsection{Explore More Diverse Verification Methods}
There's a lot of work in capturing the physical state of the environment and building a virtual representation of it.
Althought SBS shows a lot of promise, in terms of its effectiveness and generality, spatial verification shows poor generality 
under certain conditions and categorical classification seems to only work with small data sets.  We look to expand our 
exploration in the two pieces of work that showed fractional success.  We must characterize the conditions for which these
verification approaches work well and formulate algorithms to detect whether those conditions hold in the data prior to
initiating those verification processes.  We can partition the characterization to discover the statistical or semantic 
properties or pieces of information that must be known about the deployment beforehand and tackle those problems only.

For the others, we must explore different techniques that generalize better.  Although it is valuable to solve the problem for
a small set of homogenous buildings, the real value comes through generalization, since in order to have widespread impact
solutions must be brought to a large fraction of the building stock, quickly.



\subsection{Deeper Exploration of Control Applications}
We did not get a chance to support the kind of control application proposed in the beginning on the dissertation.  It did not
allow us to experiment with vairous kinds of actuator interfaces for integrating them more generally into the architecture.
We intend to expand our control application work by implementing model-predictive control processes on building deployments
that use StreamFS.  We also need to closely examine a diverse set of control interfaces and APIs in order to generalize
the control interface exposed through StreamFS.

We only explored the class of controllers that consume binary signals.  Specifically, we intergrated with the ACme~\cite{acme}
wireless power meter to turn devices on and off, remotely.  There are other kinds of controllers, that do not accept binary input.
They input variable controllers, set-point driven controllers, parameterized controllers, time-based controllers, etc.  For example, 
controllers based on physical models may used physical configuration-based parameters to determine how to drive the 
load for the system controlling the space.  These must be considered and applications should be explored in this context
before a determination can be made.


\subsection{Version Control For Buildings}
Provenance checking is important in many systems.  The building has many actors interacting with it and distributed changes cause
the once efficient configurations to slowly deviate back to an inefficient state.  Version control would allow us to track changes
in the deployment and associated configuration decisions.  It also introduces the notion of state rollback, whereby we roll the associated state
information back to a previous, safe, efficient state.  We would like to explore how rollbacks manifest themselves in the 
physical environment, since it involves not just rolling back the settings, register values, processes, etc. but the instant 
rollback operation affects how the physical resources in the environment are activated.  Also, certain conditions \emph{cannot}
be rolled back, such as the weather conditions, so a rollback operation needs to check if the proper rollback conditions
are in place before the rollback is committed and executed.

In addition, how do we determine commit conflicts between two committers.  In a traditional version control system it is based on the actual
data being written to the file.  In the building context, conflict must be determined by models.  Model based on first-principles
of the underlying physics or statistical models that used several empirically-derived parameters to project the state of 
the system at some point in time.  For example, if two commits are made to change the setpoint of a thermometer, how to we determine
that the setpoints will conflict and how to we resolve them.  Both commits are writing to the same devices, so that cannot be the sole
criteria.  Any approach must consider how the set point affects the state of the room temepature that is controlled by that setting
or the behavior of the air-handling unit when commits are happening quickly.  Either involves the use of a model to determine
what the correct behavior is and project whether the new set point will lead to the right behavior.
We look to explore these and other related concept further and hope that it leads to smart, software-defined buildings.









