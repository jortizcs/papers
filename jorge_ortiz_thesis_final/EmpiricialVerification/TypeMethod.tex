\section{Categorical Verification Methodology}
For categorical classification we took a use a very simple approach.  For every trace, we partition the range into 10 bins
and take the average.  We sort the bins and take the top 2 and the combine it with the average.  This combination forms
our feature vector with three components.

We run our type analysis on three data sets from separate buildings.  The first is a from the University of Tokyo.  It contain 
six types of sensors measuring power, pressure, temperature, CO2, light, and occupancy.
The other is a deployment in Sutardja Dai Hall at UC Berkeley which measures four different type that include lumens, CO2, 
temperature, and humidity.
Finally, we used a data set from Soda hall at UC Berkeley which contains 23 different types.

These data sets were chosen in such a way that we could use their categorical information to verify our approach for classifying them
according to statistical markers of categorical difference.  All the data either comes from sensors embedded in a space or a sub-system taking
a physical reading or represents a set-point setting for an actuators that control the environment.
In some cases, we are easily able to separate the stream categorically, using simple statistical summaries, while other stream -- particularly,
the ones \emph{not} actually be generated by a physical phenomenon (i.e. temperature set point) -- are statistical indistinguishable
from their physical-measurement counterpart; their differences are semantic, not behavioral.  We present our analysis and results
in this section.
