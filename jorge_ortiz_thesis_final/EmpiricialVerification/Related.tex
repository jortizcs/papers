\section{Related Work}

The research community has addressed the detection of abnormal energy-consumption in buildings in numerous ways \cite{katipamula:1review2005,katipamula:2review2005}.
% Detecting abnormal energy-consumption in buildings has recently received the attention of the research community and different approaches have been studied.
% 

The rule-based techniques rely on a priori knowledge, they assert the sustainability of a system by identifying a set of undesired behaviors.
Using a hierarchical set of rules, Schein et al.\ propose a method to diagnose HVAC systems \cite{schein:hvacr2006}.
In comparison, state machine models take advantage of historical training data and domain knowledge to learn the states and transitions of a system.
The transitions are based on measured stimuli identified through a domain expertise.
State machines can model the operation of HVAC systems \cite{patnaik:toist2011} and permit to predict or detect the abnormal behavior of HVAC's components \cite{bellala:buildsys2012}.
However, the deployment of these methods require expert knowledge and are mostly applied to HVAC systems.

In~\cite{seem:energybldg2007}, the authors propose a simple unsupervised approach to monitor the average and peak daily consumption of a building and uncover outlier, nevertheless, the misbehaving devices are left unidentified.

Using regression analysis and weather variables the devices energy-consumption is predicted and abnormal usage is highlighted.
The authors of~\cite{brown:buildperf2012} use kernel regression to forecast device consumption and devices that behave differently from the predictions are reported as anomalous.
Regression models are also used with performances indices to monitor the HVAC's components and identify inefficiencies \cite{zhou:wiley2009}.
The implementation of these approaches in real situations is difficult, since it requires a training dataset and non-trivial 
parameter tuning.

Similar to our approach, previous studies identify abnormal energy-consumption using frequency analysis and unsupervised anomaly detection methods.
The device's consumption is decomposed using Fourier transform and outlier values are detected using clustering 
techniques \cite{Bellala_buildsys11,wrinch:pes2012,chen:aaaiw2011}. %jakkula
However, these methods assume a constant periodicity in the data and this causes many false positives in alarm reporting.  %, thus, they report devices usages happening at unusual times although they may not correspond to a faulty operation.
We do not make any assumption about the device usage schedule.  We only observe and model device relationships.
% We take advantage of a recent frequency analysis technique that enables us uncover the inter-device relationships~\cite{romain:iotapp12}.
% The identified anomalies correspond to devices that deviate from their normal relationship to other devices.

Reducing a building's energy consumption has also received a lot of attention from the research community.
% Since HVACs are one of the major electricity consumer in the buildings, several researchers have mainly focused on reducing the consumption of HVACs.
The most promising techniques are based on occupancy model predictions as they ensure that empty rooms are not over conditioned needlessly.
Room occupancy is usually monitored through sensor networks \cite{agarwal:ipsn2011,erickson:ipsn2011} or the computer network traffic \cite{kim:buildsys2010}.
These approaches are highly effective for buildings that have rarely-occupied rooms (e.g. conference room) and studies show that such approaches
 can achieve up to 42\% annual energy saving.
% However, these occupancy model predictions track human activity through sensor networks that usually imply the extra cost and privacy concerns.
SBS is fundamentally different from these approaches.  SBS identifies the abnormal usage of any devices rather than optimizing the normal usage of specific devices.
Nevertheless, the two approaches are complementary and energy-efficient buildings should take advantage of the synergy between them.

Recently, there has been increased interest in minimizing building energy consumption.  Our approach
differs quite substantially from related work.
Agarwal et al.~\cite{occmodels_buildsys11} present a parameter-fitting 
approach for a Gaussian model to fit the parameters of an occupancy model to match the occupancy data 
with a small data set.  
The model is then used to drive HVAC settings to reduce energy consumption.  We ignore occupancy entirely 
in our approach.  It appears as a hidden factor in the correlation patterns we observe.

Kim et al.~\cite{kim:buildsys2010} use branch-level energy monitoring and IP traffic from user's PCs to determine the
causal relationships between occupancy and energy use.  Their approach is most similar to ours.  Understanding how IP 
traffic, as a proxy for occupancy, correlates with energy use can help determine where inefficiencies may lie.

%{\bf Anomaly Detection Using Projective Markov Models in a Distributed Sensor Network~\cite{buildanomaly}}

%{\bf Duty-cycling buildings aggressively: The next frontier in HVAC control}~\cite{AgarwalBDGW11}

In each of these studies and others~\cite{kaminthermo, buildanomaly}, occupancy is used as a trigger
that drives efficient resource-usage policies.  Efficiency
when unoccupied means shutting everything off and efficiency when a space is occupied means anything
can be turned on.  There is no question that this is an excellent way to identify savings opportunities, however, we
take a fundamentally different approach.  We are agnostic to the underlying cause or driver for efficient
policies to be implemented.  More generally, we look to understand \emph{how the equipment is used in
concert}.  This may help uncover unexpected underlying relationships and can be used in an anomaly detection application
to establish ``(in)efficient'', ``(ab)normal'' usage patterns.  The latter 
should identify savings opportunities in cases where the space is unoccupied as well 
as occupied, because it has to do with the underlying behavior of the machines and how they generally work
together.  Our approach could help achieve both generality and scale for such an application.
This article focuses on the first step of this application, the identification of correlated devices.
% Our approach is flexible to be helpful when the assumption does not hold by either allowing the
% user to specify what efficient usage is or through the discovery of efficient usage over various time scales.
% In either case, the flexibility of our approach is its strength from the perspective of generality and scale.







SBS is a practical method for mining device traces, uncovering hidden relationships and abnormal behavior. 
In this paper, we validate the efficacy of SBS using the sensor metadata (i.e. device types and location), however, these 
tags are not needed by SBS to uncover devices relationships.
Furthermore, SBS requires no prior knowledge about the building and deploying our tool to other buildings requires no human intervention --
neither extra sensors nor a training dataset is needed. 

%best effort
SBS is a best effort approach that takes advantage of all the existing building sensors.
For example, our experiments revealed that SBS indirectly uncovers building occupancy through device use (e.g. the elevator in the Building 2). 
The proposed method would benefit from existing sensors that monitor room occupancy as well (e.g. those deployed in~\cite{agarwal:ipsn2011,erickson:ipsn2011}).  % albeit they are no needed.
Savings opportunities are also observable with a minimum of 2 monitored devices and building energy consumption can be better understood after using SBS.

% good=bad..
SBS constructs a model for normal inter-device behavior by looking at the usage patterns over time, thus, we run the risk that
a device that constantly misbehaves is labeled as normal.  % is considered as normal by SBS.
Nevertheless, building operators are able to quickly identify such perpetual anomalies by validating the clusters of correlated devices uncovered by SBS.
The inspection of these clusters is effortless compare to the investigation of the numerous raw traces.  
Although this kind of scenario is possible it was not observed in our experiments.
%Note, that this type of anomaly is unseen in our experiments and is expected to be rare.

%EMD
%Striping sensor data with EMD is beneficial for other work.
In this paper, we analyze only the data at medium frequencies, however, we observe that data at the high frequencies and residual data (Figure \ref{fig:heatmap}) also permits us to determine the device type.  % as well.
This information is valuable to automatically retrieve and validate device labels -- a major challenge in building metadata
management.

There has been much research work on sensor stream clustering and trace analysis. Chen and Tu~\cite{DStream} investigate 
how to cluster data streams in real-time using a density-based approach with a two-tiered framework. The first tier captures 
the dynamics of a data stream with a density decaying technique and then maps it to a grid.  The second tier computes a grid 
density based on how it clusters the grid. Their approach differs from ours in that they focus on decreasing algorithm 
complexity for real-time sensor stream clustering.  We run our analysis on historical traces and use correlation analysis
in our clustering algorithm.

Kapitanova et al.~\cite{failure} describe a technique to monitor sensor operations in the home and identify sensor failures. 
The classifier is trained on historical sensor data to obtain the relationship between sensors, assuming the number and location of 
sensors is known.  When a failure or removal of a sensor occurs, the classifier's behavior deviates and the event is captured. Our method does not require any prior knowledge and instead tries to cluster feeds to discover their relative placement.

Lu and Whitehouse~\cite{blueprints} formulate a new algorithm, particularly leveraging the semantic constraints interpreted from sensor 
data to determine sensor locations. The algorithm identifies how many rooms are present using motion sensors and determines room position based on physical constraints. Finally, it maps each sensor into the associated room. Our efforts focus on using intrinsic patterns typically pre-existing in building system sensor feeds to uncover physical relationships.

Fontugne et al.~\cite{IOT} propose a new method to decompose sensor signals with EMD.
They extract the intrinsic usage pattern from the raw traces and show that sensors close to each other have higher intrinsic correlation. However, they do not explore the observation more deeply by answering whether there is a statistically discoverable boundary between sensor clusters in different rooms, or if there is a uniform threshold in the correlation coefficients able to be generalized to different rooms.

Fontugne et al.~\cite{SBS} carry on the work and propose an unsupervised method to monitor sensor behavior in buildings. They constructed 
a reference model out of the underlying pattens, obtained with EMD,  and use it to compare future activity against it.  They report an anomaly whenever a device deviates from the reference. This work exploits EMD as a method to detrend the signals and capture the inter-device relationships.

Much work utilizes EMD on medical data~\cite{ecg}, speech analysis~\cite{speech}, image processing~\cite{ip} 
and climate analysis~\cite{climate}. Our method adopts EMD to determine whether a discoverable statistical boundary exists in sensors traces
from sensors in different rooms and whether such a boundary
 can be generalized across rooms with various kinds of sensors.