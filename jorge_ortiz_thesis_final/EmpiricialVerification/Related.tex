\section{Related Work}

SBS is a practical method for mining device traces, uncovering hidden relationships and abnormal behavior. 
In this paper, we validate the efficacy of SBS using the sensor metadata (i.e. device types and location), however, these 
tags are not needed by SBS to uncover devices relationships.
Furthermore, SBS requires no prior knowledge about the building and deploying our tool to other buildings requires no human intervention --
neither extra sensors nor a training dataset is needed. 

%best effort
SBS is a best effort approach that takes advantage of all the existing building sensors.
For example, our experiments revealed that SBS indirectly uncovers building occupancy through device use (e.g. the elevator in the Building 2). 
The proposed method would benefit from existing sensors that monitor room occupancy as well (e.g. those deployed in~\cite{agarwal:ipsn2011,erickson:ipsn2011}).  % albeit they are no needed.
Savings opportunities are also observable with a minimum of 2 monitored devices and building energy consumption can be better understood after using SBS.

% good=bad..
SBS constructs a model for normal inter-device behavior by looking at the usage patterns over time, thus, we run the risk that
a device that constantly misbehaves is labeled as normal.  % is considered as normal by SBS.
Nevertheless, building operators are able to quickly identify such perpetual anomalies by validating the clusters of correlated devices uncovered by SBS.
The inspection of these clusters is effortless compare to the investigation of the numerous raw traces.  
Although this kind of scenario is possible it was not observed in our experiments.
%Note, that this type of anomaly is unseen in our experiments and is expected to be rare.

%EMD
%Striping sensor data with EMD is beneficial for other work.
In this paper, we analyze only the data at medium frequencies, however, we observe that data at the high frequencies and residual data (Figure \ref{fig:heatmap}) also permits us to determine the device type.  % as well.
This information is valuable to automatically retrieve and validate device labels -- a major challenge in building metadata
management.

There has been much research work on sensor stream clustering and trace analysis. Chen and Tu~\cite{DStream} investigate 
how to cluster data streams in real-time using a density-based approach with a two-tiered framework. The first tier captures 
the dynamics of a data stream with a density decaying technique and then maps it to a grid.  The second tier computes a grid 
density based on how it clusters the grid. Their approach differs from ours in that they focus on decreasing algorithm 
complexity for real-time sensor stream clustering.  We run our analysis on historical traces and use correlation analysis
in our clustering algorithm.

Kapitanova et al.~\cite{failure} describe a technique to monitor sensor operations in the home and identify sensor failures. 
The classifier is trained on historical sensor data to obtain the relationship between sensors, assuming the number and location of 
sensors is known.  When a failure or removal of a sensor occurs, the classifier's behavior deviates and the event is captured. Our method does not require any prior knowledge and instead tries to cluster feeds to discover their relative placement.

Lu and Whitehouse~\cite{blueprints} formulate a new algorithm, particularly leveraging the semantic constraints interpreted from sensor 
data to determine sensor locations. The algorithm identifies how many rooms are present using motion sensors and determines room position based on physical constraints. Finally, it maps each sensor into the associated room. Our efforts focus on using intrinsic patterns typically pre-existing in building system sensor feeds to uncover physical relationships.

Fontugne et al.~\cite{IOT} propose a new method to decompose sensor signals with EMD.
They extract the intrinsic usage pattern from the raw traces and show that sensors close to each other have higher intrinsic correlation. However, they do not explore the observation more deeply by answering whether there is a statistically discoverable boundary between sensor clusters in different rooms, or if there is a uniform threshold in the correlation coefficients able to be generalized to different rooms.

Fontugne et al.~\cite{SBS} carry on the work and propose an unsupervised method to monitor sensor behavior in buildings. They constructed 
a reference model out of the underlying pattens, obtained with EMD,  and use it to compare future activity against it.  They report an anomaly whenever a device deviates from the reference. This work exploits EMD as a method to detrend the signals and capture the inter-device relationships.

Much work utilizes EMD on medical data~\cite{ecg}, speech analysis~\cite{speech}, image processing~\cite{ip} 
and climate analysis~\cite{climate}. Our method adopts EMD to determine whether a discoverable statistical boundary exists in sensors traces
from sensors in different rooms and whether such a boundary
 can be generalized across rooms with various kinds of sensors.