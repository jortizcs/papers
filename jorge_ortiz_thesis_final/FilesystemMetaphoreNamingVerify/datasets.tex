\section{Experimental Building Deployments and Datasets}

In this thesis, we run all of our experiments and analysis on data from the four buildings.  They serve as 
the main sites for experiemental data collection and/or sites from which we obtained data dumps from their building
management system.

% In addition, we expect the anomaly detector to identify discontinuities in these relationships that represent obvious electricity saving opportunities (e.g. a room HVAC left on during night while the room lights have been turned off).

% Due to privacy concern this dataset is not publicly available on the Internet but accessible upon request.

\subsection{Cory Hall - UC Berkeley}
Cory Hall, at UC Berkeley, is a 5-story building hosting mainly classrooms, meeting rooms, laboratories and a datacenter.
% This building was completed in 1950, thus its infrastructure is significantly different from the Japanese one.
The HVAC system in the building is centralized and serves several floors per unit.
There is a separate unit for an internal fabricated laboratory, inside the building.
%Nevertheless, we notice an independent HVAC system that was serving a particular laboratory; the Microfabrication Laboratory (Microlab).

\subsection{Sutardja Dai Hall - UC Berkeley}
Sutardja Dai Hall is a large building at Berkeley.  It is a 7-story, 141000 square-foot 
 building which contains classrooms, meeting rooms, laboratories, a nano-fabrication laborty, and a cafe.
% This building was completed in 1950, thus its infrastructure is significantly different from the Japanese one.
The HVAC system in the building is centralized and serves several floors per unit.
There is also a separate unit for an internal fabricated laboratory, inside the building.  It was built in 2009.

\subsection{Soda Hall - UC Berkeley}
Soda Hall is another building at Berkeley.  It is the main building for the computer science department.
It was built in 1994 and also has 7 floors and a centralized HVAC system.  It contains classrooms, 
offices, server rooms, and open office spaces for several labs.


\subsection{Engineering Building 2 - Todai}\label{data:engbldg2}
%The data from building 1 is collected at the Engineering Building 2 of the Hongo campus. 
Engineering building 2, at the University of Tokyo (Todai), is a 12-story building completed in 2005.  It contains
classrooms, laboratories, offices and server rooms.  
The electricity consumption of the lighting and HVAC systems of 231 rooms is monitored by 135 sensors.
Rather than a centralized HVAC system, small, local HVAC systems are set up throughout the buidling.  
The HVAC systems are classified into two categories, EHP (Electrical Heat Pump) and GHP (Gas Heat Pump).
The GHPs are the only devices that serve numerous rooms and multiple floors.  The 5 GHPs in the dataset serve 154 rooms.
The EHP and lighting systems serve only pairs of rooms and which are directly controlled by the occupants.
In addition, the sensor metadata provides device-type and location information (room number), 
therefore, the electricity consumption of each pair of rooms is separately monitored.

