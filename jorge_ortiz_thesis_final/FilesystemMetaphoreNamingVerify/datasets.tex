\section{Building Datasets}

In this thesis, we run all of our experiments and analysis on data from the four buildings.  They serve as 
the main sites for experiemental data collection and/or sites from which we obtained data dumps from their building
management system.

% In addition, we expect the anomaly detector to identify discontinuities in these relationships that represent obvious electricity saving opportunities (e.g. a room HVAC left on during night while the room lights have been turned off).

% Due to privacy concern this dataset is not publicly available on the Internet but accessible upon request.


\subsection{Engineering Building 2 - Todai}\label{data:engbldg2}
%The data from building 1 is collected at the Engineering Building 2 of the Hongo campus. 
Engineering building 2, at the University of Tokyo (Todai), is a 12-story building completed in 2005.  It contains
classrooms, laboratories, offices and server rooms.  
The electricity consumption of the lighting and HVAC systems of 231 rooms is monitored by 135 sensors.
Rather than a centralized HVAC system, small, local HVAC systems are set up throughout the buidling.  
The HVAC systems are classified into two categories, EHP (Electrical Heat Pump) and GHP (Gas Heat Pump).
The GHPs are the only devices that serve numerous rooms and multiple floors.  The 5 GHPs in the dataset serve 154 rooms.
The EHP and lighting systems serve only pairs of rooms and which are directly controlled by the occupants.
In addition, the sensor metadata provides device-type and location information (room number), 
therefore, the electricity consumption of each pair of rooms is separately monitored.

