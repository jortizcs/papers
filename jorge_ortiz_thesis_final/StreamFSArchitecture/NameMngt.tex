\section{Name Management}

StreamFS manages two namespaces.  The first is a flat namespaces that identifies a particular
object instance.  The second is a hiearchical namespace that identifies the current instance
of a particular object.  In this section, we discuss why we have two namespaces, how
they are managed, and how they are used by the query interface and analytical layer.

\subsection{Object identifier namespace}

Each new object that's created is assigned a 128-bit unique identifier that uniquely identifies the object.
Because StreamFS support multiple names that refer to the same object, we created a namespace that is 
flat and large enough to uniquely identify objects in the building uniquely.  The unique identifier
is randomly constructed and the probability of colissions is small because of the large namespace.
StreamFS only assigns a unique identifier to stream files.  We will discuss the various file types in
section~\ref{chap:naming}.

% \begin{itemize}
% \item 96 high-order bits identify the object
% \item 32 low-order bits identify the version of the object
% \end{itemize}

% In this paper we concentrate on the high-order bits used to identify the object.  Management of the low-order
% bits is the subject of related, ongoing work.

\subsection{Hierarchical Namespace}

Hierarchicaly naming schemes are an effective way to organize information, particularly for a relatively small amount of information
where the access patterns are well defined and groups across buildings have a lot of overlap.  For example, upon close examination of
the naming scheme for points across Sutardja Dai Hall, Soda Hall, Cory Hall, and the University of Tokyo Engineering Building 2, we 
observe that there are 2 overlapping group types.  All the point name refer to the location of the sensor and the system that it is 
associated with.  For example, `SODA1R430A\_ART' encodes the name of the building and the room number but also encodes the HVAC subsystem id --
referred to by the 5th character which is a `1'.  The other common encoding include the type of sensor and implies the S.I. units of measure.
Based on our experience with anlysis jobs on building sensor data, we decide it was less import from a naming persepctive than 
an interpretive one.

The number of sensors in the building can easily grow into the thousands or ten of thousands.  We apply the principals articulated 
in~\cite{hierarchy_is_dead}, which asserts that hierarchical organization of file is ineffective when dealing with a large number of files
and that databases are poor at providing direct access to the data, but provide a good way to find the information we are looking for.
We combine these two, as suggested by the authors.  We expose a hierarchical namespace that gives the user direct access to the data
through a familiar organization of that data.  The organization itself is directly traversable.  Moreover, we separate the metadata from
the naming structure, so that users looking for various kinds of information can quickly locate it.
We discuss our naming structure in more detail in Chapter~\ref{chap:naming} and give an overview of fundamental challenges that emerge when
naming must reflect physical associations and the physical environment is changing.

