\chapter{StreamFS System Architecture}

StreamFS is a system that we built to address some of the shortcomings in the current architecture.  StreamFS uses the filesystem
metaphore to represent all building information.  This include sensors, actuators, location, processing elements, 
streams, categorical organization, etc.  It borrow several mechanisms uses in a Unix-style filesystem, namely, files, folders,
and the pipe abstraction for processing streaming data.  It also borrows the notion that all interaction is through the filesystem.
This eases management of both the raw data and the processing elements that produce derivative streams for further processing.

In this section, we give an overview of the architecture -- all the components, their organization, and how they interact with one
another.  StreamFS consists of over 20,000 lines of code and was implemented in mostly Java.  It was deployed across multiple
buildings and several applications were built on top of it over a 2 years period.

\section{Name Management}
The name management layer addresses point \ref{nw} in section~\ref{sec:shortcomings}.  It provides a high-level
narrow waist for access sensors in context specified in the name itself.
StreamFS manages two namespaces.  The first is a flat namespaces that identifies a particular
object instance.  The second is a hiearchical namespace that identifies the current instance
of a particular object in some context, specified by the path for the object.  
We support two namespaces in order to uniquely identify sensors and actuators while supporting multiple names.

Multiple names is a requirement in building applications.  Sensors and actuators can be accessed in various 
way, depending on the application.  Some applications wish to access the sensor in the context of its placement 
in space.  For example, the path \texttt{/soda/4F/410/temp} can refer to a temperature sensor in 410 of a building
named `soda'.  The same temperature sensor drives the actuation profile for the HVAC system that serves room
410 and for other applications it may be important to access it in the system context through a name that specifies
that relationship, such as \texttt{/soda/hvac/ahu1/temp}.
In the rest of this section we discuss how both namespaces are managed and implemented.

\subsection{Object identifier namespace}

Each new object that's created is assigned a 128-bit unique identifier that uniquely identifies the object.
The namespace is large enough to support to many object with low probability of collisions, even across
StreamFS instances.
Because StreamFS support multiple names that refer to the same object, we created a namespace that is 
flat and large enough to uniquely identify objects in the building uniquely.  The unique identifier
is randomly constructed and the probability of colissions is small because of the large namespace.
StreamFS only assigns a unique identifier to stream files.  We will discuss the various file types in
section~\ref{chap:naming}.

% \begin{itemize}
% \item 96 high-order bits identify the object
% \item 32 low-order bits identify the version of the object
% \end{itemize}

% In this paper we concentrate on the high-order bits used to identify the object.  Management of the low-order
% bits is the subject of related, ongoing work.

\subsection{Hierarchical Namespace}

Hierarchicaly naming schemes are an effective way to organize information, particularly for a relatively small amount of information
where the access patterns are well defined and groups across buildings have a lot of overlap.  For example, upon close examination of
the naming scheme for points across Sutardja Dai Hall, Soda Hall, Cory Hall, and the University of Tokyo Engineering Building 2, we 
observe that there are 2 overlapping group types.  All the point name refer to the location of the sensor and the system that it is 
associated with.  For example, `SODA1R430A\_ART' encodes the name of the building and the room number but also encodes the HVAC subsystem id --
referred to by the 5th character which is a `1'.  The other common encoding include the type of sensor and implies the S.I. units of measure.
Based on our experience with anlysis jobs on building sensor data, we decide it was less import from a naming persepctive than 
an interpretive one.

The number of sensors in the building can easily grow into the thousands or ten of thousands.  We apply the principles articulated 
in~\cite{hierarchy_is_dead}, which asserts that hierarchical organization of files is ineffective when dealing with a large number of files
and that databases are poor at providing direct access to the data, but provide a good way to find the information we are looking for.
We combine these two, as suggested by the authors.  We expose a hierarchical namespace that gives the user direct access to the data
through a familiar organization of that data.  The organization itself is directly traversable.  Moreover, we separate the metadata from
the naming structure, so that users looking for various kinds of information can quickly locate it.

The decision to separate these also gives our implementation \emph{better scalability}.  The growth of the namespace, metadata, and data
happen at different rates.  We can tailor the acquisition of information about the files depending on the type of information
being requested.  For example, if the query is metadata-related, we send it to the metadata management cluster, which not only stores
the metadata, but also indexes it accordingly.  Furthmore, the namespace is easily extendable and provides a natural
way to group items which eases \emph{management and access}.

We discuss our naming structure in more detail in Chapter~\ref{chap:naming} and give an overview of fundamental challenges that emerge when
naming must reflect physical associations and the physical environment is changing.

\subsection{Implementation Details}

The name management layer is implemented behind HAProxy, an open source load balancer. The implementation includes 
a name registry and a name server.  Several name server handles requests that are forwarded
from HAPRoxy to one of the name servers.  Each of the name server knows of each of the databases that contains names. 
In all our deployments, we only had a single server.  However, for deployments that are large, we put the names in multiple, 
replicated databases with a write-through update policy.  Reads are done from any of the database servers, randomly, since
they all contain the same information.  Each name server has a preference, so the load is properly distributed.

\begin{figure}[h!] %htbp
\centering
\includegraphics[width=.55\columnwidth]{figs/name_reg}
\caption{Name management layer implemented behind HAProxy.  Name servers handle individual requests and use the name registration table
query handle the request accordingly.}
\label{fig:nameserver}
\end{figure}

The write-through policy is implemented with a write lock.  Whenever a name server receives a request to create or delete a file it informs the
other name servers that it wishes to acquire a lock.  To prevent deadlock, we force a lock-acquisition order.  A lock is not acquired unless every
name server agrees to give up the lock to the requesting name server.  Once the lock is acquired, the name server performs the same write on each
server.  The name server then releases the lock by contacting the other name servers in reverse order.  If a name server has given up a lock
and not received a release, the lock is released automatically after some time.  If a name server goes down, the name server that acquired the lock
assumes the release was successful.  The name server list is immutable, they are restarted in practice if they go down.

A layer of memcached~\cite{memcached} is used to reduce the load on the databases.  Writes immediately invalidate any entries in memcached.  
We also include file metadata in the memcached layer, so its use reduces the load on both the name register and the metadata data store.
The security manager essentially maintains an access control list and set of operations that are supported by each file.  By defualt, 
security is disabled, but some of our deployments did enable it.

The name management layer consists of 3 dependent components, each following the principles of horizontal \emph{scalability}.  The namespaces are
managed in single replicatable, relational database.  The metadata is managed in a separate MongoDB~\cite{mongodb} database, which is itself
shardable.  The data associated with streams is managed in a shardable timeseries database.  We follow the principle of scal-out
system in sub-component for \emph{scalability}.













\section{Entity-relationship model}

%In the vast majority of cases, in-time answers are imperative.  
Tracking the operational energy consumption of a building requires the ability answer a series of questions
about energy flow -- energy data aggregated across multiple logical classes to determine how, where, and how 
much is being used.  Sometimes, it even involves extrapolating forward in time to estimate 
future consumption patterns that could influence immedaite decisions.  The ability to \emph{slice and dice} 
the data allows the analyst
to gain better insight into how the energy is being used, where it can be used more effectively, and how
to change the operation of the building -- through better equipment or activity scheduling -- 
in order to optimize and reduce its energy consumption.
% For building-oriented energy analytics applications the building manager and occupants typically want answers
% to the following set of questions:
Below is a typical list of questions:

\begin{enumerate}
\item How much energy is consumed in this room/floor/building?  On average?
\item What is the current power draw by this pump? cooling tower? heating sub-system?  Over
		the last month?
\item How much power is this device currently drawing? Over the last hour?
%\item What percentage of the total building power is being drawn by the plug-load devices? 
\item How much energy have I consumed today?  Versus yesterday?
\item How much energy does the computing equipment in this building consume?
%\item {\bf For all queries:} What's the trend over time?
\end{enumerate}
\vspace{0.08in}

Notice, these question span spatial, temporal, and other arbitrary aggregates -- some physical, some
categorical.  There is also
an implicit hierarchical aspect to the grouping, in some cases.  For example, there are many
rooms on a floor and many floors in a building.  Naturally, to answer the first question we can
aggregate the data from the room up to the whole building.  This hierarchical relationship
is not as evident in the HVAC sub-components specified in the second question.  However,
local hierarchically relationships \emph{do exist}.  For example, the cooling system consists
of the set of pumps, cooling towers, and condensers in the HVAC system that push condensor
fluid and water to remove heat from spaces in the building.

We can model this as a set of objects and inter-relationships which inform how
to \emph{drill-down}, \emph{roll-up}, and \emph{slice and dice} the data -- traditional OLAP operations.
The main difference between this setup and traditional OLAP is the underlying dynamics of the
inter-relationships: objects, particularly those meant to represent physical entities, are added and removed and 
their inter-relationships change over time.  \emph{The natural evolution of buildings and activities 
within them makes tracking energy-flow fundamentally challenging}.

In this paper, we show how the entity-relationship model~\cite{Chen76theentity-relationship} helps simplify 
this problem, both as an interface
to the user and a data structure for the aggregation processes.  We argue that the use of this model is a cleaner
fit for this application scenario because it captures important semantic information about the real-world;
facts critical for picking which questions to ask and how to answer them.  In contrast, it has been shown 
that a relational model loses this information~\cite{SenkoDB}.


Lets examine the requirements for answering the first question.
A building is unware that there are rooms. Typically spaces in a building are called \emph{zones} and, 
at construction time, walls are added to make rooms within zones.  This makes rooms an abstract
entity, used to group associated items with respect to it.  It also means
%The basic control unit for the Heating, Ventilation, and Air conditioning (HVAC) system as well as the electrical 
%panels and plugs, is a zone, not a room.  
we typically do not have a single meter that is measuring the energy of a room; it
must be calculated from the set of energy-consuming constituents.

What are the energy consuming constituents of a typical room?  It is the set of energy-consumers that
are active within or onto the room.  Broadly, it consists of three things:

\begin{itemize}
\item Plug-loads
\item Lights
\item HVAC
\end{itemize}
\vspace{0.08in}

For simplicity of demonstration, lets consider only plug-loads.  In our construction of an entity-relationship
graph lets assume there are nodes for each plug-load item and each room.  For the room in question, the relationship
between the plug-loads and the room is child to parent, respectively.  The total energy consumed by
the plug-loads can be aggregated at the parent node, the room, so the user can query the room for
the total.  Over time, plug-loads are removed and added to/from the room, but the relationship does not
change.  This simplifies the query; to obtain the total consumption over time, the query need only
go to the room node.  The parent-child relationship informs which constituents to aggregate over time
to calculate the total.

To realize this design we need to maintain the entity-relationship graph, present it to the user in a meaningful
way; allowing them to update it directly to capture physical state and relationship changes.  We also need to
use this graphical structure to direct data flow throughout the underlying network.  This allows us
to accurately maintain the running aggregates as the deployment and activities churn.

We present the graph to the user through a filesystem-like naming and linking mechanisms.  The combination of a
hierarhical naming scheme and support for symbolic links allows the user to access and manipualte underlying objects
and relationships.  Moreover, the underlying graph structure is overloaded with upstream communication mechanisms
and buffering to allow data to flow from the data-producing leave nodes to the aggregation-performing
parent nodes.  Furthermore, the buffering lets us deal with the streaming nature of data flow from the physical
world to StreamFS and lets us maintain a real-time view of energy flow in the system.
Traversing the graph provides a natural way for the user to implicitly execute the OLAP operations necessary 
to give the user the kind of insight into energy usage in the building necessary to understand, optimize and 
reduce it.
\section{Time-series Data Store}

The timeseries data store addresses point \ref{ts} in section~\ref{sec:shortcomings}.
Data collected from sensor is timeseries in nature.  A sensor produces data periodically.  The important aspect of
the stream are the name of the feed, the time the reading was received, and the value for that reading.  There is also
metadata that needs to be stored about the stream.  For example, we want to know what the units of measure are, 
the make/model of the sensor, the date it was installed, any calibration parameters or other information that will help 
the user locate the sensor or interpret it correct.  We actively separate the storage of the metadata from the storage 
of the data.

In constructing a design for the data store, we considered 3 main questions:

\begin{enumerate}
\item What is the typical access pattern or what are the top queries?
\item Should we compress it?
\item How is the data stored long-term?
\end{enumerate}

The typically access pattern is that of scans.  Many of the applications that we consider that make use of historical data, fetch the data
is a temporally meaningful manner.  The query specifies the interval of time over which to fetch the data from a particular feed
and either perform cleaning operations on the data, display it, or adjust the scan parameters for a subsequent query.
The data is largely self-similar and highly compressible.  Simple compression tests we ran on real data showed a compression factor 
between 15 and 30.  Also, the data is essentially append-only, forever.  It can grow quite large, but grows have a fairly 
slow rate, especially after compression.  For example, the total footprint of the SDH deployment, uncompressed 
is nearly 100 GB, however, after compression it is only about 4 GB.
All timeseries data is stored as a 3-tuple that included the name of the stream, a timestamp, and value.  The name we use in
the datastore is the unique id that is generated by StreamFS.  %The human-readable name is fetched

\subsection{Implementation Details}

\begin{figure}[h!] %htbp
\centering
\includegraphics[width=.55\columnwidth]{figs/tsdstore}
\caption{The timeseries data store.  We use OpenTSDB; a timeseries data-store that runs in a cluster setting over
HBase.}
\label{fig:tsdb}
\end{figure}

We use OpenTSDB~\cite{opentsdb} as our primary data store. We enable compression  and index on the name and timestamp 
of the feed.  OpenTSDB is a timeseries data store built on HBase~\cite{HBase}.  HBase is designed to scale horizontally for very
large data sets.  OpenTSDB is a good choice because the compression features keep the footprint small/fast while the append-only 
workload requires a scalable solution.

\section{Publish-Subscribe Subsystem}

StreamFS uses a flexible construction of the publish/subscribe model in order to support a wide range of applications.  
Publish/subscribe is necessary is physical data application development in order to scale in the number of supported
applications.  The publish/subscribe model used in StreamFS provides mechanisms that enable a flexible combination 
of space and time decoupling that enable StreamFS to support of a wide arrange of application requirements, as described by
Eugster et al.~\cite{eugster}..

Our pub/sub engine is also tightly coupled with the namespaces expose to users, and this design choice allows an application
to control the space coupling between the publisher and the subscriber (similar to TIBCO~\cite{tibco}).

\subsection{Space decoupling}
By its very design, space decoupling is achieved.  Publisher do not hold a reference to the subscriber and subscribers do not
hold references to publisher.  However, because of the coupling of a full pathname and an object, subscriptions to topics
expressed as a full pathname refer to the single publisher.

\subsection{Time decoupling}
Time decoupling is achievable through the timeseries data store.  Publisher push data to StreamFS whether or not subscribers are
online.  Moreover, data may be received at the subscriber even if the publisher becomes disconnected.  Currently, subscribers do
not receive all information that was missed.  In order to achieve fill time-decoupling, we allow the subcription
target to enable or disable the option to buffer all missed readings for an associated subscription target, while the subscription
target it offline.

\subsection{Synchronization decoupling}
Sychronization decoupling is achieved by the publisher and subscribers through StreamFS.  Events are received out of sequence
from their arrival to StreamFS.  This is true even when the subscription target is a processing element.  The thread that buffers
incoming data for each processing element is seperate from the thread where the process is executed.

\section{Data Cleaning and Real-time Processing}

StreamFS provides sophisticated mechanisms to process data in real time.  Processing features in StreamFS address
point \ref{proc} in section~\ref{sec:shortcomings}.
Sensor data is fundamentally challenging to deal with because much of it must be cleaned before it can be processed.  For example,
it is not uncommon to receive readings that is out of operational range, that is erroneous with respect to the previous observed trend,
or to stop receiving readings altogether.  This implies the need for processing jobs to provide a level of filtering over the raw streams.
Once the data is cleaned, it is typically consumed by more sophisticated processes that aggregate or use it for control
of equipment.  We provide the mechanisms for handling both classes of processing jobs with our process management layer.
% In the next section we will discuss our process management layer and how users can both submit jobs to StreamFS for management or link
% their own external processing elements so that they can be managed through StreamFS but run outside of StreamFS.
We address \emph{re-sampling} and \emph{processing models}.  The incoming data does not have a common
time source, so combining the signals meaningfully involves interpolation.  There are various options that we
provide for performing the interpolation, chosen by the user depending on the units of the data.  For example,
temperature data may involve fitting a heat model with the data to attain missing values in time.  

For jobs are need to clean the data and wish to run short-lived, simple operations, we provide an interface for
\emph{internal} processing.  The user submits a job and we schedule it in a machine in the processing cluster.
For jobs that are more complex and require client-side lirbaries, we offer a facility where the process is allowed
to run on the client side, but is entirely managed by StreamFS.  We provide a client stub that essentially runs like a
mini-job scheduler on the client side and communicate with StreamFS to execute file operations that affect locally-running
jobs.  We discuss the details for both kinds of jobs in section~\ref{sec:internalprocs} and \ref{sec:externalprocs}.

% Aggregation is done as a function of the underlying constituents: they can be combined arbritarily, by adding
% subtracting, multiplying or dividing corresponding values.  We provide an interface to the user that
% allows them to specify how to combine the aggregate signals as a function of the child nodes in the entity-graph.
% Futhermore, they can filter the data by unit.  This kind of flexibility useful for visualizing
% energy consumption over time.

Finally, since data is coming in at different rates from different sensors and is produced asynchronously from processing elements.
For certain processes, processing the incoming data as quickly as possible is key, however, this is challenging for several reasons:
1) a process may subscribe to multiple, independent streams with asychronized report schedules and 2) interpolated values
should be avoided to minimize prediction inaccuracies in interpolated values.  Therefore, a process actually wants all the freshest
data from all the streams they are subscribing to, while minimizing the average time that the data for each respective stream has 
been waiting in the buffer.  We address these problem through a freshness scheduler that is presented in section~\ref{sec:freshness}.



\subsection{Implementation Details}

\begin{figure}[h!] %htbp
\centering
\includegraphics[width=.55\columnwidth]{figs/procmngr}
\caption{The process manager manages a cluster of processing element and connection to external processing units.  It works
closely with the subscription manager to forward data between elements.}
\label{fig:procmngr}
\end{figure}


The process manager works closely with the name register to manage process definition files and process instances.  Process definition files
are those that are submitted to the server by the user, that define a function to run on the streaming data.  Once data is piped to the process
definition file, the process manager spawns and instance of the definition file on one of the process-element (PE) execution servers.  The PE
creates a buffer for incoming data and sets up a job to run periodically according to the specification for the internal processing job.
The internal processing job is mapped to a file that is accessible in StreamFS.  It contains various statistics about the job that is running, such
as the streams that feed it, the last time it ran, the amount of time it took to run, the period of execution, etc.
If the user deletes the file, the process manager contact the corresponding PE server that contains the job and the job is killed.  Once the job
is killed it informs the process manager which informs the name register to remove the file.

For jobs that are more complex and need to run externally, we created an client-stub that runs like a mini-PE.  It spawns a job on the client
side when a user pipes data into it.  It manages all instances of running jobs on the client server and it processes requests associated with
operations on the corresponding instance file represented in StreamFS.  Figure~\ref{fig:procmngr} shows the components of the StreamFS architecture
that handles all processing elements.

















\begin{figure}[t!] %htbp
\centering
\includegraphics[width=0.75\columnwidth]{figs/sfsarch}
\caption{StreamFS system architecture.}
\label{fig:sfsarch}
\end{figure}

\section{Related work}

\begin{itemize}
\item dashboard
\item andrew's lightin control work
\item Kamin's hvac control work
\item BEMs
\item sMAP stuff
\item Buildsys 2010 work~\cite{hbci}
\end{itemize}
% \section{Summary}

% In this chapter 
We described the details and motivation in the process management and related components.  We introduced the 
notion of internal and external processing.  The former is used for small, simple data-cleaning jobs while the latter
is for integrating external processing jobs written in the client's native language.
We also showed how we combine the entity-relationship graph to provide the infrastructure necessary to support OLAP-style queries.
This is an important features, since many of the queries posed in the building domain have the following properties:

\begin{enumerate}
\item Temporally-driven, scan-heavy queries.
\item Hierarchical, unit-specific aggregates.
\end{enumerate}

Dynamic aggregation is an efficient design for these kinds of queries.  Unlike traditional OLAP, where the timestamps
are uniform across other dimensions, we must interpolate the values to keep the ``OLAP cube'' populated with data at all
intervals.  It is also necessary to provide accurate aggregates in time.

Finally, we articulate our observation of the importance of scheduling with jobs that want a set of readings that are collectively
the latest -- the collective buffer freshness is maximized.  We formalize the problem and present an algorithm solution and evaluation.
In the next chapter we discuss the files and associated semantics in StreamFS.  We show can they related to traditional filesystems
and discuss the motivation for its design.  We also present the mathematical tools for verifying the relationships between sensors that
is constructed through the namespace.


