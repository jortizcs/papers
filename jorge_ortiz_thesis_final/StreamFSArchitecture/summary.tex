\section{Summary}

% StreamFS consists of over 20,000 lines of code and was implemented in mostly Java.  It was deployed across multiple
% buildings and several applications were built on top of it over a 2 years period.

In this chapter we gave an overview of the main components in StreamFS.  Each of the components addresses the concerns stated in 
section~\ref{sec:shortcomings}.  The filesystem name server expose a uniform namespace for access sensors and actuators in 
deployed throughout the building.  The timeseries database serve to store data streaming physical information and 
is optimized for the scan-style queries posed by applications.  These address points \ref{nw} and \ref{ts}.
We also include a pub-sub system which serves multiple purposes.  It provides real-time data forwarding for external
applications and forwards data internally to processing units that are specified or linked by the user.
This addresses points \ref{rt}.  Finally, we introduce processing elements, both internal and external to address
point \ref{proc}.  We also introduce an entity-relationship graph to deal with indirect relationships that are
expressed in the construction of names in the system.

In the next chapter we talk more about processing and discuss the details in the scheduler that help enable applications
that have certain delivery requirement.
