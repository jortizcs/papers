\section{Publish-Subscribe Subsystem}

StreamFS uses a flexible construction of the publish/subscribe model in order to support a wide range of applications.  
Publish/subscribe is necessary is physical data application development in order to scale in the number of supported
applications.  The publish/subscribe model used in StreamFS provides mechanisms that enable a flexible combination 
of space and time decoupling that enable StreamFS to support of a wide arrange of application requirements, as described by
Eugster et al.~\cite{eugster}..

Our pub/sub engine is also tightly coupled with the namespaces expose to users, and this design choice allows an application
to control the space coupling between the publisher and the subscriber (similar to TIBCO~\cite{tibco}).

\subsection{Space decoupling}
By its very design, space decoupling is achieved.  Publisher do not hold a reference to the subscriber and subscribers do not
hold references to publisher.  However, because of the coupling of a full pathname and an object, subscriptions to topics
expressed as a full pathname refer to the single publisher.

\subsection{Time decoupling}
Time decoupling is achievable through the timeseries data store.  Publisher push data to StreamFS whether or not subscribers are
online.  Moreover, data may be received at the subscriber even if the publisher becomes disconnected.  Currently, subscribers do
not receive all information that was missed.  In order to achieve fill time-decoupling, we allow the subcription
target to enable or disable the option to buffer all missed readings for an associated subscription target, while the subscription
target it offline.

\subsection{Synchronization decoupling}
Sychronization decoupling is achieved by the publisher and subscribers through StreamFS.  Events are received out of sequence
from their arrival to StreamFS.  This is true even when the subscription target is a processing element.  The thread that buffers
incoming data for each processing element is seperate from the thread where the process is executed.
