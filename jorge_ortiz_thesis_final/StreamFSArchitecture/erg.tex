\section{Entity-relationship model}

%In the vast majority of cases, in-time answers are imperative.  
Tracking the operational energy consumption of a building requires the ability answer a series of questions
about energy flow -- energy data aggregated across multiple logical classes to determine how, where, and how 
much is being used.  Sometimes, it even involves extrapolating forward in time to estimate 
future consumption patterns that could influence immedaite decisions.  The ability to \emph{slice and dice} 
the data allows the analyst
to gain better insight into how the energy is being used, where it can be used more effectively, and how
to change the operation of the building -- through better equipment or activity scheduling -- 
in order to optimize and reduce its energy consumption.
% For building-oriented energy analytics applications the building manager and occupants typically want answers
% to the following set of questions:
Below is a typical list of questions:

\begin{enumerate}
\item How much energy is consumed in this room/floor/building?  On average?
\item What is the current power draw by this pump? cooling tower? heating sub-system?  Over
		the last month?
\item How much power is this device currently drawing? Over the last hour?
%\item What percentage of the total building power is being drawn by the plug-load devices? 
\item How much energy have I consumed today?  Versus yesterday?
\item How much energy does the computing equipment in this building consume?
%\item {\bf For all queries:} What's the trend over time?
\end{enumerate}
\vspace{0.08in}

Notice, these question span spatial, temporal, and other arbitrary aggregates -- some physical, some
categorical.  There is also
an implicit hierarchical aspect to the grouping, in some cases.  For example, there are many
rooms on a floor and many floors in a building.  Naturally, to answer the first question we can
aggregate the data from the room up to the whole building.  This hierarchical relationship
is not as evident in the HVAC sub-components specified in the second question.  However,
local hierarchically relationships \emph{do exist}.  For example, the cooling system consists
of the set of pumps, cooling towers, and condensers in the HVAC system that push condensor
fluid and water to remove heat from spaces in the building.

We can model this as a set of objects and inter-relationships which inform how
to \emph{drill-down}, \emph{roll-up}, and \emph{slice and dice} the data -- traditional OLAP operations.
The main difference between this setup and traditional OLAP is the underlying dynamics of the
inter-relationships: objects, particularly those meant to represent physical entities, are added and removed and 
their inter-relationships change over time.  \emph{The natural evolution of buildings and activities 
within them makes tracking energy-flow fundamentally challenging}.

In this paper, we show how the entity-relationship model~\cite{Chen76theentity-relationship} helps simplify 
this problem, both as an interface
to the user and a data structure for the aggregation processes.  We argue that the use of this model is a cleaner
fit for this application scenario because it captures important semantic information about the real-world;
facts critical for picking which questions to ask and how to answer them.  In contrast, it has been shown 
that a relational model loses this information~\cite{SenkoDB}.


Lets examine the requirements for answering the first question.
A building is unware that there are rooms. Typically spaces in a building are called \emph{zones} and, 
at construction time, walls are added to make rooms within zones.  This makes rooms an abstract
entity, used to group associated items with respect to it.  It also means
%The basic control unit for the Heating, Ventilation, and Air conditioning (HVAC) system as well as the electrical 
%panels and plugs, is a zone, not a room.  
we typically do not have a single meter that is measuring the energy of a room; it
must be calculated from the set of energy-consuming constituents.

What are the energy consuming constituents of a typical room?  It is the set of energy-consumers that
are active within or onto the room.  Broadly, it consists of three things:

\begin{itemize}
\item Plug-loads
\item Lights
\item HVAC
\end{itemize}
\vspace{0.08in}

For simplicity of demonstration, lets consider only plug-loads.  In our construction of an entity-relationship
graph lets assume there are nodes for each plug-load item and each room.  For the room in question, the relationship
between the plug-loads and the room is child to parent, respectively.  The total energy consumed by
the plug-loads can be aggregated at the parent node, the room, so the user can query the room for
the total.  Over time, plug-loads are removed and added to/from the room, but the relationship does not
change.  This simplifies the query; to obtain the total consumption over time, the query need only
go to the room node.  The parent-child relationship informs which constituents to aggregate over time
to calculate the total.

To realize this design we need to maintain the entity-relationship graph, present it to the user in a meaningful
way; allowing them to update it directly to capture physical state and relationship changes.  We also need to
use this graphical structure to direct data flow throughout the underlying network.  This allows us
to accurately maintain the running aggregates as the deployment and activities churn.

We present the graph to the user through a filesystem-like naming and linking mechanisms.  The combination of a
hierarhical naming scheme and support for symbolic links allows the user to access and manipualte underlying objects
and relationships.  Moreover, the underlying graph structure is overloaded with upstream communication mechanisms
and buffering to allow data to flow from the data-producing leave nodes to the aggregation-performing
parent nodes.  Furthermore, the buffering lets us deal with the streaming nature of data flow from the physical
world to StreamFS and lets us maintain a real-time view of energy flow in the system.
Traversing the graph provides a natural way for the user to implicitly execute the OLAP operations necessary 
to give the user the kind of insight into energy usage in the building necessary to understand, optimize and 
reduce it.