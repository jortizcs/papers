\section{Time-series Data Store}


The data collected from sensor in timeseries in nature.  A sensor produces data periodically.  The imporant aspect of
the stream are the name of the feed, the time the reading was received, and the value for that reading.  There is also
metadata that needs to be stored about the stream.  For example, we want to know what the units of measure are, 
the make/model of the sensor, the date it was installed, any calibration parameters or other information that will help 
the user locate the sensor or interpret it correct.  We actively separate the storage of the metadata from the storage 
of the data.

In constructing a design for the data store, we considered 3 main questions:

\begin{enumerate}
\item What is the typical access pattern or what are the top queries?
\item Should we compress it?
\item How is the data stored long-term?
\end{enumerate}

The typically access pattern is that of scans.  Many of the applications that we consider that make use of historical data, fetch the data
is a temporally meaningful manner.  The query specifies the interval of time over which to fetch the data from a particular feed
and either perform cleaning operations on the data, display it, or adjust the scan parameters for a subsequent query.
The data is largely self-similar and highly compressable.  Simple compression tests we ran on real data showed a compression factor 
between 15 and 30.  Also, the data is essentially append-only, forever.  It can grow quite large, but grows have a fairly 
slow rate, especially after compression.  In either case, we decided a good option would be to use OpenTSDB~\cite{opentsdb}.
We enabled the compression feature and indexed based on the same and timestamps of the feed.

