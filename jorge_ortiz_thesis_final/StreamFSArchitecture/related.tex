\section{Related Work}
StreamFS borrows ideas from many places.  Naturally, in adopting filesystem as the main abstraction we borrow directly
from the design of Unix-style filesystem and pipes~\cite{Ritchie74theunix}.  We translate these ideas into 
our implementation, transforming these into a web-services architecture combined with multiple databases.
Our filesystem is also hierarchical, but it is \emph{completely divorced from the approach to storage}.  File information is
spread across multiple databases in StreamFS.  The name is made accessible through a RESTful interface or socket connection, but
only the name and its structure is used to identify the object.  Moreover, the notion of symbolic and hard links are
used as a way to support multiple names for the name unique object.  In traditional filesystems, these inform
the access pattern to the file on disk.  Our filesystem API is \emph{not} POSIX compliant, since we introduce new file types
and associated file-specific semantics.  We discuss these in detail in the next chapter.

StreamFS constructs an entity-relationship graph~\cite{Chen76theentity-relationship} from the files in the system, in order to 
represent different physical and semantics relationships.  We use it throughout the platform for analytical processing and querying.
We couple the namespace that is exposed through the filesystem with an in-memory graph that graphically represents the inter-relationships
between the files.  This graph is used to provide OLAP-style queries, similar to the work by Chen et al.~\cite{Chen2008_olapgraph}.  
These authors describe how they translate OLAP operations to a graph.  Their decomposition is different in that nodes in the graph 
explicitly represent measures and dimensions.  We do represent the time dimension as a separate snapshot of the graph at another
point in time.  By separating the raw data from the graphical representation this becomes a trivial transformation.

The use of a filesystem representation has been explored by researchers.  Tilak et al.~\cite{BingFS} use the filesystem 
interface as a way to represent different naming conventions for a deployment of sensors.	StreamFS in similar in that it also 
adopts the construct for naming purposes.  StreamFS also adopts it as a way to expose a uniform access layer for sensors and
actuators in the world.  Actuator, in many ways, are also modeled like a periphal device.
However, StreamFS is more general and adopts more filesystem features and principals.
They do not support symbolic links and they couple the network organization of a wireless sensor network with the naming structure.
Filipponi el al.~\cite{wsnfuse} create a mountable filesystem that is POSIX compliant, making the deployment nodes
look like a peripheral device that can be directly written to.  In contrast to both pieces of work, we do not attempt
to make a POSIX compliant filesystem.  We adopt \emph{conceptual} filesystem construct and re-interpret them as features in 
a web architecture.  We tightly link the notion of a sensor and its data through the special file types and operations to 
handle sensor data queries and organization.
% OLAP + graph~\cite{Chen2008_olapgraph}

% Entity-relationship graphs~\cite{Chen76theentity-relationship}

% Filesystems~\cite{FastFS},Unix OS, which introduces filesystem naming and pipes~\cite{Ritchie74theunix}

% File system for sensor deployments from Binghamton\cite{BingFS}, and WSNFuse by Filipponi et al.~\cite{wsnfuse}.

The foundation of our processing framework is a pub/sub system architecture, similar 
to~\cite{Eugster01contentbasedpubsub, Rosenblum97adesign, tspaces, tibco}.  StreamFS has a flexible pub/sub coupling architecture,
where user options can separate certain dependencies between producers and consumers.  We also make use of the ERG to 
do topic-matching.  This is crucial for building applications, where analysis is tightly coupled with context.
In addition, our processing framework revisits many issues related to dataflow processing systems~\cite{CullerCSD92716,ptolemy2001,ptolemy2007}.
StreamFS is focused on providing a tool that cleans the incoming data and provides real-time data to consumers/applications
for the building.  It is not focused on true-real time constraints or modeling.
% Pub/sub~\cite{Eugster01contentbasedpubsub, Rosenblum97adesign, tspaces, tibco}

% Ptolemy project and dataflow graphs for cyber-physical systems~\cite{ptolemy2001,ptolemy2007}, 
% graphical programming of CPS~\cite{Cheong2005,CheongEECS200615}



% HomeOs~\ref{homeos} is related.  So is BOSS~\cite{boss} and FIAP~\cite{fiap} and BuildingDepot~\cite{bdepot}.







