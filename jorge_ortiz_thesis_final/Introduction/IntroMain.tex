\chapter{Introduction}

\section{Physical and Pervasive Computing}
\section{Computing As a Service}
\section{The Built Environment}
Buildings consume 40\% of the energy produced in the United States and nearly three quarters of 
the electricity produced~\cite{epabuildings}.  It is conjectured that as much as 30-80\% of their energy is 
wasted~\cite{waste_science, next10_waste}.  With such astounding figures, buildings are a clear optimization
target.  In order to optimize building performance, we need fine grained visibility into the energy flow
throughout the building.  With over 70\% of commercial buildings, 100,000 square feet or 
larger, having a building management system~\cite{cbecs2003}, the infrastructure to attain that level of 
visibility is available and widespread.  

\section{Pervasive Computing Applications in Buildings}
\section{Research Statement And Hypothesis}
% \emph{What are the architectural components and technical challenges in the design of an information system
% that enables new and supports old classes of applications in the built environment?}  
\emph{How can we incorporate filesystem and database constructs and what are the technical challenges in a system that supports applications
in the built environment?}
Given the emerging applications in
the built environment it is clear that the old information system design is not sufficiently open, flexibile, nor
scalable enough to support them.  Old information systems are tightly integrated from the field-level sensor to
the central supervisor control system.  There are two integration points in traditional systems that we argue 
are either fundamentally flawed or insufficient for emerging applications.  We describe the components that 
currently exist and identify those that are missing.  We show how these components/services enable emerging applications.  We also
discuss the technical challenges that must be solved in order to provide the correct semantics for these services.
Furthermore, we discuss a component that is fundamental for providing correct information to applications 
and formalize the notion of verification in the context of the built environment.  We provide several algorithmic 
solutions to these problems, which lay the foundation for a fundamental service in the broader architecture.

\section{Thesis Roadmap}
It start here and ends there.