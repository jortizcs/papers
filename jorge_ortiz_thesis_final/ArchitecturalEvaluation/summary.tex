\section{Summary}

\begin{table}
\begin{center}
  \begin{tabular}{| r | l | }
    \hline
    \textbf{No. deployments}    &   7                       \\ \hline
    \textbf{Total files}        &   $>$ 10k                 \\ \hline
    \textbf{Locations}          &   University of Tokyo,    \\
                                &   UCB Cory, SDH,          \\
                                &   Stanford Y2E2,          \\
                                &   Intel, Nokia,           \\
                                &   Samsung                 \\ \hline
    \textbf{Total data}         &   ~1TB over 2 years       \\  \hline
  \end{tabular}
\caption{This table summarizes the deployment statistics of for StreamFS over a two years.}
\label{tab:sfs_depstats}
\end{center}
\end{table}

We demonstrate the versatility of StreamFS in its ability to serve a diverse set of applications.
This clearly demonstrates its \emph{generalizability}
and \emph{ease of management} through the file system abstraction. % and data-processing ``pipe'' abstraction.
%Energy Lens
In the Energy Lens application we examined three system challenges -- mobility, consistency management, and disconnected operation
 -- for enabling and energy 
analytics applications in buildings.  StreamFS plays a crucial role in providing the kinds of services necessary 
for collecting and managing deployment data and metadata as well as providing live aggregate statistics on the deployment
to the end-user.  Furthermore, we demonstrated the \emph{extensibility} of the system, as all the StreamFS
features were agile enough to deal with the evolving dynamics of the deployment and the services built to support
fundamental challenges with consistency management and disconnected operation.

% However, other challenges remain, particularly those related to scaling to an entire buildings, integrating
% many more streaming data sources, and providing streaming analytics for immediate display to building occupants.
% We see an opportunity to combine these with control in order to empower building occupants to literally take
% control of their energy footprint.  The components of our architecture are simple, and simplicity is important for scale and
% generalizability.  We hope that with the right tools and information, people will be motivated to act, and large
% energy waste reduction can be achieved.

Altogether, StreamFS was deployed across 7 buildings in 2 countries, with very different climates, systems, and 
usage patterns.  Table~\ref{tab:sfs_depstats} summarizes these.

Our largest deployment had $>$ 7k feeds simultaneously feeds a single deployment.  Several hundred derivative streams were generated
in these deployments as well.  Several of these required a cluster of processing elements and datastore elements.  Typically, no larger
than 2 machines per component.  This demonstrates the \emph{scalability} of StreamFS in practice.

