\section{API Overview}

In this section we give an overview of the application programming interface.  We provide a description of
function calls and the corresponding HTTP/REST version.  A full tutorial of the HTTP/REST interface can be found 
in Appendix~\ref{appendix:tutorial}.

\begin{table}[h]
\begin{center}
\begin{tabular}{| r | l |}
	\hline
	\textbf{function} & \textbf{description} \\ \hline
	create & Create a file. Overlaps with several other calls.    \\ \hline

	delete & Delete a file. \\ hline

	register & Join as data source.    \\ \hline

	push & Write to associated stream file.  \\ \hline

	label & Add an attribute-value pair.  \\ \hline

	rlabel & Remove an attribute-value pair.  \\ \hline
\end{tabular}
\caption{Overview of StreamFS file-related API calls.  Library written in Java, PHP, and C.}
\label{tab:api_calls1}
\end{center}
\end{table}

Table~\ref{tab:api_calls1} gives and overview of API calls that are available through the standard library, available in several
languages.  It provides functions for creating and deleting files as well as decorating them with attribute-value pairs.
The Energy Lens application makes extensive use of this library interface for dealing with consistency-management
features and ``atomic'' update operations.  Applications that run on the phone, typically use the HTTP/REST
interface.


\begin{table}[h]
\begin{center}
\begin{tabular}{| r | l |}
	\hline
	\textbf{callback} & \textbf{description} \\ \hline
	search & General search of terms through file name and content.  Similar to grep.    \\ \hline

	filter & Used to filter the returned list by attribute-value pair or path.    \\ \hline

	query & Timeseries query.    \\ \hline

\end{tabular}
\caption{Summary of control interface callbacks in StreamFS.  Library written in Java, PHP, and C.}
\label{tab:api_calls2}
\end{center}
\end{table}

The search API is simple (summarized in Table~\ref{tab:api_calls2}).  It includes three main calls, depending 
on what kind of data you are searching for.  
If the application wishes to make a general search based on the name of the file or the metadata it is decorated with, 
the \texttt{search} call allows you to specify 
a set of terms to look for or to specify specific attribute-value pairs.  It performs the search as a \emph{filter-forward}.
It uses provides a general \texttt{path} sub-option where you can specify that the path match a regular expression.
For example, \texttt{search(type:plug-load).path(/410/*)} returns all the files that contain the ``type:plug-load''
attribute-value pair in their metadata and \texttt{/410/} as a prefix for any of their names.  This example 
AND's the search criteria. 
The filter call can be combined with the search call to filter on metadata values returned from the \texttt{path} search.
For example, \\ \texttt{search(type:plug-load).path(/410/*).filter(owner:jorge)} 
fetches the same files as before but filters those that contain the attribute-value pair ``owner:jorge''.
Finally, the \texttt{query} call runs a timeseries query on a specified path or all stream files
that match a regular expression specified by through the \texttt{path} option.

The pub-sub library is used extensively in all our applications.  It is specifically used in 
the Energy Lens for initiating aggregation processing.  Table~\ref{tab:api_calls3} summarizes
the two main API calls for dealing with subscriptions and pipes.


\begin{table}[h]
\begin{center}
\begin{tabular}{| r | l |}
	\hline
	\textbf{callback} & \textbf{description} \\ \hline
	pipe & Pipes the list of specified streams to \\
		 & either an active process or a process \\
		 & definition file.    \\ \hline

	subscribe & Create an subscription from a 		 \\
			  & set of streams to an external target \\
			  & through a callback function. 		 \\ \hline

\end{tabular}
\caption{Summary of control interface callbacks in StreamFS.  Library written in Java, PHP, and C.}
\label{tab:api_calls3}
\end{center}
\end{table}

Internally, the mechanism for dealing with both is similar, however, the semantics of each call is differe.t % two calls implies that
Pipes direct streams to user-defined internal/external processing elements, while subscriptions are
for external sinks running on the client.  Subscriptions are managed through an \texttt{HTTP POST}
operation.  If the user is using a client stub, the \texttt{POST} is unmarshalled and a callback
is triggered with the body of the \texttt{POST} submission.

% \begin{table}[h]
% \begin{center}
% \begin{tabular}{| r | l |}
% 	\hline
% 	\textbf{callback} & \textbf{description} \\ \hline
% 	ctrlRcv & Function that handles the reception of control data.    \\ \hline

% 	remove & Informs device of removal.    \\ \hline

% 	suspend & Tells device to stop sending data.  \\ \hline

% 	resume & Tells device to resume sending data.  \\ \hline
% \end{tabular}
% \caption{Summary of control interface callbacks in StreamFS.  Library written in Java, PHP, and C.}
% \label{tab:api_calls2}
% \end{center}
% \end{table}

% Full RESTful tutorial is in the Appendix, 