


\subsection{Methodology}\label{method}
% Fundamentally, the readings are driven by the same underlying phenomena: 
% weather and occupancy.  Weather influences \emph{all} the data similarly.  Occupancy, however, changes
% throughout the building and should be used as a differentiating component in the trace
% comparisons.  Sensors that share spatio-temporal elements should be correlated after the removal
% of the underlying trend driven by the weather.  In order to find unique relationships we need to remove 
% this common trend.

% \subsection{Empirical Mode Decomposition}
Empirical Mode Decomposition (EMD) \cite{huang:emd1998} is a new technique used for detrending data.
Specifically, EMD detrends non-stationary, non-linear timeseries data.  
% A trend is defined as 
% an intrinsically determined monotonic function within a certain temporal span or a function in which there 
% can be at most one extremum within that temporal span.  
A non-stationary signal is a signal whose mean and
variance change over time.  EMD is a process, not a theoretical tool, and its main use is for removing trends 
to enable more useful spectral analysis.

We describe the EMD process as follows:  for a signal \emph{X(t)}, let $m_1$ be the mean of its upper and
lower envelopes as determined from a cubic-spline interpolation of local maxima and minima. The locality 
is determined by an arbitrary parameter.

\begin{enumerate}
\item The first component $h_1$ is computed: $h_1=X(t)-m_1$
\item In the second sifting process, $h_1$ is treated as the data, and $m_{11}$ is the mean of $h_1$'s upper and lower envelopes: $h_{11}=h_1-m_{11}$
\item The procedure is repeated $k$ times, until $h_{1k}$ is a function: $h_{1(k-1)}-m_{1k}=h_{1k}$
\item Then it is designated as $c_1=h_{1k}$, the first functional component from the data, which contains the shortest period component of the signal. We separate it from the rest of the data: $X(t)-c_1 = r_1$, and the procedure is
repeated on $r_j: r_1-c_2 = r_2,\dots,r_{n-1} - c_n = r_n$
\end{enumerate}

The result is a set of functions called intrinsic mode functions (IMF); the number of functions in 
the set depends on the original signal~\cite{emd_process}.  An IMF is any 
function with the same number of extrema and zero crossings, with its envelopes being symmetric with respect to zero.
We run our correlation analysis on the shared IMF outputs between a pairs of traces.  In order to ensure 
that the IMFs corresponding to two distinct traces are on the same time scale, we use 
bivariate EMD \cite{rilling:biemd2007} to decompose two traces at once.

We use EMD to detrend each of the traces and pay particularly close attention to the high-frequency IMFs.  Our 
hypothesis is that correlating at the higher frequencies will yield more meaningful comparisons.
