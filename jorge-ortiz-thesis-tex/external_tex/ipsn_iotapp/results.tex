\subsection{Results}

We test our hypothesis in this section by using EMD to remove low-frequency trends in the data
and run correlation calculation at overlapping IMF timescales.  We discover that EMD allows us
to find and compare high-frequency instrinsic behavior that is spatially correlated across
sensors.  We begin with a small set of three sensors (EHP, GHP, light) and expand our scope
to include all the sensors in the dataset.



\subsubsection{Initial analysis}
Lets consider the simple example of Section \ref{problem} where we would like to know if the EHP trace is correlated with the two other traces.
Recall that the correlation coefficients of the raw feeds was $0.7715$ and $0.6370$, corresponding to the light 
and GHP, respectively.
As stated in previous section this result is correct but not so meaningful, since most of the traces
display the same diurnal pattern.
Figure \ref{fig:emd} and Figure \ref{fig:emd2} show the EMD decomposition of the three traces.
For each trace, EMD has retrieved three IMFs that highlight the higher frequencies of the traces.

Figure~\ref{fig:emd} shows the normalized raw trace (top) and EMD output IMFs and residual as well as the 
correlation coefficients calculated on the IMFs for the EHP and
light traces.  The correlation coefficients are $0.43909$, $0.49344$ and $0.63469$ corresponding to the IMF1, 
IMF2, and IMF3, respectively.  Notice the high correlation between the high-frequency IMFs.
We know that the light and EHP serve the same room, and their high-frequency IMF correlation corroborates
our prior knowledge.
Figure~\ref{fig:emd2} shows a complementary result, for the EHP and GHP comparison.
The correlation coefficients for the EHP and GHP IMFs suggest that the two may be independent.  In fact, they
\emph{are} indepdent; they serve completely different rooms in the building!

EMD allows us to remove low-frequency trends that add noise to the original analysis.
By comparing IMFs, we see both intrisically correlated and \emph{uncorrelated} behavior.  In the next
section we expand our analysis and show the effectiveness of our methodology. 
% Although promising, these results must be validated across the rest of the
% dataset to confirm their significance.  


\subsubsection{Validation}
To validate the effectiveness of our approach, we analyze the same three-week time span for \emph{all} 674 
sensors deployed in the building.
For each trace $S$ we compute two scores: (1) the correlation coefficient between $S$ and the EHP trace
and (2) the average value of the IMF correlation coefficients.

\begin{figure}[tbh!]
\centering
 \subfloat[Raw traces correlation coefficients]{\label{fig:histo1}\includegraphics[width=.43\textwidth]{figs/allFloors_week1_week4_corr_abs-eps-converted-to}}
 \subfloat[Average IMFs correlation coefficients]{\label{fig:histo2}\includegraphics[width=.43\textwidth]{figs/allFloors_week1_week4_emd_abs-eps-converted-to}}
 \caption{Distribution of the correlation coefficients of the raw traces and correlation coefficients average of the corresponding IMFs using 3 weeks of data from 674 sensors.}
\label{fig:histo}
\end{figure}

\begin{figure}
\centering
\includegraphics[width=.45\textwidth]{figs/floorMap.png}
\caption{Map of the floor where the analyzed EHP serves (room $C2$). The location of the sensors identified as related by the proposed approach are highlighted, showing a direct relationship between IMF correlation and spatial proximity.}
\label{fig:map}
\end{figure}

Figure \ref{fig:histo1} shows the distribution correlation coefficients.  Notice
that a large fraction of the dataset is correlated with the EHP trace.
\emph{Half} the traces have a correlation coefficient higher than $0.36$.  As expected, the underlying
trend is shared by a large number of device.
Although the highest score (i.e. $0.7715$) corresponds to the light in the same room that the EHP serves,
there are 118 pumps, serving all areas of the building, with a correlation higher than $0.6$.
Using only these results, it is not clear where the threshold should be set.  The distribution is close to 
uniform, making it difficult to 
know of how well your threshold discriminates against unrelated traces.
% Moreover, the distribution of the traces is almost uniform, thus, discriminating correlated traces is a laborious task.

Figure \ref{fig:histo2} shows the distribution of the average correlation value for the IMFs of
each trace and the EHP.  The number of traces correlated in the high frequency IMFs is significantly smaller
than the previous results. It's clear from the distribution that only a small set of devices are
\emph{intrinsically correlated} with the EHP.  In fact, \emph{only 10 traces out of 674} yielded a score higher than 
$0.25$. This allows us to easily rank traces by correlation.

Upon closer inspection of the 10 most correlated IMF traces, we find that there is a spatial relationship
between the EHP and the ten devices.  In fact, there is a direct relationship between score and distance from
the areas served by the EHP.  Figure~\ref{fig:map} shows a map of the floor that contains the rooms served by this
EHP.  The EHP directly serves room $C2$.  We introduce a correlation threshold to cluster correlated traces by score.
We highlight rooms by the threshold setting on the IMF correlation score.
When we set the threshold at $0.5$ we see that the sensors that have a correlation higher fall within room $C2$ --
the room served directly by the EHP.  As we relax the threshold, lowering it to $0.25$ and $0.1$ we see radial expansion from $C2$.  The trace with the highest score, $0.522$, is the trace corresponding to the lighting system \emph{in
the same room}.
The two highest scores for this floor (i.e. $0.316$ and $0.279$) are the light and EHP traces from next door, room $C1$.
Lower values correspond to sensors measuring activities in other rooms that have no specific relationship to the analyzed trace.  The results show a direct relationship between IMF correlation and spatial proximity and \emph{supports our initial
hypothesis}.


% Interestingly, the IMFs correlation coefficients reveal the spatial correlation of the sensors.
% Figure \ref{fig:map} is the map floor where the EHP trace is measured.
% Specifically, the EHP reports heating activity in the room $C2$.
% in the simple scenario the GHP is located in the room A5.


