\begin{abstract}
% % Buildings in the United State consume 41\% of the total energy produced.
% % In order to reduce the electrical footprint of buildings we present a new methodology to monitor building consumption and identify saving opportunities.
% % The proposed method uncovers the relationships between the building's electrical devices and determines the devices normal behavior.
% % Furthermore, by monitoring the devices relationships over time and comparing it to their usual relationships the misbehaving devices are automatically identified.
% % We demonstrate that these misbehaviors correspond to incorrect usages of the devices thus sources of electricity wastes.
% % The main challenge in this approach is to retrieve devices intrinsic-relationships from the sensor raw data.
% % Indeed these relationships are hidden by noise and common trends inherent to all the sensors in the building.
% % We overcome this issue by filtering the raw data using a recent signal processing technique: Empirical Mode Decomposition.
% % The proposed method is evaluated with 18 weeks of data from two buildings located in U.S. and Japan.
% % In spite of the national post-Fukushima measures to reduce the electricity consumption in Japan the proposed method found several saving opportunities in the Japanese dataset.
% % In the other dataset it also detected several electricity wastes that account for up to 2500~kWh.
% 
% A typical medium-to-large building %($\geq 10^{5}$ square-feet) 
% contains thousands of sensors, monitoring the HVAC system, lighting, and
% other operational sub-systems.  With the increased push for operational efficiency,
% operators are relying more on historical data processing to uncover opportunities for savings.  
% However, they are overwhelmed 
% with the deluge of data and seek more efficient ways to identify potential problems.  In this paper we present a new approach,
% called the Strip and Bind Method (SBM), for uncovering potential problems in equipment behavior and in-concert usage.  
% SBM uncovers relationships between devices and constructs a model for their in-concert usage.  It then flags deviations from the model as abnormal.
% 
% Unlike other approaches, SBM requires no a priori 
% knowledge about the building and runs on all sensors data -- treating each stream as a time-varying signal.
% % Since most sensors in the building are driven by the same underlying activity and weather patterns, we first had to remove
% % the dominant trend to obtain the instrinstic behavioral pattern.  Detrending is done using 
% % Empirical Mode Decomposition -- a recent signal processing tool.  We then construct a correlation-based model on the underlying 
% % intrinsic signals. 
% We run SBM on a set of building sensor traces; each containing up to several thousand sensors and over 500 GB of data, collected over
% 18 weeks from two separate buildings in different geographic locations and climates.  We demonstrate that, in many cases, SBM uncovers
% misbehaviors correspond to inefficient device usage that leads to energy waste.  The average waste uncovered is as high as 2500 kWh per 
% device.


A typical large building contains thousands of sensors, monitoring the HVAC system, lighting, and other operational sub-systems.
With the increased push for operational efficiency, operators are relying more on historical data processing to uncover opportunities for energy-savings.
However, they are overwhelmed with the deluge of data and seek more efficient ways to identify potential problems.
In this paper, we present a new approach called the Strip, Bind and Search (SBS); a method for uncovering abnormal 
equipment behavior and in-concert usage patterns.
SBS uncovers relationships between devices and constructs a model for their usage pattern relative to other devices.
It then flags deviations from the model. 
% Unlike other approaches, SBS requires no a priori knowledge about the building.
We run SBS on a set of building sensor traces; each containing hundred sensors reporting data flows over 18 weeks from two separate buildings with fundamentally different infrastructures.  
We demonstrate that, in many cases, SBS uncovers misbehavior corresponding to inefficient device usage that leads to energy waste.  
The average waste uncovered is as high as 2500~kWh per device. 
\end{abstract}
