\section{Data sets}
We evaluate SBS using data collected from buildings in two different geographic locations.  
One is a new building on main campus of the University of Tokyo and the other is an older building at 
the University of California, Berkeley.

\subsection{Engineering Building 2 - Todai}\label{data:engbldg2}
%The data from building 1 is collected at the Engineering Building 2 of the Hongo campus. 
Engineering building 2, at the University of Tokyo (Todai), is a 12-story building completed in 2005 and is now 
hosting classrooms, laboratories, offices 
and server rooms.  
The electricity consumption of the lighting and HVAC systems of 231 rooms is monitored by 135 sensors.
Rather than a centralized HVAC system, small, local HVAC systems are set up throughout the buidling.  
The HVAC systems are classified into two categories, EHP (Electrical Heat Pump) and GHP (Gas Heat Pump).
The GHPs are the only devices that serve numerous rooms and multiple floors.  The 5 GHPs in the dataset serve 154 rooms.
The EHP and lighting systems serve only pairs of rooms and which are directly controlled by the occupants.
In addition, the sensor metadata provides device-type and location information (room number), 
therefore, the electricity consumption of each pair of rooms is separately monitored.

The dataset contains 10 weeks of data starting from June 27, 2011 and ending on September 5, 2011.
This period of time is particularly interesting for two reasons: 1) in this region, the summer is the most energy-demanding 
season and 2) the building manager actively works to curtail energy usage as much as possible due to the 
Tohoku earthquake and Fukushima nuclear accident.

Furthermore, this dataset is a valuable ground truth to evaluate the Strip and Bind portions of SBS.
Since the light and HVAC of the rooms are directly controlled by the room's occupants, we expect SBS to uncover verifiable devices 
relationships.  
% In addition, we expect the anomaly detector to identify discontinuities in these relationships that represent obvious electricity saving opportunities (e.g. a room HVAC left on during night while the room lights have been turned off).

% Due to privacy concern this dataset is not publicly available on the Internet but accessible upon request.

\subsection{Cory Hall - UC Berkeley}
Cory Hall, at UC Berkeley, is a 5-story building hosting mainly classrooms, meeting rooms, laboratories and a datacenter.
This building was completed in 1950, thus its infrastructure is significantly different from the Japanese one.
The HVAC system in the building is centralized and serves several floors per unit.
There is a separate unit for an internal fabricated laboratory, inside the building.
%Nevertheless, we notice an independent HVAC system that was serving a particular laboratory; the Microfabrication Laboratory (Microlab).

This dataset consists of 8 weeks of energy consumption traces measured by 70 sensors starting on April $5^{th}$, 2011.
In contrast to the other dataset, a variety of devices are monitored, including, electric receptacles on certain floors, most of the HVAC components, 
 power panels and whole-building consumption.

These two building infrastructures are fundamentally different.  
This enables us to evaluate the practical efficacy of the proposed, unsupervised method in two very different environments.


\subsection{Data pre-processing}
Data pre-processing is not generally required for the proposed approach.  
Nevertheless, we observe in a few exceptional cases that sensors reporting excessively high values (i.e. values higher than the device actual capacity) that  greatly alter the performance of SBS by inducing a large bias in the computation of the correlation coefficient.
Therefore, we remove values that are higher than the maximum capacity of the devices, from the raw data.

% % In addition, to compare signals of the same length, the raw data is arranged such that the energy consumption of each device is reported every 5 minutes. 

