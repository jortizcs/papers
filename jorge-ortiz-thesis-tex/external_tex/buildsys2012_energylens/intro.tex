\subsection{Introduction}
The United States leads the world in per-capita energy consumption.
Our electricity use has consistently increased over the last 40 years~\cite{oecd2011} and other parts of the world are rising all 
too rapidly.  With the specter of climate change and the increasing cost of energy, we must explore new
ways for individuals to gain visibility and insight into their energy consumption in order to optimize and reduce it. 
With the increasing penetration of embedded sensors in the environment and
the continued rise in smartphone adoption, we see an opportunity for smartphones to bridge the physical world
to our computational infrastructure and provide an `energy lens' on the physical world.  

We use mobile phones to construct an entity-relationship 
graph of the physical world and combine it with streaming sensor data in order to perform detailed energy-attribution.
We limit the scope of the world to a single building domain.  We have designed and implemented a real-time, mobile energy auditing
application, called the `Energy Lens', that allows us to collect information about 
things throughout the building and how they are related to each other.  For example, computer X is inside 
room Y and connected to meter Z.  Then, we use these relationships to guide our data look-up and analytical
calculations.  For example, the load curve of room Y consists of the sum of all the power traces for loads
inside room Y.  We use the mobile smartphone as the main input tool.  Our work examines \emph{three main challenges} in setting up and 
deploying a real, whole-building infrastructure to support real-time, 
fined grained energy analytics.  

The first challenge is related to tracking and mobility.
The use of mobile phones presents classical, fundamental challenges related to mobility.  Typically, mobility
refers to the phone, as the person carrying it moves from place to place.  However, in the energy-attribution
context, we are also referring to the movement of energy-consuming objects.  Tracking their relationships to spaces 
and people is as important as tracking people.  We describe how we deal with \emph{both moving people and 
moving objects} and show that these historically difficult problems can be addressed relatively easily, if the proper infrastructure is 
in place.  %We provide evidence that the approach is simple, incrementally deployable, and scalable.

The second challenge is about capturing the inter-relationship semantics and having these inform our  analytics.
We adopt the general notion of physical tags that identify objects in the world.  Our system uses \emph{QR codes} to tag things and locations 
in the physical world.  However, \emph{any tag that provides a unqiue identifier for an object could serve the same purpose}.
Once tagged, there are three types of interactions -- 
registration, linking, and scanning -- which establish important relationships.  Registration is the act of creating a virtual object 
to represent a physical one.  Linking captures the relationship between pairs of objects.  Scanning is the act of performing an item-lookup.
Each of these interactions requires a set of swiping gestures.  Linking requires two tag swipes while the other two actions
require a single tag swipe.  Internally, we maintain a \emph{entity-relationship graph (ERG)} of things, people, and locations, that gets
updated through these sets of gestures.

The third challenge is about indoor network connectivity and access.
In order to connect these components, we rely on having `ubiquitous' network connectivity.  However, in practice, network
\emph{availability} is intermittent and our system must deal with the challenges of intermittency.  We discuss how caching
and logging are used to address these challenges.  Moreover, when connectivity is re-established, we must deal with
applying updates to the ERG, as captured by the phone while disconnected.  
% Conflicts can also occur during an update.  For example, the two updates may disagree about which items are attached
% to which meters.  We implement a very simple conflict resolution scheme, described in section~\ref{sec:conflicts}.
% Finally, certain physical-state transitions are represented as a set of updates to the ERG that must be applied 
% atomically.  We implement transactions in the log-replay and transaction manager.
% Our `Energy Lens' system is deployed in a building on our campus.  We discuss
% its architecture and our design choices.  
  
% We also discuss novel strategies for tracking moving people/things and describe how we implement these in our system.  In summary, our work
% makes the following contributions:

% \begin{itemize}
% \item We design and implement a system that captures and combines physical entities, their inter-relationships, and real-time sensor data 
% 		in buildings.% using mobile phones, qr code, and a cloud-based infrastructure.
% \item We observe that certain combinations of swipes give us useful information to set the location of people and things over time.
% 		We codify this observation in our \emph{context-tracker} and use it to maintain consistency between the entity-relationship graph and the 
% 		state of the physical world.  To the best of our knowledge, this is radically different from the approaches in standard 
% 		localization techniques.  However, we argue that it can be used to \emph{enhance} their accuracy and overall performance.
% \item We implement a prefetching algorithm to obtain context-dependent information to both improve performance and
% 		enable disconnected operation.  We also design and implement a log-replay and transaction manager over our data management layer.  We describe how different conflict-resolution policies can be implemented and our rationale for the policies we chose.
% \end{itemize}

% \vspace{0.08in}

% In the next sections we go through a motivating scenario.  We then discuss some related work, followed 
% by the system architecture, evaluation, and future directions.
